\input{../Preambulos/preambulo_materiales}
\usepackage{minted}

\author{M. en C. Gustavo Contreras Mayén. \texttt{gux7avo@ciencias.unam.mx}}
\title{Evaluación del Examen Final \\ {\large Curso Física Computacional}}
\date{ }
\begin{document}

\maketitle
\fontsize{14}{14}\selectfont

\Large{Becerril Hernández Gustavo.}

\section{Problema 1.}

\begin{enumerate}
\item Es muy favorable que aclares en qué orden se tienen los archivos, si son módulos o los archivos con la solución.
\item Hay un error en la llamada a tu funcion \texttt{metodo\_secante} ya que en lo que entiendo es tu módulo, la función se llama \texttt{secante}.
\item Hay errores de dedo en varias líneas de código que no permiten la ejecución.
\item Modificando esos errores hay otro
\begin{verbatim}
while (abs(x[-1]-x[-2])>delta):

TypeError: '>' not supported between instances of 'float' and 'function'
\end{verbatim}
¿Probaste tu código?
\item \textbf{Calificación: 0.4 puntos}
\end{enumerate}

\section{Problema 2.}

\begin{enumerate}
\item Se aclaró en las instrucciones para el examen, que utilizasen un archivo *.py más que un notebook de Jupyter, si hay 8 ejercicios en el examen final, se espera que hayan 8 archivos .py, cada uno con la solución a cada ejercicio. Esta metodología nunca se utilizó en el curso!
\item La función que mandas llamar en la celda \texttt{exacta\_barramasa} no existe en tu módulo, ahí tienes una función \texttt{exacta2\_barramasa}.
\item No me explico cómo ejecutaste el código sin haberte percatado de esta situación.
\item Usas la técnica de integración del trapecio, ¿por qué esa técnica con el error de integración mayor? Teniendo la regla de Simpson 1/3, la de 3/8, la integración de Romberg. Revisando debidamente el problema, no tendrías que \enquote{escoger aleatoriamente los valores de $n$}, que estrictamente, tu secuencia no es aleatoria!
\item Ajustando los errores en las llamadas a las funciones de tu módulo, el código no se ejecuta por que dentro, siguen los errores!!
\item \textbf{Calificación: 0 puntos}
\end{enumerate}

\section{Problema 3.}

\begin{enumerate}
\item Es la única celda que se ejecuta sin errores!
\item Llegas al valor de la raíz media cuadrática para la corriente, pero lo que muestras en la terminal ya es otro valor!
\item \textbf{Calificación: 0.7 puntos}
\end{enumerate}

\section{Problemas 4 - 8}

\begin{enumerate}
\item Sin evidencia de solución.
\item \textbf{Calificación: 0 puntos para cada problema}
\end{enumerate}

\end{document}