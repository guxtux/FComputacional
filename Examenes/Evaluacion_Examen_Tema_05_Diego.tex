\input{../Preambulos/preambulo_materiales}

\author{M. en C. Gustavo Contreras Mayén. \texttt{gux7avo@ciencias.unam.mx}}
\title{Evaluación del Examen Final \\ {\large Curso Física Computacional}}
\date{ }
\begin{document}

\maketitle
\fontsize{14}{14}\selectfont

\section{Problemas 1a y 1b.}

\begin{enumerate}
\item Bien por iniciar con una revisión de la función a integrar, aunque ya no es necesario que ocupases otros intervalos.
\item En la parte de ocupar el número de puntos aleatorios $n$ que va de $250$ a $\num{d4}$ muestras claramente el resultado, se puede incluir un ciclo \texttt{for} para que te devuelva las cuentas en un solo paso.
\item Hay un detalle en tus códigos para identificar las figuras, para separar las gráficas repites \texttt{figure(2)}, cuando debe de ser \texttt{figure(3)}. Procura revisar bien esos pasos.
\item \textbf{Calificación: 1 punto.}
\end{enumerate}

\section{Problema 2.}

\begin{enumerate}
\item Inicias debidamente la revisión del problema, el abordaje que planteas es una manera de verlo, aunque también debes de considerar que el hemisferio es un objeto con coordenadas $(x, y, z)$. Pero en ambos casos, se puede aprovechar al máximo la geometría del problema: en el caso 2d que propones, puedes revisar que basta con calcular la mitad de la figura que propones.
\item Implementas bien el algoritmo para el método del dardo.
\item La solución es consistente con lo esperado.
\item \textbf{Calificación: 1 punto.}
\end{enumerate}

\end{document}