\documentclass[12pt]{beamer}
\newenvironment{ConCodigo}[1]
  {\begin{frame}[fragile,environment=ConCodigo]{#1}}
  {\end{frame}}
\graphicspath{{Imagenes/}{../Imagenes/}}
\usepackage[utf8]{inputenc}
\usepackage[spanish]{babel}
\usepackage{hyperref}
\usepackage{etex}
\reserveinserts{28}
\usepackage{amsmath}
\usepackage{amsthm}
\usepackage{mathtools}
\usepackage{multicol}
\usepackage{multirow}
\usepackage{tabulary}
%\usepackage{tabularx}
\usepackage{booktabs}
\usepackage{nccmath}
\usepackage{biblatex}
\usepackage{epstopdf}
\usepackage{graphicx}
\usepackage{siunitx}
\sisetup{scientific-notation=true}
%\usepackage{fontspec}
\usepackage{lmodern}
\usepackage{float}
\usepackage[format=hang, font=footnotesize, labelformat=parens]{caption}
\usepackage[autostyle,spanish=mexican]{csquotes}
\usepackage{standalone}
\usepackage{tikz}
\usepackage[siunitx]{circuitikz}
\usetikzlibrary{arrows,patterns,shapes}
\usetikzlibrary{decorations.markings}
\usetikzlibrary{arrows}
\usepackage{color}
%\usepackage{beton}
%\usepackage{euler}
%\usepackage[T1]{fontenc}
\usepackage[sfdefault]{roboto}  %% Option 'sfdefault' only if the base font of the document is to be sans serif
\usepackage[T1]{fontenc}
\renewcommand*\familydefault{\sfdefault}
\DeclareGraphicsExtensions{.pdf,.png,.jpg}
\usepackage{hyperref}
\renewcommand {\arraystretch}{1.5}
\newcommand{\python}{\texttt{python}}
\usefonttheme[onlymath]{serif}
\setbeamertemplate{navigation symbols}{}
\usetikzlibrary{patterns}
\usetikzlibrary{decorations.markings}
\tikzstyle{every picture}+=[remember picture,baseline]
%\tikzstyle{every node}+=[inner sep=0pt,anchor=base,
%minimum width=2.2cm,align=center,text depth=.15ex,outer sep=1.5pt]
%\tikzstyle{every path}+=[thick, rounded corners]
\setbeamertemplate{caption}[numbered]
\newcommand{\ptm}{\fontfamily{ptm}\selectfont}
%Se usa la plantilla Warsaw modificada con spruce
\mode<presentation>
{
  \usetheme{Warsaw}
  \setbeamertemplate{headline}{}
  \useoutertheme{default}
  \usecolortheme{beaver}
  \setbeamercovered{invisible}
}
\AtBeginSection[]
{
\begin{frame}<beamer>{Contenido}
\normalfont\mdseries
\tableofcontents[currentsection]
\end{frame}
}

\usepackage{listings}
\lstset{ %
language=Python,                % choose the language of the code
basicstyle=\small,       % the size of the fonts that are used for the code
numbers=left,                   % where to put the line-numbers
numberstyle=\small,      % the size of the fonts that are used for the line-numbers
stepnumber=1,                   % the step between two line-numbers. If it is 1 each line will be numbered
numbersep=5pt,                  % how far the line-numbers are from the code
backgroundcolor=\color{white},  % choose the background color. You must add \usepackage{color}
showspaces=false,               % show spaces adding particular underscores
showstringspaces=false,         % underline spaces within strings
showtabs=false,                 % show tabs within strings adding particular underscores
frame=single,   		% adds a frame around the code
tabsize=2,  		% sets default tabsize to 2 spaces
captionpos=b,   		% sets the caption-position to bottom
breaklines=true,    	% sets automatic line breaking
breakatwhitespace=false,    % sets if automatic breaks should only happen at whitespace
escapeinside={\%},          % if you want to add a comment within your code
stringstyle =\color{magenta},
keywordstyle = \color{blue},
commentstyle = \color{green},
identifierstyle = \color{red}
}
\begin{document}
\title{Tema 2 - Operaciones matem\'{a}ticas b\'{a}sicas}
\subtitle{Integraci\\{o}n num\'{e}rica II}
%\subsubtitle{Curso de F\'{i}sica Computacional}
\author{M. en C. Gustavo Contreras May\'{e}n}
%\email{curso.fisica.comp@gmail.com}
%\ptsize{10}
\maketitle
\fontsize{14}{14}\selectfont
\spanishdecimal{.}
\begin{frame}{Contenido}
\tableofcontents[pausesections]
\end{frame}
\section{Reglas de Simpson}
\begin{frame}
\frametitle{Regla de $1/3$ de Simpson}
La regla de $1/3$ de Simpson se obtiene de las f\'{o}rmulas de Newton-Cotes con $n=2$, es decir, haciendo una interpolaci\'{o}n con una par\'{a}bola a trav\'{e}s de tres nodos, como se muestra en la siguiente figura:
\begin{center}
\begin{tikzpicture}
\draw [->] (0,0) -- node [near end, left]{$f(x)$} (0,3);
\draw [->] (0,0) -- (7,0);
\draw (7.4,0) node {x};
\draw [blue] (1,0.5) .. controls(2.5,2) .. (6,0.4);
\foreach \x in {1,1.5,...,6} \draw (\x,0) circle (0.03cm);
\draw (1,-0.3) node {$x_{0}$};
\draw (1,-0.7) node {a};
\draw (1.5,-0.3) node {$x_{1}$};
\draw (2,-0.3) node {$x_{2}$};
\draw (2.5,-0.3) node {$x_{3}$};
\draw (3,-0.3) node {$x_{4}$};
\draw (5.5,-0.3) node {$x_{n-1}$};
\draw (6,-0.3) node {$x_{n}$};
\draw (6,-0.7) node {b};
\draw [dashed] (1,0) -- (1,0.5);
\draw [dashed] (1.5,0) -- (1.5,1.05);
\draw [dashed] (2,0) -- (2,1.37);
\draw [dashed] (2.5,0) -- (2.5,1.6);
\draw [dashed] (3,0) -- (3,1.56);
\draw [dashed] (3.5,0) -- (3.5,1.48);
\draw [dashed] (4,0) -- (4,1.27);
\draw [dashed] (4.5,0) -- (4.5,1.09);
\draw [dashed] (5,0) -- (5,0.85);
\draw [dashed] (5.5,0) -- (5.5,0.65);
\draw [dashed] (6,0) -- (6,0.43);
\draw [<->] (2.5,1) -- node [midway, below] {h} (3,1);
\end{tikzpicture}
\end{center}
\end{frame}