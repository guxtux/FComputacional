\documentclass[12pt]{beamer}
\usepackage{../Estilos/BeamerFC}
\usepackage{../Estilos/ColoresLatex}
\input{../Preambulos/pre_codigo}
\input{../Preambulos/preambulo_Beamer_Dresden_seahorse}
\usefonttheme{serif}


\title{\large{Operaciones matemáticas básicas}}
\subtitle{¿Hacia dónde va el Tema 2?}
\author{M. en C. Gustavo Contreras Mayén}
\date{1/septiembre/2022}

\begin{document}
\maketitle

\section*{Contenido}
\frame{\tableofcontents[currentsection, hideallsubsections]}

\section{Iniciamos Tema 2}
\frame{\tableofcontents[currentsection, hideothersubsections]}
\subsection{Objetivos}

\begin{frame}
\frametitle{Objetivo General del Tema 2}
Al concluir el Tema 2 del curso, el alumno empleará las técnicas más comunes de operaciones matemáticas necesarias para la solución de problemas numéricos aplicados en la física.
\end{frame}
\begin{frame}
\frametitle{Contenido a revisar}
Este tema es una base importante de todo el curso, ya que se establecen las operaciones de:
\setbeamercolor{item projected}{bg=ao(english),fg=white}
\setbeamertemplate{enumerate items}{%
\usebeamercolor[bg]{item projected}%
\raisebox{1.5pt}{\colorbox{bg}{\color{fg}\footnotesize\insertenumlabel}}%
}
\begin{enumerate}[<+->]
\item Interpolación.
\item Cálculo de raíces.
\item Diferenciación.
\item Integración numérica.
\end{enumerate}
\end{frame}

\subsection*{Interpolación}

\begin{frame}
\frametitle{Interpolación}
\setbeamercolor{item projected}{bg=ashgrey,fg=burgundy}
\setbeamertemplate{enumerate items}{%
\usebeamercolor[bg]{item projected}%
\raisebox{1.5pt}{\colorbox{bg}{\color{fg}\footnotesize\insertenumlabel}}%
}
\begin{enumerate}[<+->]
\item Método de Lagrange.
\item Método de Newton.
\item Extrapolación.
\end{enumerate}
\end{frame}
\begin{frame}
\frametitle{Ejemplo de interpolación}
La viscosidad cinemática $\mu_{k}$ del agua varía con la temperatura $T$ de la siguiente manera:
\pause
\begin{table}[H]
\centering
\begin{tabular}{c | c | c | c | c | c | c | c }
$T$ & $0$ & $21.1$ & $37.8$ & $54.4$ & $71.1$ & $87.8$ & $100$ \\ \hline
$\mu_{k}$ & $1.79$ & $1.13$ & $0.696$ & $0.519$ & $0.338$ & $0.321$ & $0.296$ 
\end{tabular}
\end{table}
\pause
¿Cuál será el valor de $\mu_{k}$ para $T = 10^{\circ},30^{\circ},60^{\circ}$ y $90^{\circ}$?
\end{frame}
\begin{frame}
\frametitle{Los datos experimentales}
\begin{figure}
    \centering
    \includegraphics[scale=0.575]{Imagenes/Intro_Interpolacion_01.eps}
\end{figure}
\end{frame}
\begin{frame}
\frametitle{¿Cómo calcular los datos faltantes?}
\begin{figure}
    \centering
    \includegraphics[scale=0.575]{Imagenes/Intro_Interpolacion_02.eps}
\end{figure}
\end{frame}
\begin{frame}
\frametitle{Una interpolación lineal}
\begin{figure}
    \centering
    \includegraphics[scale=0.575]{Imagenes/Intro_Interpolacion_03.eps}
\end{figure}
\end{frame}
\begin{frame}
\frametitle{Una interpolación lineal}
\begin{figure}
    \centering
    \includegraphics[scale=0.575]{Imagenes/Intro_Interpolacion_04.eps}
\end{figure}
\end{frame}
\begin{frame}
\frametitle{Una interpolación cuadrática}
\begin{figure}
    \centering
    \includegraphics[scale=0.575]{Imagenes/Intro_Interpolacion_05.eps}
\end{figure}
\end{frame} 


\subsection*{Cálculo de raíces}

\begin{frame}
\frametitle{Cálculo de raíces}
\setbeamercolor{item projected}{bg=auburn,fg=aureolin}
\setbeamertemplate{enumerate items}{%
\usebeamercolor[bg]{item projected}%
\raisebox{1.5pt}{\colorbox{bg}{\color{fg}\footnotesize\insertenumlabel}}%
}
\begin{enumerate}[<+->]
\item Bisección.
\item Método de la secante.
\item Falsa posición.
\item Newton - Raphson.
\end{enumerate}
\end{frame}
\begin{frame}
\frametitle{Encontrando raíces de una función}
\begin{figure}
    \centering
    \includegraphics[scale=0.6]{Imagenes/Encuentra_Raices_01.eps}
\end{figure}
\end{frame}

\subsection*{Diferenciación}

\begin{frame}
\frametitle{Diferenciación numérica}
\setbeamercolor{item projected}{bg=babyblue,fg=cadmiumgreen}
Método de diferencias:
\setbeamertemplate{enumerate items}{%
\usebeamercolor[bg]{item projected}%
\raisebox{1.5pt}{\colorbox{bg}{\color{fg}\footnotesize\insertenumlabel}}%
}
\begin{enumerate}[<+->]
\item Hacia adelante.
\item Hacia atrás.
\item Centrales.
\item De orden mayor.
\end{enumerate}
\end{frame}

\subsection*{Integración}

\begin{frame}
\frametitle{Integración numérica}
\setbeamercolor{item projected}{bg=brightgreen,fg=brandeisblue}
\setbeamertemplate{enumerate items}{%
\usebeamercolor[bg]{item projected}%
\raisebox{1.5pt}{\colorbox{bg}{\color{fg}\footnotesize\insertenumlabel}}%
}
\begin{enumerate}[<+->]
\item Fórmulas Newton-Cotes.
\item Fórmulas Gaussianas.
\end{enumerate}
\end{frame}
\begin{frame}
\frametitle{Fórmulas Newton-Cotes}
\setbeamercolor{item projected}{bg=bronze,fg=bubblegum}
\setbeamertemplate{enumerate items}{%
\usebeamercolor[bg]{item projected}%
\raisebox{1.5pt}{\colorbox{bg}{\color{fg}\footnotesize\insertenumlabel}}%
}
\begin{enumerate}[<+->]
\item Trapecio.
\item Trapecio compuesta.
\item Trapecio recursiva.
\item Simpson $1/3$
\item Simpson $3/8$
\item Método de Romberg.
\end{enumerate}
\end{frame}
\begin{frame}
\frametitle{Aproximación polinomial de $f(x)$}
\begin{center}
    \begin{tikzpicture}[font=\footnotesize, scale=1.25]
        \draw [->] (0,0) -- node [near end, left]{$f(x)$} (0,3);
        \draw [->] (0,0) -- (7,0);
        \draw (7.4,0) node {x};
        \draw [blue] (1,0.5) .. controls(2.5,2) .. (6,0.4);
        \foreach \x in {1,1.5,...,6} \draw (\x,0) circle (0.03cm);
        \draw (1,-0.3) node {$x_{0}$};
        \draw (1,-0.7) node {a};
        \draw (1.5,-0.3) node {$x_{1}$};
        \draw (2,-0.3) node {$x_{2}$};
        \draw (2.5,-0.3) node {$x_{3}$};
        \draw (3,-0.3) node {$x_{4}$};
        \draw (5.5,-0.3) node {$x_{n-1}$};
        \draw (6,-0.3) node {$x_{n}$};
        \draw (6,-0.7) node {b};
        \draw [dashed] (1,0) -- (1,0.5);
        \draw [dashed] (1.5,0) -- (1.5,1.05);
        \draw [dashed] (2,0) -- (2,1.37);
        \draw [dashed] (2.5,0) -- (2.5,1.6);
        \draw [dashed] (3,0) -- (3,1.56);
        \draw [dashed] (3.5,0) -- (3.5,1.48);
        \draw [dashed] (4,0) -- (4,1.27);
        \draw [dashed] (4.5,0) -- (4.5,1.09);
        \draw [dashed] (5,0) -- (5,0.85);
        \draw [dashed] (5.5,0) -- (5.5,0.65);
        \draw [dashed] (6,0) -- (6,0.43);
        \draw [<->] (2.5,1) -- node [midway, below] {h} (3,1);
    \end{tikzpicture}
\end{center}
\end{frame}


\begin{frame}
\frametitle{Fórmulas Gaussianas}
\setbeamercolor{item projected}{bg=cadetblue,fg=bubbles}
\setbeamertemplate{enumerate items}{%
\usebeamercolor[bg]{item projected}%
\raisebox{1.5pt}{\colorbox{bg}{\color{fg}\footnotesize\insertenumlabel}}%
}
\begin{enumerate}[<+->]
\item Polinomios ortogonales.
\item Gauss-Chebyshev.
\item Gauss-Hermite.
\item Gauss-Legendre.
\end{enumerate}
\end{frame}

\section{Programas matemáticos}
\frame{\tableofcontents[currentsection, hideothersubsections]}
\subsection{Otro software}

\begin{frame}
\frametitle{¿Es necesario python?}
Si bien es cierto que existen programas o softwares tanto de pago como de tipo GNU, enfocados al cálculo numérico, \pause nos podemos preguntar: ¿por qué hacerlo con \python?
\end{frame}
\begin{frame}
\frametitle{Software matemático}
\begin{figure}
    \centering
    \only<1>{\includegraphics[scale=0.35]{Imagenes/gnu-octave-logo-lnx.png}}
    \only<2>{\includegraphics[scale=0.35]{Imagenes/scilab-computer.png}}
    \only<3>{\includegraphics[scale=0.5]{Imagenes/Wolfram-logo.png}}
    \only<4>{\includegraphics[scale=0.35]{Imagenes/logo-MATLAB.png}}
    \only<5>{\includegraphics[scale=0.15]{Imagenes/julia-computing_hd-logo.png}}
    \only<6>{\includegraphics[scale=0.7]{Imagenes/java-logo.png}}
\end{figure}
\end{frame}
\begin{frame}
\frametitle{Sintaxis en cada programa}
Cada programa/lenguaje requiere conocer su propia sintaxis y con ello, es posible resolver cada uno de los ejercicios que veremos.
\end{frame}
\begin{frame}
\frametitle{Ventaja académica}
La gran ventaja académica que tenemos al revisar la base conceptual de los procedimientos, \pause es que podremos reproducirlos en cualquiera de los lenguajes que tengamos a disposición.
\\
\bigskip
\pause
Mientras que si aprendemos la sintaxis propia, dependeremos exclusivamente de esa herramienta.
\end{frame}

\end{document}