\documentclass[12pt]{beamer}
\usepackage{../Estilos/BeamerFC}
\usepackage{../Estilos/ColoresLatex}
\input{../Preambulos/pre_codigo}
\input{../Preambulos/preambulo_Beamer_Dresden_seahorse}
\usefonttheme{serif}


\title{\large{Operaciones matemáticas básicas}}
\subtitle{¿Hacia dónde va el Tema 2?}
\author{M. en C. Gustavo Contreras Mayén}
\date{1/septiembre/2022}

\begin{document}
\maketitle

\section*{Contenido}
\frame{\tableofcontents[currentsection, hideallsubsections]}

\section{Iniciamos Tema 2}
\frame{\tableofcontents[currentsection, hideothersubsections]}
\subsection{Objetivos}

\begin{frame}
\frametitle{Objetivo General del Tema 2}
Al concluir el Tema 2 del curso, el alumno empleará las técnicas más comunes de operaciones matemáticas necesarias para la solución de problemas numéricos aplicados en la física.
\end{frame}
\begin{frame}
\frametitle{Contenido a revisar}
Este tema es una base importante de todo el curso, ya que se establecen las operaciones de:
\setbeamercolor{item projected}{bg=ao(english),fg=white}
\setbeamertemplate{enumerate items}{%
\usebeamercolor[bg]{item projected}%
\raisebox{1.5pt}{\colorbox{bg}{\color{fg}\footnotesize\insertenumlabel}}%
}
\begin{enumerate}[<+->]
\item Interpolación.
\item Cálculo de raíces.
\item Diferenciación.
\item Integración numérica.
\end{enumerate}
\end{frame}

\subsection*{Interpolación}

\begin{frame}
\frametitle{Interpolación}
\setbeamercolor{item projected}{bg=ashgrey,fg=burgundy}
\setbeamertemplate{enumerate items}{%
\usebeamercolor[bg]{item projected}%
\raisebox{1.5pt}{\colorbox{bg}{\color{fg}\footnotesize\insertenumlabel}}%
}
\begin{enumerate}[<+->]
\item Método de Lagrange.
\item Método de Newton.
\item Extrapolación.
\end{enumerate}
\end{frame}
\begin{frame}
\frametitle{Ejemplo de interpolación}
La viscosidad cinemática $\mu_{k}$ del agua varía con la temperatura $T$ de la siguiente manera:
% \begin{table}[H]
% \centering
% \begin{large} 
% \begin{tabular}{15cm}{c | c | c | c | c | c | c | c }
% $T(^\circ C)$ & $0$ & $21.1$ & $37.8$ & $54.4$ & $71.1$ & $87.8$ & $100$ \\
% \midrule
% $\mu_{k} (10^{-3}m^{2}/s)$ & $1.79$ & $1.13$ & $0.696$ & $0.519$ & $0.338$ & $0.321$ & $0.296$ 
% \end{tabular}
% \end{large}
% \end{table}
Interpolar $\mu_{k}$ para $T= 10^{\circ},30^{\circ},60^{\circ}$ y $90^{\circ}$.
\end{frame}
\begin{frame}
\frametitle{Los datos experimentales}
\begin{figure}
    \centering
    \includegraphics[scale=0.575]{Imagenes/Intro_Interpolacion_01.eps}
\end{figure}
\end{frame}
\begin{frame}
\frametitle{¿Cómo calcular los datos faltantes?}
\begin{figure}
    \centering
    \includegraphics[scale=0.575]{Imagenes/Intro_Interpolacion_02.eps}
\end{figure}
\end{frame}
\begin{frame}
\frametitle{una interpolación lineal}
\begin{figure}
    \centering
    \includegraphics[scale=0.575]{Imagenes/Intro_Interpolacion_03.eps}
\end{figure}
\end{frame}
\begin{frame}
\frametitle{Una interpolación lineal}
\begin{figure}
    \centering
    \includegraphics[scale=0.575]{Imagenes/Intro_Interpolacion_04.eps}
\end{figure}
\end{frame}
\begin{frame}
\frametitle{Una interpolación cuadrática}
\begin{figure}
    \centering
    \includegraphics[scale=0.575]{Imagenes/Intro_Interpolacion_05.eps}
\end{figure}
\end{frame} 


\subsection*{Cálculo de raíces}

\begin{frame}
\frametitle{Cálculo de raíces}
\setbeamercolor{item projected}{bg=auburn,fg=aureolin}
\setbeamertemplate{enumerate items}{%
\usebeamercolor[bg]{item projected}%
\raisebox{1.5pt}{\colorbox{bg}{\color{fg}\footnotesize\insertenumlabel}}%
}
\begin{enumerate}[<+->]
\item Bisección.
\item Método de la secante.
\item Falsa posición.
\item Newton - Raphson.
\end{enumerate}
\end{frame}

\subsection*{Diferenciación}

\begin{frame}
\frametitle{Diferenciación numérica}
\setbeamercolor{item projected}{bg=babyblue,fg=cadmiumgreen}
Método de diferencias:
\setbeamertemplate{enumerate items}{%
\usebeamercolor[bg]{item projected}%
\raisebox{1.5pt}{\colorbox{bg}{\color{fg}\footnotesize\insertenumlabel}}%
}
\begin{enumerate}[<+->]
\item Hacia adelante.
\item Hacia atrás.
\item Centrales.
\item De orden mayor.
\end{enumerate}
\end{frame}

\subsection*{Integración}

\begin{frame}
\frametitle{Integración numérica}
\setbeamercolor{item projected}{bg=brightgreen,fg=brandeisblue}
\setbeamertemplate{enumerate items}{%
\usebeamercolor[bg]{item projected}%
\raisebox{1.5pt}{\colorbox{bg}{\color{fg}\footnotesize\insertenumlabel}}%
}
\begin{enumerate}[<+->]
\item Fórmulas Newton-Cotes.
\item Fórmulas Gaussianas.
\end{enumerate}
\end{frame}
\begin{frame}
\frametitle{Fórmulas Newton-Cotes}
\setbeamercolor{item projected}{bg=bronze,fg=bubblegum}
\setbeamertemplate{enumerate items}{%
\usebeamercolor[bg]{item projected}%
\raisebox{1.5pt}{\colorbox{bg}{\color{fg}\footnotesize\insertenumlabel}}%
}
\begin{enumerate}[<+->]
\item Trapecio.
\item Trapecio compuesta.
\item Trapecio recursiva.
\item Simpson $1/3$
\item Simpson $3/8$
\item Método de Romberg.
\end{enumerate}
\end{frame}
\begin{frame}
\frametitle{Fórmulas Gaussianas}
\setbeamercolor{item projected}{bg=cadetblue,fg=bubbles}
\setbeamertemplate{enumerate items}{%
\usebeamercolor[bg]{item projected}%
\raisebox{1.5pt}{\colorbox{bg}{\color{fg}\footnotesize\insertenumlabel}}%
}
\begin{enumerate}[<+->]
\item Polinomios ortogonales.
\item Gauss-Chebyshev.
\item Gauss-Hermite.
\item Gauss-Legendre.
\end{enumerate}
\end{frame}

\end{document}