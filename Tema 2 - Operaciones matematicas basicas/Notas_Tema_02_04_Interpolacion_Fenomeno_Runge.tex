\documentclass[12pt]{beamer}
\usepackage{../Estilos/BeamerFC}
\usepackage{../Estilos/ColoresLatex}
\input{../Preambulos/pre_codigo}
\input{../Preambulos/preambulo_Beamer_Dresden_seahorse}
%\usefonttheme{serif}

\title{Técnica de interpolación de Neville}
\subtitle{Tema 2 - Operaciones matemáticas básicas}
\author{M. en C. Gustavo Contreras Mayén}
\date{}

\begin{document}
\maketitle

\section*{Contenido}
\frame{\tableofcontents[currentsection, hideallsubsections]}

\section{Interpolación de Neville}
\frame{\tableofcontents[currentsection, hideothersubsections]}
\subsection{Describiendo el método}

\begin{frame}
\frametitle{Sobre el método de Newton}
El método de interpolación de Newton implica dos pasos:
\pause
\setbeamercolor{item projected}{bg=alizarin,fg=white}
\setbeamertemplate{enumerate items}{%
\usebeamercolor[bg]{item projected}%
\raisebox{1.5pt}{\colorbox{bg}{\color{fg}\footnotesize\insertenumlabel}}%
}
\begin{enumerate}[<+->]
\item El cálculo de los coeficientes.
\item Seguido de evaluación del polinomio.
\end{enumerate}
\end{frame}
\begin{frame}
\frametitle{Sobre el método de Newton}
Esta técnica funciona bien si la interpolación se realiza \textocolor{ao(english)}{repetidamente en diferentes valores} de $x$ utilizando el \textocolor{arsenic}{mismo polinomio}.
\end{frame}
\begin{frame}
\frametitle{El método de Neville}
Si solo se \textocolor{bluegray}{interpola un punto}, una opción más conveniente es un método que calcule el interpolante en un solo paso, \pause como lo es el \textocolor{blush}{algoritmo de Neville}.
\end{frame}
\begin{frame}
\frametitle{El método de Neville}
Sea 
\begin{align*}
P_{k} \big[ x_{i} , x_{i+1}, \ldots, x_{i+k} \big]
\end{align*}
el polinomio de grado $k$ que pasa por los $k + 1$ puntos de datos:
\pause
\begin{align*}
(x_{i}, y_{i}), \, (x_{i+1}, y_{i+1}), \ldots , (x_{i+k}, y_{i+k})
\end{align*}
\end{frame}
\begin{frame}
\frametitle{El método de Neville}
Para un único punto, se tiene que:
\pause
\begin{align}
P_{0} [x_{i}] = y_{i} \label{eq:3_07}
\end{align}
\end{frame}
\begin{frame}
\frametitle{Interpolación de Neville con dos puntos}
La interpolación basada en dos puntos es:
\pause
\begin{align*}
P_{1} [x_{i}, x_{i+1}] = \dfrac{(x - x_{i+1}) P_{0} [x_{i}] + (x_{i} - x) P_{0} [x_{i+1}]}{x_{i} - x_{i+1}}
\end{align*}
\end{frame}
\begin{frame}
\frametitle{Interpolación de Neville con tres puntos}
La interpolación basada en tres puntos es:
\pause
\begin{align*}
&P_{2} [x_{i}, x_{i+1}, x_{i+2}] = \\[0.5em]
&= \dfrac{(x - x_{i+2}) P_{1} [x_{i}, x_{i+1}] + (x_{i} - x) P_{1} [x_{i+1}, x_{i+2}]}{x_{i} - x_{i+2}}
\end{align*}
\end{frame}
\begin{frame}
\frametitle{Patrón para la interpolación de Neville}
De las interpolaciones anteriores, podemos deducir una fórmula recursiva general para la interpolación de Neville:
\end{frame}
\begin{frame}
\frametitle{Patrón para la interpolación de Neville}
\begin{align}
\begin{aligned}
&P_{k} [x_{i}, x_{i+1}, \ldots, x_{i+k}] = \\[0.5em]
&= \dfrac{(x - x_{i+k}) P_{k-1} [x_{i}, x_{i+1}, \ldots, x_{i+k-1}]}{x_{i} - x_{i+k}} + \\[0.5em]
&+ \dfrac{(x_{i} - x) P_{k-1} [x_{i+1}, x_{i+2}, \ldots, x_{i+k}]}{x_{i} - x_{i+k}}
\end{aligned}
\label{eq:3_08}
\end{align}
\end{frame}
\begin{frame}
\frametitle{El método de Neville}
Dado el valor de $x$, los cálculos se pueden realizar en el siguiente formato tabular (que se muestra para cuatro puntos de datos):
\end{frame}
\begin{frame}
\frametitle{Formato tabular}
\begin{table}
\centering
\fontsize{12}{12}\selectfont
\begin{tabular}{| c | c | c | c | c |} \hline
 & $k = 0$ & $k = 1$ & $k = 2$ & $k = 3$ \\ \hline
$x_{0}$ & $P_{0} [x_{0}] = y_{0}$ & $P_{1} [x_{0}, x_{1}]$ & $P_{2} [x_{0}, x_{1}, x_{2}]$ & $P_{3} [x_{0}, x_{1}, x_{2}, x_{3}]$ \\ \hline
$x_{1}$ & $P_{0} [x_{1}] = y_{1}$ & $P_{1} [x_{1}, x_{2}]$ & $P_{2} [x_{1}, x_{2}, x_{3}]$ & \\ \hline
$x_{2}$ & $P_{0} [x_{2}] = y_{2}$ & $P_{1} [x_{2}, x_{3}]$ & & \\ \hline
$x_{3}$ & $P_{0} [x_{3}] = y_{3}$ & & & \\ \hline	
\end{tabular}
\end{table}
\end{frame}
\begin{frame}
\frametitle{El algoritmo de Neville}
El algoritmo que calcula los elementos de la tabla, que denota el número de puntos de datos por $m$, es:
\end{frame}
\begin{frame}[fragile]
\frametitle{El algoritmo de Neville}
\begin{lstlisting}[caption=Algoritmo de Neville]
y = yDatos.copy()
for k in range (1,m):
	for i in range(m-k):
		y[i] = ((x - xDatos[i+k])*y[i] + (xDatos[i] - x)*y[i+1])/ (xDatos[i] - xDatos[i+k])
\end{lstlisting}
\end{frame}
\begin{frame}
\frametitle{Entendiendo el algoritmo}
Este algoritmo funciona con arreglo unidimensional \textocolor{ao}{y}, que inicialmente contiene los valores de $y$, la segunda columna de la tabla anterior.
\end{frame}
\begin{frame}
\frametitle{Entendiendo el algoritmo}
Cada paso por el bucle exterior calcula los elementos de \textocolor{ao}{y} en la siguiente columna, \pause que sobrescriben las entradas anteriores.
\end{frame}
\begin{frame}
\frametitle{Entiendiendo el algoritmo}
Al final del procedimiento, el arreglo \textocolor{ao}{y} contiene los términos diagonales de la tabla.
\end{frame}
\begin{frame}
\frametitle{Entendiendo el algoritmo}
El valor de interpolación (evaluado en \textocolor{ao}{x}) que pasa por todos los puntos de datos es el primer elemento del arreglo \textocolor{ao}{y}.
\end{frame}
\begin{frame}
\frametitle{Manejo en el módulo}
Se recomienda que la siguiente función la incluyas en un módulo llamado \textocolor{ao}{interpolacion}, \pause para que de esa manera tengas las funciones de interpolación de Lagrange, Newton y Neville.
\end{frame}
\begin{frame}[fragile]
\frametitle{El código de interpolación}
\begin{lstlisting}[caption=La función interpolación de Neville]
def neville(xDatos, yDatos, x):
	m = len(xDatos)
	y = yDatos.copy()
	for k in range(1,m):
		y[0:m-k] = ((x - xDatos[k:m])*y[0:m-k] + \
					(xDatos[0:m-k] - x)*y[1:m-k+1])/ \
					(xDatos[0:m-k] - xDatos[k:m])

	return y[0]
\end{lstlisting}
\end{frame}
\begin{frame}
\frametitle{Enunciado del ejercicio}
Dados los siguientes puntos:
\begin{table}
\centering
\begin{tabular}{| c | c | c | c | c |}
$x$ & $4.0$ & $3.9$ & $3.8$ & $3.7$ \\ \hline
$y$ & $-0.06604$ & $-0.02724$ & $0.01282$ & $0.05382$ \\	
\end{tabular}
\end{table}
Con el método de Neville, calcula el valor de la raíz $y (x) = 0$.
\end{frame}
\begin{frame}
\frametitle{Consideración importante}
Este es un ejemplo de \textocolor{red}{interpolación inversa}, \pause en el que se intercambian los roles de \textocolor{ao}{x} e \textocolor{ao}{y}.
\end{frame}
\begin{frame}
\frametitle{Consideración importante}
En lugar de calcular \textocolor{ao}{y} en un \textocolor{ao}{x} dado, \pause encontramos \textocolor{ao}{x} que corresponde a un \textocolor{ao}{y} dado (en este caso, y = 0).
\end{frame}
\begin{frame}
\frametitle{El código en un archivo}
Revisa con cuidado el código con la implementación de la solución. 
\end{frame}

\subsection{Ejercicios a resolver}

\begin{frame}
\frametitle{Enunciado del ejercicio 1}
Con el método de Neville calcula el valor de $y$ en $x = \pi /4$, a partir del conjunto de puntos:
\begin{table}
\centering
\begin{tabular}{| c | c | c | c | c | c |}
$x$ & $0$ & $0.5$ & $1$ & $1.5$ & 2 \\ \hline
$y$ & $-1.00$ & $1.75$ & $4.00$ & $5.75$ & $7.00$ \\	
\end{tabular}
\end{table}
\end{frame}
\begin{frame}
\frametitle{Enunciado del ejercicio 2}
Dados los puntos:
\begin{table}
\centering
\begin{tabular}{| c | c | c | c |}
$x$ & $-1.2$ & $0.3$ & $1.1$ \\ \hline
$y$ & $-5.76$ & $5.61$ & $-.69$ \\	
\end{tabular}
\end{table}
Calcula el valor de $y$ en $x = 0$, usando el método de Neville y el método de Lagrange.
\end{frame}

\section{Fenómeno de Runge}
\frame{\tableofcontents[currentsection, hideothersubsections]}
\subsection{Definición}

\begin{frame}
\frametitle{Fenónemo de Runge}
Hasta el momento hemos revisado un par de estrategias para calcular un polinomio que pase por un conjunto de datos $(x_{i}, y_{i})$
\end{frame}
\begin{frame}
\frametitle{Fenónemo de Runge}
Pero hay que considerar un efecto importante al respecto: \textocolor{burntorange}{no siempre el mejor polinomio será aquel el de mayor grado $n$}.
\\
\medskip
\pause
Veamos el siguiente ejemplo: sea la función:
\begin{align*}
f (x) = \dfrac{1}{1 + 25 x^{2}}
\end{align*}
\end{frame}
\begin{frame}
\frametitle{Gráfica de la función $f(x)$}
\begin{figure}
	\centering
	\includegraphics[scale=0.47]{Imagenes/Funcion_Runge_01.eps} 
\end{figure}
\end{frame}
\begin{frame}
\frametitle{Elección de puntos para interpolar}
\begin{figure}
	\centering
	\includegraphics[scale=0.47]{Imagenes/Funcion_Runge_02.eps} 
\end{figure}
\end{frame}
\begin{frame}
\frametitle{Interpolando con el método de Lagrange}
\begin{figure}
	\centering
	\includegraphics[scale=0.45]{Imagenes/Funcion_Runge_03.eps} 
\end{figure}
\end{frame}
\begin{frame}
\frametitle{¿Qué hacemos al respecto?}
Como hemos visto en la gráfica anterior, la función que resulta del proceso de interpolación \enquote{oscila} a través de los puntos que deseamos interpolar
\end{frame}
\begin{frame}
\frametitle{El efecto Runge}
El caso es que si aumentamos el grado del polinomio, los resultados serán aún más indeseables.
\\
\medskip
\pause
A este efecto se le conoce como el \textocolor{carnelian}{efecto Runge}.
\end{frame}
\begin{frame}
\frametitle{¿Qué hacemos al respecto?}
La pregunta obligada es: \textocolor{cerulean}{¿qué podemos hacer para mejorar la interpolación?}
\end{frame}

\section{Uso de splines}
\frame{\tableofcontents[currentsection, hideothersubsections]}
\subsection{Definición}

\begin{frame}
\frametitle{¿Qué es un spline?}
En términos nada riguroso, se puede decir que un \textocolor{blue}{spline} es una función definida por una familia de polinomios \enquote{sociables}.
\\
\bigskip
\pause
Donde el término \textit{sociable} se usa para indicar que los polinomios que constituyen una función \textocolor{blue}{spline}, están estrechamente vinculados.
\end{frame}
\begin{frame}
\frametitle{El nomobre de spline}    
El nombre de \textocolor{blue}{spline}, viene del inglés ya que es un instrumento que utilizaban los ingenieros navales para dibujar curvas suaves, forzadas a pasar por un conjunto de puntos prefijados.
\end{frame}
\begin{frame}
\frametitle{Las tres B's de los splines}
El uso de las funciones splines tiene mucha aceptación y popularidad se deben a tres razones básicas:
\pause
\setbeamercolor{item projected}{bg=beige,fg=black}
\setbeamertemplate{enumerate items}{%
\usebeamercolor[bg]{item projected}%
\raisebox{1.5pt}{\colorbox{bg}{\color{fg}\footnotesize\insertenumlabel}}%
}
\begin{enumerate}[<+->]
\item \textocolor{copper}{Buenos:} Se  pueden usar en la solución de una gran variedad de problemas.
\item \textocolor{coquelicot}{Bonitos:} La teoría matemática en que se basan es muy simple y a la vez elegante.
\item \textocolor{cordovan}{Baratos:} Su cálculo es muy sencillo y económico.
\end{enumerate}
\end{frame}
\begin{frame}
\frametitle{Manejando splines con \python}
Usaremos la librería \funcionazul{scipy} que contiene varias funciones con las que ahorramos tiempo para manejar splines y ajustar funciones sociables a un conjunto de datos.
\end{frame}
\begin{frame}
\frametitle{Manejando splines con \python}
No está de más que revises la teoría al respecto, en la mayoría de los libros de análisis numérico, podrás encontrar la construcción matemática y formal de los splines.
\end{frame}
\begin{frame}
\frametitle{Funciones para los splines}
Necesitaremos de dos pasos para el uso de splines con \python, de la libería: \funcionazul{scipy.interpolate}, las cuales son:
\pause
\setbeamercolor{item projected}{bg=lilac,fg=white}
\setbeamertemplate{enumerate items}{%
\usebeamercolor[bg]{item projected}%
\raisebox{1.5pt}{\colorbox{bg}{\color{fg}\footnotesize\insertenumlabel}}%
}
\begin{enumerate}[<+->]
\item \funcionazul{splrep}: Calcula el spline básico (B-spline) para una curva 1-D.
\\
\medskip
Dados un conjunto de puntos $(x[i],y[i])$ determina una aproximación con un spline suave de grado $k$ en el intervalo $xb \leq x \leq xe$.
\seti
\end{enumerate}
\end{frame}
\begin{frame}
\frametitle{Funciones para los splines}
\setbeamercolor{item projected}{bg=lilac,fg=white}
\setbeamertemplate{enumerate items}{%
\usebeamercolor[bg]{item projected}%
\raisebox{1.5pt}{\colorbox{bg}{\color{fg}\footnotesize\insertenumlabel}}%
}
\begin{enumerate}[<+->]
\conti    
\item \textbf{splev}: Evalúa un B-spline o sus derivadas. Dados los nodos y coeficientes de un B-spline, calcula 
el valor del polinomio suave y sus derivadas.
\end{enumerate}
\end{frame}
% \begin{frame}[allowframebreaks, fragile]
% \frametitle{Código}
% \begin{lstlisting}
% import matplotlib.pyplot as plt
% import scipy.interpolate as si
% from numpy import linspace

% def trazador_cub(n):
%     xi = linspace(-1, 1, n)
%     yi = 1./(1 + 25*xi**2)
%     tck = si.splrep(xi, yi)
%     return tck

% x = linspace(-1, 1, 100)
% y = 1./(1 + 25*x**2)

% tck = trazador_cub(8)
% ys8 = si.splev(x, tck)

% tck = trazador_cub(12)
% ys12 = si.splev(x, tck)

% plt.plot(x, y)
% plt.plot(x, ys8,'+g-', label='n=8')
% plt.plot(x, ys12,'+r-',label ='n=12')
% plt.legend(loc='best')
% plt.title('Interpolacion con splines cubicos')
% plt.ylim(-0.2, 1.2)
% plt.show()
% \end{lstlisting}
% \end{frame}
\begin{frame}
\frametitle{Código implementado}
Revisa el archivo con el código que resuelve el problema con splines cúbicos.
\end{frame}
\begin{frame}
\frametitle{Resultados gráficos}
\begin{figure}
	\centering
	\includegraphics[scale=0.47]{Imagenes/Funcion_Runge_04.eps} 
\end{figure}
\end{frame}
\end{document}
