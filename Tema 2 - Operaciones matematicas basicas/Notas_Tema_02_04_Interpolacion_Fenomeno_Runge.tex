\documentclass[12pt]{beamer}
\usepackage{../Estilos/BeamerFC}
\usepackage{../Estilos/ColoresLatex}
\input{../Preambulos/pre_codigo}
\input{../Preambulos/preambulo_Beamer_Dresden_seahorse}
\usefonttheme{serif}

\title{ Consideraciones sobre las técnicas de interpolación}
\subtitle{Tema 2 - Operaciones matemáticas básicas}
\author{M. en C. Gustavo Contreras Mayén}
\date{}

\begin{document}
\maketitle

\section*{Contenido}
\frame{\tableofcontents[currentsection, hideallsubsections]}

\section{Fenómeno de Runge}
\frame{\tableofcontents[currentsection, hideothersubsections]}
\subsection{Definición}

\begin{frame}
\frametitle{Fenónemo de Runge}
Hasta el momento hemos revisado un par de estrategias para calcular un polinomio que pase por un conjunto de datos $(x_{i}, y_{i})$, pero hay que considerar un efecto importante al respecto: no siempre el mejor polinomio será aquel el de mayor grado $n$.
\\
\medskip
\pause
Veamos el siguiente ejemplo: sea la función:
\begin{align*}
f (x) = \dfrac{1}{1 + 25 x^{2}}
\end{align*}
\end{frame}
\begin{frame}
\frametitle{Gráfica de la función $f(x)$}
\begin{figure}
	\centering
	\includegraphics[scale=0.47]{Imagenes/Funcion_Runge_01.eps} 
\end{figure}
\end{frame}
\begin{frame}
\frametitle{Elección de puntos para interpolar}
\begin{figure}
	\centering
	\includegraphics[scale=0.47]{Imagenes/Funcion_Runge_02.eps} 
\end{figure}
\end{frame}
\begin{frame}
\frametitle{Interpolando con el método de Lagrange}
\begin{figure}
	\centering
	\includegraphics[scale=0.45]{Imagenes/Funcion_Runge_03.eps} 
\end{figure}
\end{frame}
\begin{frame}
\frametitle{¿Qué hacemos al respecto?}
Como hemos visto en la gráfica anterior, la función que resulta del proceso de interpolación \enquote{oscila} a través de los puntos que deseamos interpolar; aquí el caso es que si aumentamos el grado del polinomio, los resultados serán aún más indeseables.
\\
\medskip
\pause
La pregunta obligada es: \textbf{¿qué podemos hacer para mejorar la interpolación?}
\end{frame}

\section{Uso de splines}
\frame{\tableofcontents[currentsection, hideothersubsections]}
\subsection{Definición}

\begin{frame}
\frametitle{¿Qué es un spline?}
En términos nada riguroso, se puede decir que un \textcolor{blue}{spline} es una función definida por una familia de polinomios \enquote{sociables}
\\
\bigskip
\pause
Donde el término \textit{sociable} se usa para indicar que los polinomios que constituyen una función \textcolor{blue}{spline}, están estrechamente vinculados.
\end{frame}
\begin{frame}
\frametitle{El nomobre de spline}    
El nombre de \textcolor{blue}{spline}, viene del inglés ya que es un instrumento que utilizaban los ingenieros navales para dibujar curvas suaves, forzadas a pasar por un conjunto de puntos prefijados.
\end{frame}
\begin{frame}
\frametitle{Las tres B's de los splines}
El uso de las funciones splines tiene mucha aceptación y popularidad se deben a tres razones básicas:
\pause
\setbeamercolor{item projected}{bg=beige,fg=black}
\setbeamertemplate{enumerate items}{%
\usebeamercolor[bg]{item projected}%
\raisebox{1.5pt}{\colorbox{bg}{\color{fg}\footnotesize\insertenumlabel}}%
}
\begin{enumerate}[<+->]
\item \textbf{Buenos:} Se  pueden usar en la solución de una gran variedad de problemas.
\item \textbf{Bonitos:} La teoría matemática en que se basan es muy simple y a la vez elegante.
\item \textbf{Baratos:} Su cálculo es muy sencillo y económico.
\end{enumerate}
\end{frame}
\begin{frame}
\frametitle{Manejando splines con \python}
Usaremos la librería \funcionazul{scipy} que contiene varias funciones con las que ahorramos tiempo para manejar splines y ajustar funciones sociables a un conjunto de datos.
\\
\medskip
\pause
No está de más que revises la teoría al respecto, en la mayoría de los libros de análisis numérico, podrás encontrar la construcción matemática y formal de los splines.
\end{frame}
\begin{frame}
\frametitle{Funciones para los splines}
Necesitaremos de dos pasos para el uso de splines con \python, de la libería: \funcionazul{scipy.interpolate}, las cuales son:
\pause
\setbeamercolor{item projected}{bg=lilac,fg=white}
\setbeamertemplate{enumerate items}{%
\usebeamercolor[bg]{item projected}%
\raisebox{1.5pt}{\colorbox{bg}{\color{fg}\footnotesize\insertenumlabel}}%
}
\begin{enumerate}[<+->]
\item \funcionazul{splrep}: Calcula el spline básico (B-spline) para una curva 1-D.
\\
\medskip
Dados un conjunto de puntos $(x[i],y[i])$ determina una aproximación con un spline suave de grado $k$ en el intervalo $xb \leq x \leq xe$.
\seti
\end{enumerate}
\end{frame}
\begin{frame}
\frametitle{Funciones para los splines}
\setbeamercolor{item projected}{bg=lilac,fg=white}
\setbeamertemplate{enumerate items}{%
\usebeamercolor[bg]{item projected}%
\raisebox{1.5pt}{\colorbox{bg}{\color{fg}\footnotesize\insertenumlabel}}%
}
\begin{enumerate}[<+->]
\conti    
\item \textbf{splev}: Evalúa un B-spline o sus derivadas. Dados los nodos y coeficientes de un B-spline, calcula 
el valor del polinomio suave y sus derivadas.
\end{enumerate}
\end{frame}
\begin{frame}[allowframebreaks, fragile]
\frametitle{Código}
\begin{lstlisting}
import matplotlib.pyplot as plt
import scipy.interpolate as si
from numpy import linspace

def trazador_cub(n):
    xi = linspace(-1, 1, n)
    yi = 1./(1 + 25*xi**2)
    tck = si.splrep(xi, yi)
    return tck

x = linspace(-1, 1, 100)
y = 1./(1 + 25*x**2)

tck = trazador_cub(8)
ys8 = si.splev(x, tck)

tck = trazador_cub(12)
ys12 = si.splev(x, tck)

plt.plot(x, y)
plt.plot(x, ys8,'+g-', label='n=8')
plt.plot(x, ys12,'+r-',label ='n=12')
plt.legend(loc='best')
plt.title('Interpolacion con splines cubicos')
plt.ylim(-0.2, 1.2)
plt.show()
\end{lstlisting}
\end{frame}
\begin{frame}
\frametitle{Resultados gráficos}
\begin{figure}
	\centering
	\includegraphics[scale=0.47]{Imagenes/Funcion_Runge_04.eps} 
\end{figure}
\end{frame}
\end{document}
