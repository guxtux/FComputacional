\documentclass[12pt]{beamer}
\newenvironment{ConCodigo}[1]
  {\begin{frame}[fragile,environment=ConCodigo]{#1}}
  {\end{frame}}
\graphicspath{{Imagenes/}{../Imagenes/}}
\usepackage[utf8]{inputenc}
\usepackage[spanish]{babel}
\usepackage{hyperref}
\usepackage{etex}
\reserveinserts{28}
\usepackage{amsmath}
\usepackage{amsthm}
\usepackage{mathtools}
\usepackage{multicol}
\usepackage{multirow}
\usepackage{tabulary}
%\usepackage{tabularx}
\usepackage{booktabs}
\usepackage{nccmath}
\usepackage{biblatex}
\usepackage{epstopdf}
\usepackage{graphicx}
\usepackage{siunitx}
\sisetup{scientific-notation=true}
%\usepackage{fontspec}
\usepackage{lmodern}
\usepackage{float}
\usepackage[format=hang, font=footnotesize, labelformat=parens]{caption}
\usepackage[autostyle,spanish=mexican]{csquotes}
\usepackage{standalone}
\usepackage{tikz}
\usepackage[siunitx]{circuitikz}
\usetikzlibrary{arrows,patterns,shapes}
\usetikzlibrary{decorations.markings}
\usetikzlibrary{arrows}
\usepackage{color}
%\usepackage{beton}
%\usepackage{euler}
%\usepackage[T1]{fontenc}
\usepackage[sfdefault]{roboto}  %% Option 'sfdefault' only if the base font of the document is to be sans serif
\usepackage[T1]{fontenc}
\renewcommand*\familydefault{\sfdefault}
\DeclareGraphicsExtensions{.pdf,.png,.jpg}
\usepackage{hyperref}
\renewcommand {\arraystretch}{1.5}
\newcommand{\python}{\texttt{python}}
\usefonttheme[onlymath]{serif}
\setbeamertemplate{navigation symbols}{}
\usetikzlibrary{patterns}
\usetikzlibrary{decorations.markings}
\tikzstyle{every picture}+=[remember picture,baseline]
%\tikzstyle{every node}+=[inner sep=0pt,anchor=base,
%minimum width=2.2cm,align=center,text depth=.15ex,outer sep=1.5pt]
%\tikzstyle{every path}+=[thick, rounded corners]
\setbeamertemplate{caption}[numbered]
\newcommand{\ptm}{\fontfamily{ptm}\selectfont}
%Se usa la plantilla Warsaw modificada con spruce
\mode<presentation>
{
  \usetheme{Warsaw}
  \setbeamertemplate{headline}{}
  \useoutertheme{default}
  \usecolortheme{beaver}
  \setbeamercovered{invisible}
}
\AtBeginSection[]
{
\begin{frame}<beamer>{Contenido}
\normalfont\mdseries
\tableofcontents[currentsection]
\end{frame}
}

\usepackage{listings}
\lstset{ %
language=Python,                % choose the language of the code
basicstyle=\small,       % the size of the fonts that are used for the code
numbers=left,                   % where to put the line-numbers
numberstyle=\small,      % the size of the fonts that are used for the line-numbers
stepnumber=1,                   % the step between two line-numbers. If it is 1 each line will be numbered
numbersep=5pt,                  % how far the line-numbers are from the code
backgroundcolor=\color{white},  % choose the background color. You must add \usepackage{color}
showspaces=false,               % show spaces adding particular underscores
showstringspaces=false,         % underline spaces within strings
showtabs=false,                 % show tabs within strings adding particular underscores
frame=single,   		% adds a frame around the code
tabsize=2,  		% sets default tabsize to 2 spaces
captionpos=b,   		% sets the caption-position to bottom
breaklines=true,    	% sets automatic line breaking
breakatwhitespace=false,    % sets if automatic breaks should only happen at whitespace
escapeinside={\%},          % if you want to add a comment within your code
stringstyle =\color{magenta},
keywordstyle = \color{blue},
commentstyle = \color{green},
identifierstyle = \color{red}
}
\begin{document}
\title{Examen 2 - Operaciones matem\'{a}ticas b\'{a}sicas}
\subtitle{Soluci\'{o}n}
%\subsubtitle{Curso de F\'{i}sica Computacional}
\author{M. en C. Gustavo Contreras May\'{e}n}
%\email{curso.fisica.comp@gmail.com}
%\ptsize{10}
\maketitle
\fontsize{14}{14}\selectfont
\spanishdecimal{.}
\begin{frame}{Contenido}
\tableofcontents[pausesections]
\end{frame}
\section{Problema 1}
\begin{frame}
\frametitle{Problema 1}
La funci\'{o}n gamma $\Gamma (x)$, se define como la siguiente integral
	\[ \Gamma (x) = \int_{0}^{\infty} t^{x-1} e^{-t} dt\]
	que converge para todo $x$ positivo, pese a que para $0<x<1$ el integrando tiene una divergencia en $t=0$.
	\\
	Calcula num\'{e}ricamente a partir de la definci\'{o}n anterior, la $\Gamma$ para $x=10$ y $x=1/2$, valores para los cuales se conocen los resultados anal\'{i}ticos:
	\begin{eqnarray*}
		\Gamma(10) &=& 9! = 362880 \\
		\Gamma(1/2) &=& \sqrt{\pi}
	\end{eqnarray*}
\end{frame}
\begin{frame}
En cada caso debes:
	\begin{enumerate}
		\item indicar el cambio de variable utilizado.
		\item el n\'{u}mero de puntos utilizados en la discretizaci\'{o}n.
		\item el m\'{e}todo de integraci\'{o}n.
		\item el resultado obtenido.
		\item el error cometido respecto al valor anal\'{i}tico.
	\end{enumerate}
\end{frame}
\begin{frame}[fragile]
\frametitle{Problema 1, inciso a)}
	\begin{enumerate}
		\item No se requiere cambio de variable.
		\item Con $n=10$.
		\item Usando el m\'{e}todo del trapecio.
		\item El resultado es: $362877.41$
		\item El error relativo es: $7.1408 \times 10^{-6}$
	\end{enumerate}
La funci\'{o}n gamma al igual que otras funciones, vienen contenidas en el m\'{o}dulo \texttt{special} de \texttt{scipy}:
\medskip
\begin{verbatim}
from scipy import special
special.gamma(x)
\end{verbatim}
\end{frame}
\begin{frame}
\frametitle{Ejercicio 1, inciso b)}
De la definici\'{o}n de la funci\'{o}n $\Gamma (x)$, tenemos para $x=1/2$:
\[ \Gamma(1/2) = \int_{0}^{\infty} t^{1/2-1} e^{-t} dt\]
por tanto
\[ \Gamma(1/2) = \int_{0}^{\infty} t^{-1/2} e^{-t} dt\]
y aqu\'{i} es donde tendremos problemas, ya que aunque hagamos integraci\'{o}n por partes, llegaremos de nuevo a un producto con $t$ elevada a una fracci\'{o}n.
\end{frame}
\begin{frame}[fragile]
Hagamos el siguiente cambio de variable
\begin{eqnarray*}
	t &=& z^{2} \\
	dt &=& 2z dz
\end{eqnarray*}
Por lo que al sustituir en la integral inicial y revisando los nuevos l\'{i}mites de integraci\'{o}n, resulta que
\[  \Gamma(1/2) = \int_{0}^{\infty} t^{-1/2} e^{-t} dt = \int_{0}^{\infty} \left( z^{2} \right)^{-1/2} e^{-z^{2}} 2zdz \]
\end{frame}
\begin{frame}
Por tanto
\[ \Gamma(1/2) = \int_{0}^{\infty} z^{-1/2} e^{-z^{2}} 2z dz  \]
es decir
\[ \Gamma(1/2) = 2 \int_{0}^{\infty} e^{-z^{2}} dz  \]
Que es la funci\'{o}n de inter\'{e}s para integrar.
\\
\medskip
Con $n=20$ y usando el m\'{e}todo del trapecio, tenemos que el valor de la integral es: $1.752927$ con un error relativo de $1.952674 \times 10^{-2}$
\end{frame}
\section{Problema 2}
\begin{frame}
\frametitle{Problema 2}
Eval\'{u}a num\'{e}ricamente las siguientes integrales:
	\begin{eqnarray*}
		I_{1} &=& \int_{0}^{\infty} e^{-x} ln x dx \\
		I_{2} &=& \int_{0}^{1} \dfrac{1+x}{1-x^{3}} ln\dfrac{1}{x} dx
	\end{eqnarray*}
	El problema consiste en resolver las integrales con alg\'{u}n cambio de variable para tener un integrando suave en un intervalo finito.
\end{frame}
\begin{frame}
	Se debe de obtener un resultado razonablemente bueno, teniendo que evaluar el integrando final con el menor n\'{u}mero ($N$) de veces que sea posible. Como criterio de convergencia debes de usar alguna cantidad como
	\[ \epsilon = \dfrac{I - I_{N}}{I} < 10^{-n}\]
	con n = $2,3,4,5,6$
\end{frame}
\begin{frame}
En cada caso debes:
\begin{enumerate}
	\item indicar el(los) cambio(s) de variable utilizado(s).
	\item el n\'{u}mero de puntos utilizados en la discretizaci\'{o}n.
	\item el m\'{e}todo de integraci\'{o}n.
	\item el resultado obtenido.
	\item el error cometido respecto al valor de I.
\end{enumerate}
Nota: no se vale usar integraci\'{o}n por partes.
\end{frame}
\begin{frame}
\frametitle{Para la integral I$_{1}$}
El cambio de variable propuesto es:
\begin{eqnarray*}
x &=& e^{u} \\
dx &=& u e^{u}
\end{eqnarray*}
\end{frame}
\end{document}