\documentclass[letterpaper]{article}
\usepackage[utf8]{inputenc}
%\usepackage[latin1]{inputenc}
\usepackage[spanish]{babel}
\usepackage{geometry}
\usepackage{anysize}
\usepackage{graphicx} 
\usepackage{amsmath}
\usepackage{tikz}
\usetikzlibrary{patterns}
\usetikzlibrary{decorations.markings}
\usetikzlibrary{matrix}
\usepackage{xy}
\usepackage{siunitx}
\usepackage[american,cuteinductors,smartlabels]{circuitikz}
\usetikzlibrary{calc}
\usepackage{color}
%\numberwithin{equation}{list}
\marginsize{1cm}{2cm}{0cm}{2cm}  
\title{Examen Reposición Tema 5 \\ \begin{large}Curso de Física Computacional\end{large}}
\author{M. en C. Gustavo Contreras Mayén}
\date{ }
\begin{document}
\maketitle
\fontsize{14}{14}\selectfont
\spanishdecimal{.}
\begin{enumerate}
\item Resuelve la ecuación de Laplace con el siguiente arreglo donde la placa cuadrada conductora se encuentra a 1 volt, mientras que la placa cuadrada exterior, se encuentra a 0 volts. Grafica e interpreta tus resultados.
\begin{center}
\begin{tikzpicture}
	\draw [dashed](0,0) -- (6,0);
	\draw [dashed](3,-3) -- (3,3);
	\draw [draw=black, fill=black!25](2.5,-0.5) rectangle (3.5,0.5);
	\draw (3.7,0.7) node  {$1 V$};
	\draw (0.5,-2.5) rectangle (5.5,2.5);
	\draw (5.7,2.7) node  {$0 V$};
\end{tikzpicture}
\end{center}
\item Presenta una discusión sobre el método de solución llamado de Crank-Nicholson, indicando en qué tipo de EDP se puede aplicar y desarrolla un pseudo-código en Python para implementar esa solución. Debes de indicar cuáles referencias consultaste.
\end{enumerate}
\end{document}