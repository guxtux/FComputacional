\documentclass[12pt]{article}
%\usepackage[latin1]{inputenc}
\usepackage[utf8x]{inputenc}
\usepackage[spanish]{babel}
\usepackage{amsmath}
\usepackage{amsthm}
\usepackage{graphicx}
\usepackage{tabularx}
\usepackage{color}
\usepackage{float}
\usepackage{anysize}
\usepackage{listings}
\lstset{ %
language=Python,                % choose the language of the code
basicstyle=\small,       % the size of the fonts that are used for the code
numbers=left,                   % where to put the line-numbers
numberstyle=\small,      % the size of the fonts that are used for the line-numbers
stepnumber=1,                   % the step between two line-numbers. If it is 1 each line will be numbered
numbersep=5pt,                  % how far the line-numbers are from the code
backgroundcolor=\color{white},  % choose the background color. You must add \usepackage{color}
showspaces=false,               % show spaces adding particular underscores
showstringspaces=false,         % underline spaces within strings
showtabs=false,                 % show tabs within strings adding particular underscores
frame=single,   		% adds a frame around the code
tabsize=2,  		% sets default tabsize to 2 spaces
captionpos=b,   		% sets the caption-position to bottom
breaklines=true,    	% sets automatic line breaking
breakatwhitespace=false,    % sets if automatic breaks should only happen at whitespace
escapeinside={\%},          % if you want to add a comment within your code
stringstyle =\color{magenta},
keywordstyle = \color{blue},
commentstyle = \color{green},
identifierstyle = \color{red}
}
\author{M. en C. Gustavo Contreras Mayén.}
\title{EDP Parab\'{o}licas \\ \begin{Large}Curso de Fí­sica Computacional\end{Large}}
\begin{document}
\maketitle
Se inicia con el llamado a las librer\'{i}as \texttt{Numpy} y las de gr\'{a}ficos, para luego declarar constantes del material y del arreglo con el que trabajaremos.
\\
En la l\'{i}nea 23 se inicializan los puntos de la barra a $100^{\circ}C$. Entre las l\'{i}neas 26 y 30, se fija la temperatura de los extremos a $0^{\circ}C$, valor que no cambia durante la ejecuci\'{o}n del programa.
\\
En la l\'{i}nea 35 se realiza la iteraci\'{o}n con el tiempo, la siguiente l\'{i}nea itera con el esquema de diferencias finitas. En la l\'{i}nea 38, en los extremos y por cada 100 pasos en el tiempo, luego se incrementa $m$ cada 100 pasos en $t=m$. En la l\'{i}nea 42, los valores "viejos" se dejan como "nuevos".
\\
En las l\'{i}neas 45 y 46, se crean dos arreglos para obtener la rejilla que nos servir\'{a} para graficar la superficie. La funci\'{o}n \texttt{functz}, devuelve la temperatura. Posteriormente se usan las instrucciones para darle una apariencia a la gr\'{a}fica.
\begin{lstlisting}
from numpy import *
import matplotlib.pyplot as plt
from mpl_toolkits.mplot3d import Axes3D
from matplotlib import cm

fig = plt.figure()
ax = fig.add_subplot(111, projection='3d')

Nx = 101
Nt = 3000
Dx = 0.01414
Dt = 1.

KAPPA = 210.
SPH = 900.
RHO = 2700.


T =zeros((Nx,2), dtype=float)
Tpl = zeros ((Nx,31), dtype=float)


for ix in range(1,Nx-1):
    T[ix,0] = 100.0

T[0,0] = 0.0
T[0,1] = 0.0

T[Nx-1,0] = 0.0
T[Nx-1,1] = 0.0

cons = KAPPA/(SPH*RHO)*Dt/(Dx*Dx)

m = 1
for t in range(1,Nt):
    for ix in range(1,Nx-1):
        T[ix,1] = T[ix,0] + cons*(T[ix+1,0] + T[ix-1,0]-2.*T[ix,0])
    if t%100 == 0 or t == 1:
        for ix in range(1,Nx-1,2): Tpl[ix,m] = T[ix,1]
        print (m)
        m = m + 1
    for ix in range(1,Nx-1): T[ix,0] = T[ix,1]


x = list(range(1,Nx-1,2))
y = list(range(1,30))

X, Y = meshgrid(x,y)

def functz(Tpl):
    z = Tpl[X,Y]
    return z
    
Z = functz(Tpl)

surf = ax.plot_surface(X,Y,Z,rstride=1, cstride=1,linewidth=0.5,cmap=cm.coolwarm)

fig.colorbar(surf, shrink=0.5, aspect=5)
ax.set_zlim(-10, 100)
cset = ax.contourf(X,Y,Z, zdir='z',offset=-5, cmap=cm.coolwarm)
ax.set_xlabel('Posicion')
ax.set_ylabel('tiempo')
ax.set_zlabel('Temperatura')
plt.show()
\end{lstlisting}

\end{document}