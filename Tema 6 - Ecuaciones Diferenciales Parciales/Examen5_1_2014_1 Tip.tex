\documentclass[letterpaper]{article}
\usepackage[utf8]{inputenc}
%\usepackage[latin1]{inputenc}
\usepackage[spanish]{babel}
\usepackage{geometry}
\usepackage{anysize}
\usepackage{graphicx} 
\usepackage{amsmath}
\usepackage{tikz}
\usepackage{float}
%\usepackage{tkz-euclide}
%\usetkzobj{all}
\usetikzlibrary{patterns}
\usetikzlibrary{calc}
\usetikzlibrary{decorations.markings}
\usetikzlibrary{matrix}
\usepackage{xy}
\usepackage{siunitx}
\usepackage[american,cuteinductors,smartlabels]{circuitikz}
\usetikzlibrary{calc}
\usepackage{color}
%\numberwithin{equation}{list}
\marginsize{2cm}{2cm}{0cm}{2cm}  
\title{Tarea Examen 1/3 - Ecuaciones diferenciales parciales \\ \begin{large}Curso de Física Computacional\end{large}}
\author{M. en C. Gustavo Contreras Mayén}
\date{ }
\begin{document}
\maketitle
\fontsize{14}{14}\selectfont
\spanishdecimal{.}
La ecuación de Poisson en coordenadas polares es
\[ \nabla^{2} u = \dfrac{\partial^{2}u}{\partial r^{2}} + \dfrac{1}{r} \dfrac{\partial u}{\partial r} + \dfrac{1}{r^{2}} \dfrac{\partial^{2}u}{\partial \theta^{2}} = - \dfrac{\rho (r,\theta)}{\epsilon_{0}} \]
Sean $i$ y $j$ los índices para las direcciones $r$ y $\theta$, respectivamente, la expresión en diferencias finitas para la ecuación de Poisson es
\[ \begin{split}
 \nabla^{2} u =& \dfrac{u(i-1,j)-2u(i,j)+ u(i+1,j)}{\Delta r^{2}} + \\
 +& \dfrac{1}{r_{i}} \dfrac{u(i+1,j)-u(i-1,j)}{2\Delta r} \\
 +& \dfrac{1}{r_{i}^{2}} \dfrac{u(i,j-1)- 2u(i,j)+u(i,j+1)}{\Delta \theta^{2}} = - \dfrac{\rho (r,\theta)}{\epsilon_{0}}
\end{split} \]
\begin{figure}[H]
\begin{center}
\begin{tikzpicture}[font=\small]
	%\tkzDefPoint(0,0){O}
	%\tkzDefPoint(0,5){A}
	\draw (0,0) -- (10,0);
	\draw (5,0) -- (5,5);
	\draw (5,0) -- (2.5,5);
	\draw (5,0) -- (7.5,5);
	\draw [red](10,0) arc[radius=5, start angle=0, end angle=180];
	\draw [red](9,0) arc[radius=4, start angle=0, end angle=180];
	\draw [red](8,0) arc[radius=3, start angle=0, end angle=180];
	\draw [red](7,0) arc[radius=2, start angle=0, end angle=180];
	\draw [fill=black](5,2) circle (0.05) node[above=0.3]{$i-1,j$};
	\draw [fill=black](5,3) circle (0.05) node[above=0.3]{$i,j$};;
	\draw [fill=black](5,4) circle (0.05) node[above=0.3]{$i+1,j$};
	\draw [fill=black](3.65,2.7) circle (0.05) node[left=0.3]{$i,j-1$};
	\draw [fill=black](6.35,2.7) circle (0.05) node[right=0.3]{$i,j+1$};
	%\tkzDrawArc (O,A)(180)
\end{tikzpicture}
\end{center}
\caption{Notación de los índices en coordenadas polares.}
\label{fig:mediocirculo}
\end{figure}
El valor para el punto $(i,j)$ se puede calcular de
\[ \begin{split}
 2 \left[ \dfrac{1}{\Delta r^{2}} + \dfrac{1}{r^{2}_{i}\Delta \theta} \right] u(i,j) =& \dfrac{u(i-1,j)+u(i+1,j)}{\Delta r^{2}} + \\ 
+& \dfrac{1}{r_{i}} \dfrac{u(i+1,j)-u(i-1,j)}{2 \Delta r} + \\
+& \dfrac{1}{r_{i}^{2}} \dfrac{u(i,j-1)+u(i,j+1)}{\Delta \theta^{2}}
\end{split} \]
Lo que hay que aplicar es el cambio en $r$ y en $\theta$, revisar en dónde se mantiene la simetría del problema. En la figura (\ref{fig:mediocirculo}), se detalla media circunferencia; en el problema, tendrían que consierar que $0 \leq \theta \leq 2 \pi$, y que tenemos un anillo con radios $a$ y $b$.
\begin{figure}[H]
\begin{center}
\begin{tikzpicture}[font=\small]
	%\tkzDefPoint(0,0){O}
	%\tkzDefPoint(0,5){A}
	\draw (0,0) -- (10,0);
	\draw (5,0) -- (5,5);
	\draw (5,0) -- (2.5,5);
	\draw (5,0) -- (7.5,5);
	\draw [blue](5,0) circle (4.5);	
	\draw [blue](5,0) circle (2.5);
	%\draw [blue](10,0) arc[radius=5, start angle=0, end angle=180];
	%\draw [red](9,0) arc[radius=4, start angle=0, end angle=180];
	%\draw [red](8,0) arc[radius=3, start angle=0, end angle=180];
	%\draw [red](7,0) arc[radius=2, start angle=0, end angle=180];
	%\draw [fill=black](5,2) circle (0.05) node[above=0.3]{$i-1,j$};
	%\draw [fill=black](5,3) circle (0.05) node[above=0.3]{$i,j$};;
	%\draw [fill=black](5,4) circle (0.05) node[above=0.3]{$i+1,j$};
	%\draw [fill=black](3.65,2.7) circle (0.05) node[left=0.3]{$i,j-1$};
	\draw [->, thick] (5,0) -- (6.13,2.22);
	\draw [fill=black](6.13,2.22) circle (0.05) node[right=0.3]{$a$};
	\draw [->, thick] (5,0) -- (3,4.05);
	\draw [fill=black](3,4.05) circle (0.05) node[right=0.3]{$b$};
	%\tkzDrawArc (O,A)(180)
\end{tikzpicture}
\end{center}
\caption{Anillo circular en donde se debe de resolver la ecuación de Poisson.}
\label{fig:anillocircular}
\end{figure}
Como se muestra en la figura (\ref{fig:anillocircular}), pueden ir variando el valor del radio $a$, el caso inicial sería cuando $a=0$ y queda en el origen como punto de inicio, e implementar su código, eso les ayudaría para dejar un valor diferente de cero.
\end{document}