\documentclass[letterpaper]{article}
\usepackage[utf8]{inputenc}
%\usepackage[latin1]{inputenc}
\usepackage[spanish]{babel}
\usepackage{geometry}
\usepackage{anysize}
\usepackage{graphicx} 
\usepackage{amsmath}
\usepackage{lscape}
%\numberwithin{equation}{list}
\marginsize{1cm}{2cm}{0cm}{2cm}  
\title{Evaluación Tema 5 \\ \begin{large}Curso de Física Computacional\end{large}}
\author{M. en C. Gustavo Contreras Mayén}
\date{ }
\begin{document}
%\begin{landscape}
\maketitle
\fontsize{10}{10}\selectfont
\spanishdecimal{.}
\begin{flushright}
Fecha: \line(1,0){20} / \line(1,0){20} /\line(1,0){20}
\end{flushright}
Nombre: \underline{Ricardo Saúl García Jiménez}.
\\
\\
Para cada uno de los problemas de la tarea-examen, se indica la puntuación que lograste.
\begin{center}
\renewcommand{\arraystretch}{2.5}
\begin{tabular}{| l | c | c | c | c | c | c | c | c |}
\hline
 & Documentación & Modularidad & Eficiencia & Robustez & Graficación & Ejecución & Interpretación & Suma \\ \hline
Problema 1 & 0 & 1 & 1 & 1 & 1 & 1 & 0 & 5 \\ \hline
Problema 2 & 0 & 1 & 1 & 1 & 1 & 1 & 0 & 5 \\ \hline
Problema 3 & 0 & 1 & 0 & 0 & 0 & 0 & 0 & 1 \\ \hline
Problema 4 & 0 & 1 & 1 & 0 & 0 & 1 & 0 & 3 \\ \hline
Problema 5 & 0 & 1 & 0 & 0 & 1 & 0 & 0 & 2 \\ \hline
Problema 6 & 0 & 0 & 0 & 0 & 0 & 0 & 0 & 0 \\ \hline
Problema 7 & 0 & 1 & 1 & 1 & 1 & 1 & 0 & 5 \\ \hline
\end{tabular}
\end{center}
\fontsize{14}{14}\selectfont
Para todos los problemas se pidió que considerasen los atributos que se evalúan en la rúbrica.
\\
\\
Problema 1: Carece de documentación, aunque la tarea más complicada era establecer los equipotenciales, se resuelve el problema, pero faltan elementos que debieron de haberse incluido.
\\
\\
Problema 2: Bien resuelto, pero  nuevamente carece de los elementos de documentación.
\\
\\
Problema 3: Dado que se carece de documentación, no se conoce cómo es que llegaste a la expresión de diferencias finitas para la solución del problema en coordenadas polares, luego, la gráfica que resulta, no se muestra algún resultado como tal. Aquí más que irse directamente a proponer un código, debe de haber la discusión sobre cómo argumentas tu propuesta. Aunque la gráfica que se incluye no devuelve lo que sería la ''solución'' como tal, habría elementos para revisar qué anda bien o qué anda mal en el código que propones.
\\
\\
Problema 4: Carece de documentación sobre cómo asignaste la función senoidal en la distribución inicial de temperatura, en la presentación que se les envió, hay una gráfica resultado que podrían haber utilizado como referencia. No hay una interpretación de tus resultados, que de haberla hecho, podrías haber identificado que algo no estaba bien en tu planteamiento.
\\
\\
Problema 5: La termodinámica no se adapta a tu solución. Carece de documentación sobre cómo asignas las temperaturas en las dos barras, y sobre todo, la manera en que conforme evoluciona en el tiempo el cambio de temperatura, debería de haber una correspondencia con lo que sabemos de la física. La gráfica en la presentación, debió de haber servido como referencia para que lograras el resultado.
\\
\\
Problema 6: Al momento de ejecutar tu código, hay un par de errores en la rutina de graficación, y lo que muestra la ventana, no corresponde con la solución esperada.
\\
\\
Problema 7: El resultado de cómo vibra la cuerda, tiene detalles, ya que parecería en la animación que conforme se desplaza la onda, hay superpuestas otras ondas.
\\
\\
Considera que para cada ejercicio entregado, debiste de haber revisado y ejecutado los problemas, tuvieron el tiempo necesario para cotejar que sus resultados fueran los esperados, situación que no se refleja en tu examen, con excepción del último ejercicio. No consideraste la solución de cada uno de los problemas con la estructura de la rúbrica. Por ello tu calificación de este examen es 8.0 (ocho).

%\end{landscape}
\end{document}