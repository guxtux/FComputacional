\input{../Preambulos/preambulo_materiales}
\usepackage{minted}

\title{\vspace{-2cm}Ejercicio a cuenta del Tema 1 \\ {\large Curso Física Computacional} \vspace{-3ex}}
\author{M. en C. Gustavo Contreras Mayén. \texttt{gux7avo@ciencias.unam.mx}}
\date{}

\begin{document}

\fontsize{14}{14}\selectfont
\vspace{-4cm}
\maketitle

Aprendimos en la secundaria a resolver la ecuación homogénea de segundo grado:
\begin{align}
a \, x^{2} + b \, x + c = 0
\label{eq:ecuacion_01}
\end{align}
que tiene una solución analítica que se puede escribir como:
\begin{align}
x_{1,2} &= \dfrac{-b \pm \sqrt{b^{2} - 4 \, a \, c}}{2 \, a} \label{eq:ecuacion_02} \\[0.5em]
\pderivada{x}_{1,2} &= \dfrac{-2 \, c}{b \pm \sqrt{b^{2} - 4 \, a \, c}} \label{eq:ecuacion_03}
\end{align}
Revisando la expresión anterior vemos que la diferencia en el denominador nos puede incrementar el error, ya que éste aumenta cuando $b^{2} >> 4 \, a \, c$ debido a que la raíz cuadrada y el siguiente término están muy próximas a cancelarse.
\begin{enumerate}
\item Escribe un primer programa en \python{} que calcule las cuatro soluciones: \newline $(x_{1}, x_{2}, \pderivada{x}_{1}, \pderivada{x}_{2})$ para valores arbitrarios de $a$, $b$ y $c$.
\item Con un segundo programa, revisa cómo los errores obtenidos en los cálculos, aumentan conforme hay una diferencia de términos y su relación con el épsilon de la computadora. Prueba con los siguientes valores $a = 1$, $b = 1$, $c = 10^{-n}, n = 1, 2, 3, \ldots$ Considera que las soluciones de la ec. (\ref{eq:ecuacion_02}): $x_{1}, x_{2}$ son los valores exactos.
\item ¿Cómo mejorarías el programa para obtener la mayor precisión en tu respuesta?
\end{enumerate}

\textbf{Entrega del ejercicio:} La solución deberás de enviarla vía Moodle con fecha límite el día martes 30 de agosto a las 6pm.
\end{document}