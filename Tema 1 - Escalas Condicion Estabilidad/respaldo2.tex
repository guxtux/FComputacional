\frametitle{Identificadores}
Son nombres que hacen referencia a los objetos que componen un programa: constantes, variables, funciones, etc.
\\
\bigskip
Reglas para construir identificadores:
\begin{enumerate}
\item El primer car\'{a}cter debe ser una letra o el car\'{a}cter de subrayado (gui\'{o}n bajo)
\item El primer car\'{a}cter puede ir seguido de un n\'{u}mero variable de d\'{i}gitos num\'{e}ricos, letras o car\'{a}cteres de subrayado.
\item No pueden utilizarse espacios en blanco, ni s\'{i}mbolos de puntuaci\'{o}n.
\item Python distingue may\'{u}sculas y min\'{u}sculas.
\item No pueden utilizarse palabras reservadas del lenguaje.
\end{enumerate}
\end{frame}

\section{Variables}
\begin{frame}[fragile]
\frametitle{Variables}
\fontsize{12}{12}\selectfont
\begin{minipage}{5.5cm}
\begin{exampleblock}{}<1->
 \scalebox{0.8}
 \verb|>>> base = 2| \\
\end{exampleblock}
\end{minipage}

\end{frame}

\begin{exampleblock}{}<2->
 \verb|>>> print base| \\
 \pause
 \textcolor{blue}{2}
\end{exampleblock}
\begin{exampleblock}{}<3->
 \verb|>>> print "base"| \\
 \pause 
 \textcolor{blue}{\texttt{base}}
\end{exampleblock}
\begin{exampleblock}{}<4->
 \verb|>>> base = base + 1|
\end{exampleblock}
\end{minipage}
\hspace{0.5cm}
\begin{minipage}{5.5cm}
\begin{exampleblock}{}<5->
 \verb|>>> base| \\
 \pause
 \textcolor{blue}{3}
\end{exampleblock}
\begin{exampleblock}{}<6->
 \verb|>>> alt = 4| \\
\end{exampleblock}
\begin{exampleblock}{}<7->
 \verb|>>> area = base+alt; a= 3|
\end{exampleblock}
\begin{exampleblock}{}<8->
 \verb|>>> a= 2*a| \\
\end{exampleblock}
\end{minipage}
\end{frame}
\begin{frame}[fragile]
\begin{minipage}{5.5cm}
\begin{exampleblock}{}<1->
 \verb|>>> area== 2*a| \\
 \pause
 \textcolor{blue}{\texttt{False}}
\end{exampleblock}
\begin{exampleblock}{}<2->
 \verb|>>> x= "uno"; y= "dos"| \\
\end{exampleblock}
\begin{exampleblock}{}<3->
 \verb|>>> x| \\
 \pause
 \textcolor{blue}{\texttt{uno}}
\end{exampleblock}
\begin{exampleblock}{}<4->
 \verb|>>> print x| \\
 \pause
 \textcolor{blue}{\texttt{uno}}
\end{exampleblock}
\end{minipage}
\hspace{0.5cm}
\begin{minipage}{5.5cm}
\begin{exampleblock}{}<5->
 \verb|>>> x+y| \\
 \pause
 \textcolor{blue}{\texttt{unodos}}
\end{exampleblock}
\begin{exampleblock}{}<6->
 \verb|>>> print x+y| \\
 \pause
 \textcolor{blue}{\texttt{unodos}}
\end{exampleblock}
\end{minipage}
\end{frame}
\section{Listas y Tuplas}
\begin{frame}[fragile]
\frametitle{Listas y Tuplas}
\fontsize{12}{12}\selectfont
\begin{minipage}{5.5cm}
\begin{exampleblock}{}<1->
	\verb|milista=[a,"hola",3.0,True]|
\end{exampleblock}
\begin{exampleblock}{}<2->
	\verb|milista| \\
	\pause
	\textcolor{blue}{\texttt{[8,"hola",3.0,True]}}
\end{exampleblock}
\begin{exampleblock}{}<3->
	\verb|milista[0]| \\
	\pause
	\textcolor{blue}{\texttt{8}}
\end{exampleblock}
\begin{exampleblock}{}<4->
	\verb|milista[1]| \\
	\pause
	\textcolor{blue}{\texttt{'hola'}}
\end{exampleblock}
\begin{exampleblock}{}<5->
	\verb|milista[2]| \\
	\pause
	\textcolor{blue}{\texttt{3.0}}
\end{exampleblock}
\begin{exampleblock}{}<6->
	\verb|milista[1:3]| \\
	\pause
	\textcolor{blue}{\texttt{'hola',3.0}}
\end{exampleblock}
\begin{exampleblock}{}<7->
	\verb|milista[0] = 2.0|
\end{exampleblock}
\end{minipage}
\hspace{0.5cm}
\begin{minipage}{5.5cm}
\begin{exampleblock}{}<8->
	\verb|milista| \\
	\pause
	\textcolor{blue}{\texttt{[2.0,"hola",3.0,True]}}
\end{exampleblock}
\begin{exampleblock}{}<9->
	\verb|milista[-1]| \\
	\pause
	\textcolor{blue}{\texttt{True}}
\end{exampleblock}
\begin{exampleblock}{}<10->
	\verb|milista.append("otro")| \\
\end{exampleblock}
\begin{exampleblock}{}<11->
	\verb|milista| \\
	\pause
	\textcolor{blue}{\texttt{[2.0,"hola",3.0,True,'otro']}}
\end{exampleblock}
\begin{exampleblock}{}<12->
	\verb|milista[:2]| \\
	\pause
	\textcolor{blue}{\texttt{[2.0,"hola"]}}
\end{exampleblock}
\begin{exampleblock}{}<13->
	\verb|milista[1:]| \\
	\pause
	\textcolor{blue}{\texttt{[2.0,"hola",3.0,True]}}
\end{exampleblock}
\begin{exampleblock}{}<14->
	\verb|lista2=[]|
\end{exampleblock}
\end{minipage}
\end{frame}
\begin{frame}[fragile]
\fontsize{12}{12}\selectfont
\begin{minipage}{5.5cm}
\begin{exampleblock}{}<1->
	\verb|lista2| \\
	\pause
	\verb|[]|
\end{exampleblock}
\begin{exampleblock}{}<2->
	\verb|lista2.insert(1,"a")|
\end{exampleblock}
\begin{exampleblock}{}<3->
	\verb|lista2| \\
	\pause
	\textcolor{blue}{\texttt{['a']}}
\end{exampleblock}
\begin{exampleblock}{}<4->
	\verb|lista2.insert(2,"b")|
\end{exampleblock}
\begin{exampleblock}{}<5->
	\verb|lista2| \\
	\pause
	\textcolor{blue}{\texttt{['a','b']}}
\end{exampleblock}
\begin{exampleblock}{}<6->
	\verb|lt = (1,2,True, "python")|
\end{exampleblock}
\end{minipage}
\hspace{0.5cm}
\begin{minipage}{5.5cm}
\begin{exampleblock}{}<7->
	\verb|lt| \\
	\pause
	\textcolor{blue}{\texttt{(1,2, True, 'python')}}
\end{exampleblock}
\begin{exampleblock}{}<8->
	\verb|lt[0]=3| \\
	\pause
	\textcolor{blue}{\texttt{Ups, hay un error!}}
\end{exampleblock}
\begin{exampleblock}{}<9->
	\verb|3 in lt| \\
	\pause
	\textcolor{blue}{\texttt{False}}
\end{exampleblock}
\end{minipage}
\end{frame}
\begin{frame}[fragile]
\frametitle{Funci\'{o}n \texttt{range}}
La funci\'{o}n \texttt{range()} crea una lista de n\'{u}meros enteros en sucesi\'{o}n aritm\'{e}tica. La funci\'{o}n \texttt{range()} puede tener uno, dos o tres argumentos num\'{e}ricos.
\\
\bigskip
La funci\'{o}n con un \'{u}nico argumento se escribe \texttt{range(n)} y crea una lista creciente de $n$ t\'{e}rminos enteros que empieza en $0$ y acaba en $n-1$ (el incremento es unitario)
\fontsize{12}{12}\selectfont
\begin{center}
\begin{minipage}{6cm}
\begin{exampleblock}{}<1->
	\verb|range(8)| \\
	\pause
	\textcolor{blue}{[0,1,2,3,4,5,6,7]}
\end{exampleblock}
\begin{exampleblock}{}<2->
	\verb|range(3,7)| \\
	\pause
	\textcolor{blue}{[3,4,5,6]}
\end{exampleblock}
\begin{exampleblock}{}<3->
	\verb|range(4,10,2)| \\
	\pause
	\textcolor{blue}{[4,6,8]}
\end{exampleblock}
\end{minipage}
\end{center}
\end{frame}
\begin{frame}[fragile]
\frametitle{Funciones intr\'{i}nsecas}
\fontsize{12}{12}\selectfont
\begin{minipage}{5.5cm}
\begin{exampleblock}{}<1->
	\verb|x =  -5|
\end{exampleblock}
\begin{exampleblock}{}<2->
	\verb|y =  4|
\end{exampleblock}
\begin{exampleblock}{}<3->
	\verb|p =  3.1416|
\end{exampleblock}
\begin{exampleblock}{}<4->
	\verb|z =  '6.3'|
\end{exampleblock}
\begin{exampleblock}{}<5->
	\verb|print int(p)| \\
	\pause
	\textcolor{blue}{3}
\end{exampleblock}
\begin{exampleblock}{}<6->
	\verb|abs(x)| \\
	\pause
	\textcolor{blue}{5}
\end{exampleblock}
\end{minipage}
\hspace{0.5cm}
\begin{minipage}{5.5cm}
\begin{exampleblock}{}<7->
	\verb|print float(z)| \\
	\pause
	\textcolor{blue}{6.0}
\end{exampleblock}
\begin{exampleblock}{}<8->
	\verb|complex(x)| \\
	\pause
	\textcolor{blue}{\texttt{(-5+0j)}}
\end{exampleblock}
\begin{exampleblock}{}<9->
	\verb|complex(x,y)| \\
	\pause
	\textcolor{blue}{\texttt{(-5+4j)}}
\end{exampleblock}
\begin{exampleblock}{}<10->
	\verb|print round(p,2)| \\
	\pause
	\textcolor{blue}{3.14}
\end{exampleblock}
\begin{exampleblock}{}<11->
	\verb|cmp(x,y)| \\
	\pause
	\textcolor{blue}{-1}
\end{exampleblock}
\end{minipage}
\end{frame}
\begin{frame}
\fontsize{12}{12}\selectfont
\begin{center}
\begin{tabular}{| c | c |}
\hline
Operaci\'{o}n & Descripci\'{o}n \\
\hline \texttt{int(x)} & Convierte \texttt{x} a entero \\
\hline \texttt{long(x)} & Convierte \texttt{x} a entero largo \\
\hline \texttt{float(x)} & Convierte \texttt{x} a punto flotante \\
\hline \texttt{complex(x)} & Convierte \texttt{x} al complejo \texttt{x+0j} \\
\hline \texttt{complex(x,y)} & Convierte al complejo \texttt{x+yj} \\
\hline
\end{tabular}
\end{center}
\end{frame}
\begin{frame}
\fontsize{12}{12}\selectfont
\begin{center}
\begin{tabular}{| c | c |}
\hline
Funci\'{o}n & Descripci\'{o}n \\
\hline \texttt{abs(x)} & Valor absoluto de \texttt{x} \\
\hline \texttt{max(sucesion)} & Mayor elemento de la sucesi\'{o}n \\
\hline \texttt{min(sucesion)} & Menor elemento de la sucesi\'{o}n \\
\hline \texttt{round(x,n)} & Redondea $x$ al decimal $n$ \\
\hline \texttt{cmp(x,y)} & Devuelve $-1$, $0$, $1$ si $x<y$, $x==y$, $x>y$ \\
\hline
\end{tabular}
\end{center}
\end{frame}
