\documentclass[12pt]{beamer}
\usepackage[utf8]{inputenc}
%\usepackage[latin1]{inputenc}
\usepackage[spanish]{babel}
%\usetheme{Warsaw}
%\usepackage{euler}
\usepackage{amsmath}
\usepackage{amsthm}
\usepackage{multicol}
\usepackage{graphicx}
\usepackage{tikz}
\usepackage{color}
\usepackage{listings}
\lstset{ %
language=C++,                % choose the language of the code
basicstyle=\footnotesize,       % the size of the fonts that are used for the code
numbers=left,                   % where to put the line-numbers
numberstyle=\footnotesize,      % the size of the fonts that are used for the line-numbers
stepnumber=1,                   % the step between two line-numbers. If it is 1 each line will be numbered
numbersep=5pt,                  % how far the line-numbers are from the code
backgroundcolor=\color{white},  % choose the background color. You must add \usepackage{color}
showspaces=false,               % show spaces adding particular underscores
showstringspaces=false,         % underline spaces within strings
showtabs=false,                 % show tabs within strings adding particular underscores
frame=single,   		% adds a frame around the code
tabsize=2,  		% sets default tabsize to 2 spaces
captionpos=b,   		% sets the caption-position to bottom
breaklines=true,    	% sets automatic line breaking
breakatwhitespace=false,    % sets if automatic breaks should only happen at whitespace
escapeinside={\%}{)}          % if you want to add a comment within your code
}
%\usepackage{epstopdf}
\DeclareGraphicsExtensions{.pdf,.png,.jpg}
\renewcommand {\arraystretch}{1.5}
\mode<presentation>
{
  \usetheme{CambridgeUS}
  \setbeamercovered{transparent}
  % or whatever (possibly just delete it)
}
\title{Tema 0 - Escalas, condici\'{o}n y estabilidad}
\subtitle{Curso de F\'{i}sica Computacional}
\author{M. en C. Gustavo Contreras May\'{e}n}
%\email{curso.fisica.comp@gmail.com}
%\ptsize{10}
\begin{document}
\maketitle
\fontsize{16}{16}\selectfont
\spanishdecimal{.}
\begin{frame}{Contenido}
\tableofcontents[pausesections]
\end{frame}
\section{Introducci\'{o}n}
\begin{frame}
Para el curso de F\'{i}sica Computacional ser\'{a} necesario que usemos un lenguaje de programaci\'{o}n para apoyarnos en la soluci\'{o}n de los problemas y algoritmos.
\\
\bigskip
El lenguaje de nuestra elecci\'{o}n es un medio para alcanzar nuestro objetivo del curso, m\'{a}s no el fin, por lo que revisaremos lo m\'{a}s b\'{a}asico de Python, dando la oportunidad de que por tu cuenta, logres un mayor conocimiento y pr\'{a}ctica con Python.
\end{frame}
\begin{frame}
\begin{figure}
	\centering
	\includegraphics[scale=0.5]{../Tema 0 - Programacion basica Python/python-logo-master-v3-TM2.eps} 
\end{figure}
\begin{itemize}
\item Lenguaje de programaci\'{o}n de alto nivel, interpretado.
\item Desarrollado por Guido van Rossum a principios de
los años 90.
\item Es multiplataforma (UNIX, Solaris, Linux, DOS, Windows, OS/2, Mac OS, etc.)
\item Software libre: Python Software Foundation License (PSFL)
\item Tipado din\'{a}mico.
\item Fuertemente tipado.
\item Orientado a objetos.
\end{itemize}
\end{frame}
\end{document}