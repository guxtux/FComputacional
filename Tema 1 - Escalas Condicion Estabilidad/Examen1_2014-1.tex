\documentclass[11pt]{article}
\usepackage[utf8]{inputenc}
%\usepackage[latin1]{inputenc}
\usepackage[spanish]{babel}
\decimalpoint
\usepackage{anysize}
\usepackage{graphicx} 
\usepackage{amsmath}
\usepackage{float}
\usepackage{tikz}
\usepackage{color}
\marginsize{1cm}{2cm}{0cm}{2cm}  
\title{Examen 1: Errores, condición y estabilidad. \\ Curso de Física Computacional}
\author{M. en C. Gustavo Contreras Mayén}
\date{ }
\begin{document}
\maketitle
\fontsize{14}{14}\selectfont
\textbf{Indicaciones: } Para cada uno de los problemas, deber\'{a}s de anotar tu c\'{o}digo en Python, adem\'{a}s de incluir gr\'{a}ficas en un archivo jpg, si es que lo menciona la pregunta.
\\
\\
En caso de que tengas alguna complicaci\'{o}n para resolver el problema, comenta dentro del mismo c\'{o}digo para que sepamos en d\'{o}nde se te presenta la dificultad.
\begin{enumerate}
\item Se sabe que
\[ \pi = 4 - 8 \sum_{k=1}^{\infty} \left( 16 k^{2} - 1 \right)^{-1} \]
¿Cu\'{a}ntas iteraciones se necesitan para producir el resultado con diez cifras decimales de exactitud?
\item Compara gr\'{a}ficamente el valor entre la funci\'{o}n y las primeras cinco sumas parciales de la serie
\[ \arctan(x) = \sum_{k=1}^{\infty} (-1)^{k+1} \dfrac{x^{2k-1}}{2k-1}\]
\item Usando la serie de Maclaurin truncada, una función $f(x)$ con $n$ derivadas continuas se puede aproximar con un polinomio de n-\'{e}simo grado
\[ f(x) \simeq p_{n}(x) = \sum_{i=0}^{n} c_{i} x^{i} \]
donde $c_{i} = \dfrac{f^{(i)}(0)}{i!}$
Genera y compara las gr\'{a}ficas para $f(x)= e^{x}$ y los polinomios $p_{2}(x)$, $p_{3}(x)$, $p_{4}(x)$, $p_{5}(x)$. Discute tus resultados.
\item Las siguientes expresiones definen a la constante de Euler
\begin{eqnarray}
\gamma &=& \lim_{n \rightarrow \infty} \left[ \sum_{k=1}^{n} \dfrac{1}{k} - ln (n) \right] \\
\gamma &=& \lim_{k \rightarrow \infty} \left[ \sum_{k=1}^{m} \dfrac{1}{k} - ln \left( m + \dfrac{1}{2} \right) \right]
\end{eqnarray}
Escribe un programa que calcule el valor de $\gamma = 0.57721$, ¿cu\'{a}l de las dos expresiones converge m\'{a}s r\'{a}pido al valor?
\item Si la siguiente funci\'{o}n se escribe en un programa, ¿en qu\'{e} rango de $x$ aparecer\'{a} un desborde o una divisi\'{o}n entre cero originados por el error de redondeo?
\[ f(x)=\dfrac{1}{1-tanh(x)} \]
Suponiendo que el n\'{u}mero positivo m\'{a}s pequeño es $3 \times 10^{-39}$  y el \'{e}psilon de la m\'{a}quina es $1.2 \times 10^{-7}$.
\item Identifica los n\'{u}meros de punto flotante correspondientes a las siguientes cadenas de bits
\begin{enumerate}
\item \fbox{0 \hspace{0.1cm} 00000000 \hspace{0.1cm} 00000000000000000000000}
\item \fbox{1 \hspace{0.1cm} 00000000 \hspace{0.1cm} 00000000000000000000000}
\item \fbox{0 \hspace{0.1cm} 11111111 \hspace{0.1cm} 00000000000000000000000}
\item \fbox{1 \hspace{0.1cm} 11111111 \hspace{0.1cm} 00000000000000000000000}
\item \fbox{0 \hspace{0.1cm} 00000001 \hspace{0.1cm} 00000000000000000000000}
\item \fbox{0 \hspace{0.1cm} 10000001 \hspace{0.1cm} 01100000000000000000000}
\item \fbox{0 \hspace{0.1cm} 01111111 \hspace{0.1cm} 00000000000000000000000}
\item \fbox{0 \hspace{0.1cm} 01111011 \hspace{0.1cm} 10011001100110011001100}
\end{enumerate}
\item Da la representaci\'{o}n en binario con precisi\'{o}n simple de los siguientes n\'{u}meros decimales
\begin{enumerate}
\item -9876.54321
\item 0.2343375
\item -285.75
\item $10^{2}$
\item $+0.0$ y $-0.0$
\end{enumerate}
\item Determina la expresi\'{o}n binaria de $\frac{1}{3}$. ¿Cu\'{a}l es la representaci\'{o}n en presici\'{o}n simple con una longitud de 32 bits? Compara tu respuesta con el valor que te devuelve python en la terminal.
\end{enumerate}
\end{document}