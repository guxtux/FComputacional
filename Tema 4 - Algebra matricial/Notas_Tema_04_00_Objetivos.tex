\documentclass[12pt]{beamer}
\usepackage{../Estilos/BeamerFC}
\usepackage{../Estilos/ColoresLatex}

\input{../Preambulos/preambulo_Beamer_Madrid_whale}
\usefonttheme{serif}

\title{\large{ObjetivosTema 4 - Álgebra matricial }}
\author{M. en C. Gustavo Contreras Mayén}
\date{8 de noviembre de 2022}

\begin{document}
\maketitle

\section*{Contenido}
\frame{\tableofcontents[currentsection, hideallsubsections]}

\section{Las matrices}
\frame{\tableofcontents[currentsection, hideothersubsections]}
\subsection{Introducción}

\begin{frame}
\frametitle{Las matrices en física}
Las matrices se encuentran entre los objetos matemáticos más importantes de la física.
\\
\bigskip
\pause
Dos problemas computacionales principales están asociados con las matrices: \pause \textbf{\textcolor{ao}{sistemas de ecuaciones lineales}} y \textbf{\textcolor{red}{problemas de valores propios}}.
\end{frame}
\begin{frame}
\frametitle{Sistemas de ecuaciones lineales}
Un sistema de ecuaciones lineales se puede escribir de la forma:
\pause
\begin{align*}
\mathbf{A \, x} = \mathbf{b}
\end{align*}
\end{frame}
\begin{frame}
\frametitle{Solución de sistemas de ecuaciones lineales}
Los sistemas de ecuaciones lineales con los coeficientes $b_{i}$ del lado derecho distintos de cero tienen una solución única cuando el determinante de la matriz $\mathbf{A}$ es distinto de cero.
\end{frame}
\begin{frame}
\frametitle{Solución de sistemas de ecuaciones lineales}
Si todos los coeficientes $b_{i}$ son iguales a cero, entonces la solución existe si, y solo si, el determinante de la matriz $\mathbf{A}$ es cero.
\end{frame}
\begin{frame}
\frametitle{Problemas de eigevalores}
El problema de eigenvalores es la clave para los cálculos de estructura de sistemas cuánticos en física atómica, molecular, nuclear y del estado sólido.
\end{frame}
\begin{frame}
\frametitle{Tipos de problemas}
Matemáticamente, el problema de eigenvalores se puede escribir como:
\pause
\begin{align*}
\mathbf{A \, x} = \lambda \, \mathbf{x}
\end{align*}
\end{frame}
\begin{frame}
\frametitle{Los sistemas lineales en fisica}
La linealización es una suposición o aproximación común en la descripción de procesos físicos y, por lo tanto, los sistemas lineales de ecuaciones son omnipresentes en la física computacional.
\end{frame}
\begin{frame}
\frametitle{Cálculo de eigenvalores}
De hecho, las manipulaciones de matrices asociadas con la búsqueda de eigenvalores es a menudo la mayor parte del trabajo involucrado en la resolución de muchos problemas físicos.
\end{frame}
\begin{frame}
\frametitle{Tipo de matrices}
En este Tema 4, discutiremos brevemente operaciones de matrices más no triviales.
\\
\bigskip
\pause
Nuestro tratamiento aquí se limitará en gran parte a los métodos directos apropiados para matrices \enquote{densas} (donde la mayoría de los elementos son distintos de cero) de dimensión inferior a varios cientos.
\end{frame}

\section{Objetivos}
\frame{\tableofcontents[currentsection, hideothersubsections]}
\subsection{Metas esperadas}

\begin{frame}
\frametitle{Objetivos}
Al concluir el Tema 4, se espera que el alumno:
\pause
\setbeamercolor{item projected}{bg=armygreen,fg=white}
\setbeamertemplate{enumerate items}{%
\usebeamercolor[bg]{item projected}%
\raisebox{1.5pt}{\colorbox{bg}{\color{fg}\footnotesize\insertenumlabel}}%
}
\begin{enumerate}[<+->] 
\item Identificará los problemas de eigenvalores con matrices simétricas.
\item Utilizará el método de Jacobi como método iterativo para diagonalizar una matriz y obtener los eigenvalores y eigenvectores.
\item Empleará los métodos de la potencia y potencia inversa para resolver problemas de eigenvalores.
\end{enumerate}
\end{frame}

\section{Evaluación}
\frame{\tableofcontents[currentsection, hideothersubsections]}
\subsection{Ejercicios a cuenta}

\begin{frame}
\frametitle{De los ejercicios a cuenta}
Se seguirá manejando una serie de ejercicios a cuenta para entregar.
\\
\bigskip
\pause
A menos de que se indique lo contrario, la entrega de los mismos se hará al siguiente sábado vía Moodle a más tardar a las 3 pm.
\end{frame}
\begin{frame}
\frametitle{Ejercicios en clase}
También se continuará con ejercicios en clase, por lo que se recomienda asistir a las sesiones presenciales.
\end{frame}

\subsection{Examen-Tarea}

\begin{frame}
\frametitle{Del Examen Tarea}
La segunda parte del Examen Parcial 2 del curso, corresponde a la parte del Tema 4.
\\
\bigskip
\pause
En la clase del jueves 10 de noviembre se entregará la lista de ejercicios a resolver.\pause La fecha de entrega del segundo parcial completo se precisará oportunamente.
\end{frame}


\end{document}