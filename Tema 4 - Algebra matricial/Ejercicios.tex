\documentclass[12pt]{article}
%\usepackage[latin1]{inputenc}
\usepackage[utf8x]{inputenc}
\usepackage[spanish,es-noshorthands]{babel}
\usepackage{amsmath}
\usepackage{amsthm}
\usepackage{graphicx}
\usepackage{wrapfig}
\usepackage{tabularx}
\usepackage{multicol}
\usepackage{float}
\usepackage{anysize}
\usepackage{tikz}
\usetikzlibrary{patterns}
\usetikzlibrary{decorations.pathmorphing,patterns}
\usetikzlibrary{arrows,calc,patterns,decorations.markings}
\usetikzlibrary{positioning}
\spanishdecimal{.}
\linespread{1.3}
\numberwithin{equation}{section}
\marginsize{1.5cm}{1.5cm}{0cm}{2cm}
\author{M. en C. Gustavo Contreras Mayén.}
\title{Algebra matricial - Examen Tarea \\ \begin{Large}Curso de Fí­sica Computacional\end{Large}}
\date{ }
\begin{document}
\maketitle
\begin{enumerate}
\item Evaluando el determinante, identifica cuáles de las siguientes matrices, son singulares, mal condicionadas o bien condicionadas:
\begin{multicols}{2}
	\begin{enumerate}
		\item $ \mathbf{A} =
				\begin{pmatrix}
					1 & 2 & 3 \\
					2 & 3 & 4 \\
					3 & 4 & 5
				\end{pmatrix} $
		\item $ \mathbf{A} =
				\begin{pmatrix}
					2.11 & -0.80 & 1.72 \\
					-1.84 & 3.03 & 1.29 \\
					-1.57 & 5.25 & 4.30
				\end{pmatrix} $
		\item $ \mathbf{A} =
				\begin{pmatrix}
					2 & -1 & 0 \\
					-1 & 2 & -1 \\
					0 & -1 & 2
				\end{pmatrix} $
		\item $ \mathbf{A} =
				\begin{pmatrix}
					4 & 3 & -1 \\
					7 & -2 & 3 \\
					5 & -18 & 13
				\end{pmatrix} $ 
	\end{enumerate}
\end{multicols}
\item Dada la descomposición $\mathbf{A} = \mathbf{LU}$, calcular $\mathbf{A}$ y $\vert \mathbf{A} \vert$
	 \begin{enumerate}
	 	\item $ \mathbf{L} =
	 			\begin{pmatrix}
	 				1 & 0 & 0 \\
	 				1 & 1 & 0 \\
	 				1 & 5/3 & 1
				\end{pmatrix} \hspace{0.5cm}
				\mathbf{U} =
				\begin{pmatrix}
					1 & 2 & 4 \\
					0 & 3 & 21 \\
					0 & 0 & 0
				\end{pmatrix} $
		\item $ \mathbf{L} =
	 			\begin{pmatrix}
	 				2 & 0 & 0 \\
	 				-1 & 1 & 0 \\
	 				1 & -3 & 1
				\end{pmatrix} \hspace{0.5cm}
				\mathbf{U} =
				\begin{pmatrix}
					2 & -1 & 1 \\
					0 & 1 & -3 \\
					0 & 0 & 1
				\end{pmatrix} $
	 \end{enumerate}
\item Usando los resultados de la descomposición LU
\[  \mathbf{A} =
	\mathbf{LU} =
	\begin{pmatrix}
		1 & 0 & 0 \\
		3/2 & 1  & 0 \\
		1/2 & 11/13 & 1
	\end{pmatrix}
	\begin{pmatrix}
		2 & -3 & -1 \\
		0 & 13/2 & -7/2 \\
		0 & 0 & 32/13
	\end{pmatrix}	 \]
	para resolver $\mathbf{Ax} = \mathbf{b}$, donde $\mathbf{b}^{T} = [1 \hspace{0.3cm} -1 \hspace{0.3cm} 2 ]$.
\item Resolver la ecuación $\mathbf{Ax} = \mathbf{b}$ con el método de eliminación de Gauss, donde
\[  \mathbf{A} =
	\begin{pmatrix}
		0 & 0 & 2 & 1 & 2 \\
		0 & 1 & 0 & 2 & -1 \\
		1 & 2 & 0 & -2 & 0 \\
		0 & 0 & 0 & -1 & 1 \\
		0 & 1 & -1 & 1 & -1
	\end{pmatrix} \hspace{1.5cm}
	\mathbf{b} =	
	\begin{pmatrix}
		1 \\
		1 \\
		-4 \\
		-2 \\
		-1
	\end{pmatrix} \]
\item Encontrar $\mathbf{L}$ y $\mathbf{U}$ tales que:
\[ \mathbf{A} = \mathbf{LU} =
	\begin{pmatrix}
		4 & -1 & 0 \\
		-1 & 4 & -4 \\
		0 & -1 & 4
	\end{pmatrix} \]
	usando a) la descomposición de Doolittle y b) la descomposición de Choleski.
\item Utiliza la descomposición de Doolittle para resolver $\mathbf{Ax}=\mathbf{b}$, donde
\[  \mathbf{A} =
	\begin{pmatrix}
		-3 & 6 & -4 \\
		9 & -8 & 24 \\
		-12 & 24 & -26 \\
	\end{pmatrix} \hspace{1.5cm}
	\mathbf{b} =	
	\begin{pmatrix}
		-3 \\
		65 \\
		-42
	\end{pmatrix} \]
\item Resolver $\mathbf{Ax} = \mathbf{b}$ por el método de descomposición de Doolittle, donde
\[  \mathbf{A} =
	\begin{pmatrix}
		2.34 & -4.10 & 1.78 \\
		-1.98 & 3.47 & -2.22 \\
		2.36 & -15.17 & 6.18 \\
	\end{pmatrix} \hspace{1.5cm}
	\mathbf{b} =	
	\begin{pmatrix}
		0.02 \\
		-0.73 \\
		-6.63
	\end{pmatrix} \]
\item Resolver $\mathbf{AX} = \mathbf{B}$ por el método de descomposición de Doolittle, donde
\[  \mathbf{A} =
	\begin{pmatrix}
		4 & -3 & 6 \\
		8 & -3 & 10 \\
		-4 & 12 & -10 \\
	\end{pmatrix} \hspace{1.5cm}
	\mathbf{B} =	
	\begin{pmatrix}
		1 & 0 \\
		0 & 1 \\
		0 & 0 
	\end{pmatrix} \]
\item Determinar $\mathbf{L}$ que resulta de la descomposición de Choleski para la matriz diagonal
\[ \begin{pmatrix}
		\alpha_{1} & 0 & 0 \ldots \\
		0 & \alpha_{2} & 0 & \ldots \\
		0 & 0 & \alpha_{3} & \ldots \\
		\vdots & \vdots &\vdots & \ddots
\end{pmatrix} \]
\item Un ejemplo clásico de una matriz mal condicionada, es la matriz de Hilbert
\[ \mathbf{A} = 
	\begin{bmatrix}
		1 & 1/2 & 1/3 & \ldots \\
		1/2 & 1/3 & 1/4 & \ldots \\
		1/3 & 1/4 & 1/5 & \ldots \\
		\vdots & \vdots & \vdots & \ddots
	\end{bmatrix} \]
Escribe un programa en python que resuelva el sistema $\mathbf{A}\mathbf{x} = \mathbf{b}$ por el método de Doolittle, donde $\mathbf{A}$ es una matriz de Hilbert arbitraria de $n \times n$ y \[ b_{i} = \sum_{j=1}^{n} A_{ij} \]
\\
\\
El programa no debe de utilizar un valor inicial para $n$, sino que en tiempo de ejecución, se determine para qué valor de $n$, la solución es exacta al menos hasta seis cifras significativas comparada con la solución exacta
\[\mathbf{x} = [1 \hspace{0.2cm} 1 \hspace{0.2cm} 1 \hspace{0.2cm} \ldots ]^{T} \]
\item Resolver las siguientes ecuaciones simétricas tridiagonales
\[ \begin{split}
4 x_{1} - x_{2} =& 9 \\
-x_{i-1} + 4x_{i} - x_{i+1} =& 5, \hspace{1cm} i=2,\ldots,n-1 \\
-x_{n-1} + 4 x_{n} =& 5
\end{split} \]
con $n=10$.
\item El sistema mostrado en la figura consiste en $n$ resortes lineales que soportan $n$ masas. La constante de los resortes se indican por $k_{i}$, mientras que el peso de las masas, es $W_{i}$ y $x_{i}$ son los desplazamientos de las masas (medidos de la posición donde el resorte no está deformado). La llamada \emph{formulación de desplazamiento} se obtiene escribiendo la ecuación de equilibrio para cada masa y sustituyendo $F_{i} = k_{i}(x_{i+1}-x_{i})$ para la fuerza en los resortes. El resultado es un conjunto de ecuaciones simétricas y tridiagonal:
\\
\begin{minipage}{0.5\textwidth}
\[ \begin{split}
(k_{1} + k_{2})x_{1} - k_{2}x_{2} =& W_{1} \\
-k_{i} x_{i-1} + (k_{i} + k_{i+1}) x_{i} - k_{i+1} x_{i+1} =& W_{i}, \hspace{1cm} i=2,3,\ldots,n-1 \\
-k_{n} x_{n-1} + k_{n} x_{n} =& W_{n}
\end{split} \]
\end{minipage}
\begin{minipage}{0.5\textwidth}
\begin{figure}[H]
\centering
\begin{tikzpicture}
\tikzstyle{spring}=[thick,decorate,decoration={zigzag,pre length=0.1cm,post
  length=0.1cm,segment length=6}]
	\draw (0,0) [pattern=north west lines] rectangle(1,0.25);
	\draw [spring](0.5,0) -- (0.5,-1) node [midway, right=0.4cm] {$k_{1}$}; 
	\draw (0,-1) rectangle node {$W_{1}$} (1,-2);
	\draw (1,-1) -- (1.5,-1);
	\draw [->]  (1.3,-1) -- node [near end, right=0.2]{$x_{1}$}(1.3,-1.7); 
	\draw [spring](0.5,-2) -- (0.5,-3) node [midway, right=0.4cm] {$k_{2}$};
	\draw (0,-3) rectangle node {$W_{2}$} (1,-4);
	\draw (1,-3) -- (1.5,-3);
	\draw [->]  (1.3,-3) -- node [near end, right=0.2]{$x_{2}$}(1.3,-3.7); 
	\draw [spring](0.5,-4) -- (0.5,-5) node [midway, right=0.4cm] {$k_{3}$};
	\draw (0.5,-5.1) node {$\vdots$};
	\draw [spring](0.5,-5.5) -- (0.5,-6.5) node [midway, right=0.4cm] {$k_{n}$};
	\draw (0,-6.5) rectangle node {$W_{n}$} (1,-7.5);
	\draw (1,-6.5) -- (1.5,-6.5);
	\draw [->]  (1.3,-6.5) -- node [near end, right=0.2]{$x_{n}$}(1.3,-7.2); 
\end{tikzpicture}
\end{figure}
\end{minipage}
\\
\\
\\
Escribe un programa que resuelva este conjunto de ecuaciones para los valores dados de $n$, $k$ y $W$. Considera $n=5$ y 
\[ \begin{split} 
k_{1} = k_{2} = k_{3} = 10 \mbox{ N/mm} \hspace{2cm} k_{4} = k_{5} = 5 \mbox{ N/mm} \\
W_{1} = W_{3} = W_{5} = 100 \mbox{ N} \hspace{2cm} W_{2} = W_{4} = 50 \mbox{ N}
\end{split} \]

\end{enumerate}	
\end{document}