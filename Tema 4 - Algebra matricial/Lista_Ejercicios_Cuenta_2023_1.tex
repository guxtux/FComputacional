\input{../Preambulos/preambulo_materiales}
\usetikzlibrary{patterns, arrows}
\usetikzlibrary{decorations.markings}
\usetikzlibrary{matrix}

\title{Ejericicios a cuenta Tema 4 \\ {\large Curso Física Computacional}}
\author{M. en C. Gustavo Contreras Mayén. \texttt{gux7avo@ciencias.unam.mx}}
\usepackage{tabularray}
\UseTblrLibrary{amsmath}
\date{ }
\begin{document}

\maketitle
\fontsize{14}{14}\selectfont
\noindent
Utilizando el método de Householder, reduce a su forma tridiagonal simétrica las siguientes matrices:
\begin{enumerate}
\item
\begin{align*}
\vb{A} =
\begin{+bmatrix}
2 & 5 & 6 & 8 & 1 & 9 & 5 & 4 & 7 & 6 \\
5 & 2 & 8 & 3 & 7 & 9 & 1 & 6 & 5 & 4 \\
6 & 8 & 1 & 7 & 6 & 4 & 9 & 3 & 2 & 8 \\
8 & 3 & 7 & 6 & 2 & 9 & 4 & 2 & 8 & 6 \\
1 & 7 & 6 & 2 & 9 & 6 & 1 & 3 & 8 & 4 \\
9 & 9 & 4 & 9 & 6 & 7 & 4 & 6 & 3 & 2 \\
5 & 1 & 9 & 4 & 1 & 4 & 6 & 8 & 7 & 2 \\
4 & 6 & 3 & 2 & 3 & 6 & 8 & 4 & 5 & 2 \\
7 & 5 & 2 & 8 & 8 & 3 & 7 & 5 & 1 & 6 \\
6 & 4 & 8 & 6 & 4 & 2 & 2 & 2 & 6 & 7
\end{+bmatrix}
\end{align*}
\item
\begin{align*}
\vb{A} = 
\begin{+bmatrix}
2 & 5 & 7 & 6 & 1 & 3 & 5 & 4 & 6 & 9 & 5 \\
5 & 1 & 8 & 4 & 9 & 7 & 6 & 2 & 3 & 4 & 8 \\
7 & 8 & 4 & 8 & 6 & 3 & 2 & 1 & 7 & 8 & 4 \\ 
6 & 4 & 8 & 1 & 3 & 8 & 6 & 5 & 4 & 7 & 7 \\
1 & 9 & 6 & 3 & 7 & 5 & 1 & 2 & 8 & 4 & 6 \\
3 & 7 & 3 & 8 & 5 & 2 & 9 & 7 & 6 & 2 & 5 \\
5 & 6 & 2 & 6 & 1 & 9 & 4 & 6 & 2 & 8 & 7 \\
4 & 2 & 1 & 5 & 2 & 7 & 6 & 2 & 8 & 4 & 3 \\
6 & 3 & 7 & 4 & 8 & 6 & 2 & 8 & 1 & 6 & 8 \\
9 & 4 & 8 & 7 & 4 & 2 & 8 & 4 & 6 & 7 & 1 \\
5 & 8 & 4 & 7 & 6 & 5 & 7 & 3 & 8 & 1 & 4    
\end{+bmatrix}    
\end{align*}
\item
\begin{align*}
\vb{A} =
\begin{+bmatrix}
5 & 8 & 3 & 9 & 4 & 6 & 2 & 8 & 9 & 4 & 1 & 6 & 7 \\
8 & 2 & 4 & 3 & 9 & 8 & 4 & 5 & 3 & 2 & 2 & 8 & 7 \\
3 & 4 & 1 & 5 & 9 & 6 & 3 & 8 & 6 & 5 & 2 & 6 & 5 \\
9 & 3 & 5 & 1 & 2 & 5 & 8 & 9 & 4 & 6 & 7 & 1 & 3 \\
4 & 9 & 9 & 2 & 3 & 5 & 6 & 9 & 8 & 4 & 1 & 5 & 7 \\
6 & 8 & 6 & 5 & 5 & 2 & 4 & 1 & 8 & 6 & 5 & 9 & 7 \\
2 & 4 & 3 & 8 & 6 & 4 & 1 & 5 & 4 & 9 & 7 & 5 & 2 \\
8 & 5 & 8 & 9 & 9 & 1 & 5 & 2 & 7 & 6 & 5 & 3 & 1 \\
9 & 3 & 6 & 4 & 8 & 8 & 4 & 7 & 1 & 4 & 6 & 8 & 2 \\
4 & 2 & 5 & 6 & 4 & 6 & 9 & 6 & 4 & 3 & 7 & 5 & 4 \\
1 & 2 & 2 & 7 & 1 & 5 & 7 & 5 & 6 & 7 & 2 & 8 & 6 \\
6 & 8 & 6 & 1 & 5 & 9 & 5 & 3 & 8 & 5 & 8 & 1 & 2 \\
7 & 7 & 5 & 3 & 7 & 7 & 2 & 1 & 2 & 4 & 6 & 2 & 8
\end{+bmatrix}
\end{align*}
\end{enumerate}
\newpage
\textbf{Notas importantes}:
\begin{enumerate}[label=\roman*.]
\item Resolver con código los tres incisos otorga $1$ punto.
\item Realizar a mano todo el cálculo de la reducción tridiagonal del inciso $1$, otorgará $2$ puntos.
\item Realizar a mano todo el cálculo de la reducción tridiagonal del inciso $3$, otorgará $4$ puntos.
\end{enumerate}
El puntaje mínimo para estos ejercicios es de $1$ (entregando el inciso $1$) y el puntaje máximo es $7$ puntos (entregando los incisos $1$, $2$ y $3$).
\end{document}