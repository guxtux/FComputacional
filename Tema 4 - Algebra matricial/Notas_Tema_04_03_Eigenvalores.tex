\documentclass[12pt]{beamer}
\usepackage{../Estilos/BeamerFC}
\usepackage{../Estilos/ColoresLatex}
\usepackage{courier}
\usepackage{listingsutf8}
\usepackage{listings}
\usepackage{xcolor}
\usepackage{textcomp}
\usepackage{color}
\definecolor{deepblue}{rgb}{0,0,0.5}
\definecolor{brown}{rgb}{0.59, 0.29, 0.0}
\definecolor{OliveGreen}{rgb}{0,0.25,0}
% \usepackage{minted}

\DeclareCaptionFont{white}{\color{white}}
\DeclareCaptionFormat{listing}{\colorbox{gray}{\parbox{0.98\textwidth}{#1#2#3}}}
\captionsetup[lstlisting]{format=listing,labelfont=white,textfont=white}
\renewcommand{\lstlistingname}{Código}


\definecolor{Code}{rgb}{0,0,0}
\definecolor{Keywords}{rgb}{255,0,0}
\definecolor{Strings}{rgb}{255,0,255}
\definecolor{Comments}{rgb}{0,0,255}
\definecolor{Numbers}{rgb}{255,128,0}

\makeatletter

\newif\iffirstchar\firstchartrue
\newif\ifstartedbyadigit
\newif\ifprecededbyequalsign

\newcommand\processletter
{%
  \ifnum\lst@mode=\lst@Pmode%
    \iffirstchar%
        \global\startedbyadigitfalse%
      \fi
      \global\firstcharfalse%
    \fi
}

\newcommand\processdigit
{%
  \ifnum\lst@mode=\lst@Pmode%
      \iffirstchar%
        \global\startedbyadigittrue%
      \fi
      \global\firstcharfalse%
  \fi
}

\lst@AddToHook{OutputOther}%
{%
  \lst@IfLastOtherOneOf{=}
    {\global\precededbyequalsigntrue}
    {}%
}

\lst@AddToHook{Output}%
{%
  \ifprecededbyequalsign%
      \ifstartedbyadigit%
        \def\lst@thestyle{\color{orange}}%
      \fi
    \fi
  \global\firstchartrue%
  \global\startedbyadigitfalse%
  \global\precededbyequalsignfalse%
}

\lstset{ 
language=Python,                % choose the language of the code
basicstyle=\footnotesize\ttfamily,       % the size of the fonts that are used for the code
numbers=left,                   % where to put the line-numbers
numberstyle=\scriptsize,      % the size of the fonts that are used for the line-numbers
stepnumber=1,                   % the step between two line-numbers. If it is 1 each line will be numbered
numbersep=5pt,                  % how far the line-numbers are from the code
backgroundcolor=\color{white},  % choose the background color. You must add \usepackage{color}
showspaces=false,               % show spaces adding particular underscores
showstringspaces=false,         % underline spaces within strings
showtabs=false,                 % show tabs within strings adding particular underscores
frame=single,   		% adds a frame around the code
tabsize=2,  		% sets default tabsize to 2 spaces
captionpos=t,   		% sets the caption-position to bottom
breaklines=true,    	% sets automatic line breaking
breakatwhitespace=false,    % sets if automatic breaks should only happen at whitespace
escapeinside={| |},  % if you want to add a comment within your code
stringstyle =\color{OliveGreen},
otherkeywords={as, np.array, np.concatenate, np.linspace, linspace, interpolate.interp1d, kind, plt.plot, .copy, np.arange, np.cos, np.pi, lw, ls, label, splrep, splev, plt.legend, loc, plt.title, plt.ylim, plt.show, sign, math.ceil, math.log, np.sqrt, np.exp, np.zeros, plt.xlabel, plt.ylabel, plt.xlim, np.identity, random, np.dot, np.outer, np.diagonal },             % Add keywords here
keywordstyle = \color{blue},
commentstyle = \color{darkcerulean},
identifierstyle = \color{black},
literate=%
         {á}{{\'a}}1
         {é}{{\'e}}1
         {í}{{\'i}}1
         {ó}{{\'o}}1
         {ú}{{\'u}}1
%
%keywordstyle=\ttb\color{deepblue}
%fancyvrb = true,
}

\lstdefinestyle{FormattedNumber}{%
    literate={0}{{\textcolor{red}{0}}}{1}%
             {1}{{\textcolor{red}{1}}}{1}%
             {2}{{\textcolor{red}{2}}}{1}%
             {3}{{\textcolor{red}{3}}}{1}%
             {4}{{\textcolor{red}{4}}}{1}%
             {5}{{\textcolor{red}{5}}}{1}%
             {6}{{\textcolor{red}{6}}}{1}%
             {7}{{\textcolor{red}{7}}}{1}%
             {8}{{\textcolor{red}{8}}}{1}%
             {9}{{\textcolor{red}{9}}}{1}%
             {.0}{{\textcolor{red}{.0}}}{2}% Following is to ensure that only periods
             {.1}{{\textcolor{red}{.1}}}{2}% followed by a digit are changed.
             {.2}{{\textcolor{red}{.2}}}{2}%
             {.3}{{\textcolor{red}{.3}}}{2}%
             {.4}{{\textcolor{red}{.4}}}{2}%
             {.5}{{\textcolor{red}{.5}}}{2}%
             {.6}{{\textcolor{red}{.6}}}{2}%
             {.7}{{\textcolor{red}{.7}}}{2}%
             {.8}{{\textcolor{red}{.8}}}{2}%
             {.9}{{\textcolor{red}{.9}}}{2}%
             {\ }{{ }}{1}% handle the space
         ,%
          %mathescape=true
          escapeinside={__}
          }



\usepackage[siunitx]{circuitikz}
\usepackage{parskip}
\usetikzlibrary{arrows,patterns,shapes}
\usetikzlibrary{decorations.markings}
\usetikzlibrary{arrows}
\usepackage{blkarray}
\usetheme{Madrid}
\usecolortheme{whale}
%\useoutertheme{default}
\setbeamercovered{invisible}
% or whatever (possibly just delete it)
\setbeamertemplate{section in toc}[sections numbered]
\setbeamertemplate{subsection in toc}[subsections numbered]
\setbeamertemplate{subsection in toc}{\leavevmode\leftskip=3.2em\rlap{\hskip-2em\inserttocsectionnumber.\inserttocsubsectionnumber}\inserttocsubsection\par}
% \setbeamercolor{section in toc}{fg=blue}
% \setbeamercolor{subsection in toc}{fg=blue}
% \setbeamercolor{frametitle}{fg=blue}
\setbeamertemplate{caption}[numbered]

\setbeamertemplate{footline}
\beamertemplatenavigationsymbolsempty
\setbeamertemplate{headline}{}


\makeatletter
% \setbeamercolor{section in foot}{bg=gray!30, fg=black!90!orange}
% \setbeamercolor{subsection in foot}{bg=blue!30}
% \setbeamercolor{date in foot}{bg=black}
\setbeamertemplate{footline}
{
  \leavevmode%
  \hbox{%
  \begin{beamercolorbox}[wd=.333333\paperwidth,ht=2.25ex,dp=1ex,center]{section in foot}%
    \usebeamerfont{section in foot} \insertsection
  \end{beamercolorbox}%
  \begin{beamercolorbox}[wd=.333333\paperwidth,ht=2.25ex,dp=1ex,center]{subsection in foot}%
    \usebeamerfont{subsection in foot}  \insertsubsection
  \end{beamercolorbox}%
  \begin{beamercolorbox}[wd=.333333\paperwidth,ht=2.25ex,dp=1ex,right]{date in head/foot}%
    \usebeamerfont{date in head/foot} \insertshortdate{} \hspace*{2em}
    \insertframenumber{} / \inserttotalframenumber \hspace*{2ex} 
  \end{beamercolorbox}}%
  \vskip0pt%
}
\makeatother

\makeatletter
\patchcmd{\beamer@sectionintoc}{\vskip1.5em}{\vskip0.8em}{}{}
\makeatother

\usefonttheme{serif}
\DeclareSIUnit\megapascal{\mega\pascal}

\title{\large{Obteniendo eigenvalores}}
\subtitle{Tema 4 - Álgebra matricial}
\author{M. en C. Gustavo Contreras Mayén}
\date{}

\begin{document}
\maketitle

\section*{Contenido}
\frame{\frametitle{Contenido} \tableofcontents[currentsection, hideallsubsections]}

\section{Sucesión Sturm}
\frame[allowframebreaks]{\frametitle{Temas} \tableofcontents[currentsection, hideothersubsections]}
\subsection{Definiendo la sucesión}

\begin{frame}
\frametitle{Calculando los eigenvalores}
En principio, los eigenvalores de una matriz $\vb{A}$ pueden determinarse encontrando las raíces de la ecuación característica $\abs{\vb{A} - \lambda \vb{I}} = 0$.
\\
\bigskip
\pause
Este método no es práctico para matrices grandes, \pause porque la evaluación del determinante involucra $n^{3}/3$ multiplicaciones.
\end{frame}
\begin{frame}
\frametitle{Para el caso de una matriz tridiagonal}
Sin embargo, si la matriz es tridiagonal (también suponemos que es simétrica), su polinomio característico es:
\pause
\renewcommand{\arraystretch}{0.9}
\begin{align*}
P_{n} (\lambda) = \abs{\vb{A} - \lambda \vb{I}} = 
\mdet{
d_{1} {-} \lambda & c_{1} & 0 & 0 & \ldots & 0 \\
c_{1} & d_{2} {-} \lambda & c_{2} & 0 & \ldots & 0 \\
0 & c_{2} & d_{3} {-} \lambda & c_{3} & \ldots & 0 \\
0 & 0 & c_{3} & d_{4} {-} \lambda & \ldots & 0 \\
\vdots & \vdots & \vdots & \vdots & \ddots & \vdots \\
0 & 0 & \ldots & 0 & c_{n-1} & d_{n} {-} \lambda
}
\end{align*}
\end{frame}
\begin{frame}
\frametitle{Obteniendo el polinomio}
Se puede calcular con solo $3 (n - 1)$ multiplicaciones usando la siguiente secuencia de operaciones:
\pause
\begin{eqnarray}
\begin{aligned}
P_{0} (\lambda) &= 1 \\[0.5em] \pause
P_{1} (\lambda) &= d_{1} - \lambda \\[0.5em] \pause
P_{i} (\lambda) &= d_{i} - \lambda - c_{i-1}^{2} \, P_{i-2} (\lambda) \hspace{1cm} i = 2, 3, \ldots, n
\end{aligned}
\label{eq:ecuacion_09_49}
\end{eqnarray}
\end{frame}
\begin{frame}
\frametitle{Definiendo la sucesión}
Los polinomios:
\begin{align*}
P_{0} (\lambda), \, P_{1} (\lambda), \,  \ldots, \, P_{n} (\lambda)
\end{align*}
\pause 
forman la \textbf{\textcolor{ao}{sucesión de Sturm}}.
\end{frame}
\begin{frame}
\frametitle{Propiedad de la sucesión}
Que tiene la siguiente propiedad: \pause El número de cambios de signo en la sucesión
\begin{align*}
P_{0} (a), P_{1} (a), \ldots, P_{n} (a)
\end{align*}
es igual al número de raíces de $P_{n} (\lambda)$ que son menores que $a$.
\end{frame}
\begin{frame}
\frametitle{Propiedad de la sucesión}
Si un miembro $P_{i} (a)$ de la sucesión es cero, su signo debe tomarse opuesto al de $P_{i-1} (a)$.
\\
\bigskip
\pause
Como veremos más adelante, la propiedad de sucesión de Sturm hace posible poner entre paréntesis los valores propios de una matriz tridiagonal.
\end{frame}

\subsection{Función \texttt{sturmSuc}}

\begin{frame}
\frametitle{La función \texttt{sturmSuc}}
Dados \funcionazul{d}, \funcionazul{c} y $\lambda$, la función \funcionazul{sturmSuc} devuelve la sucesión de Sturm:
\pause
\begin{align*}
P_{0} (\lambda), P_{1} (\lambda) , \ldots, P_{n} (\lambda)
\end{align*}
\end{frame}
\begin{frame}
\frametitle{La función \texttt{numLambdas}}
La función \funcionazul{numLambdas} devuelve el número de cambios de signo en la sucesión: \pause como se mencionó anteriormente, esto es igual al número de valores propios que son menores que $\lambda$.
\end{frame}

\subsection{Ejercicio 1}

\begin{frame}
\frametitle{Enunciado del ejercicio 1}
Usando la propiedad de sucesión Sturm demuestra que el eigenvalor más pequeño de $\vb{A}$ está en el intervalo $(0.25, 0.5)$, donde:
\pause
\begin{align*}
\vb{A} = 
\begin{bmatrix}
2 & -1 & 0 & 0 \\
-1 & 2 & -1 & 0 \\
0 & -1 & 2 & -1 \\
0 & 0 & -1 & 2
\end{bmatrix}
\end{align*}
\end{frame}
\begin{frame}
\frametitle{Solución}
Tomando $\lambda = 0.5$, \pause tenemos que:
\pause
\begin{eqnarray*}
\begin{aligned}
d_{i} - \lambda &= 1.5 \\[0.5em] \pause
c_{i-1}^{2} &= 1
\end{aligned}
\end{eqnarray*}
\end{frame}
\begin{frame}
\frametitle{La secuencia Sturm}
Por lo que la secuencia Sturm es:
\begin{eqnarray*}
\begin{aligned}
P_{0} (0.5) &= 1 \\ \pause 
P_{1} (0.5) &= \pause 1.5 \\ \pause 
P_{2} (0.5) &= \pause 1.5 (1.5) - 1 =  \pause 1.25 \\ \pause 
P_{3} (0.5) &= \pause 1.5 (1.25) - 1.5 = \pause 0.375 \\ \pause 
P_{4} (0.5) &= \pause 1.5 (0.375) - 1.25 = \pause -0.6875
\end{aligned}
\end{eqnarray*}
\pause
Dado que la secuencia tiene un cambio de signo, entonces existe un eigenvalor más pequeño que $0.5$
\end{frame}
\begin{frame}
\frametitle{Repetimos el procedimiento}
Ahora con $\lambda = 0.25$, se repite el procedimiento, encontramos que:
\pause
\begin{align*}
d_{i} - \lambda &= 1.75 \\
c_{i}^{2} &= 1
\end{align*}
\end{frame}
\begin{frame}
\frametitle{La secuencia Sturm}
Por lo que la secuencia Sturm ahora es:
\begin{eqnarray*}
\begin{aligned}
P_{0} (0.25) &= 1 \\ \pause 
P_{1} (0.25) &= \pause 1.7 \\ \pause 
P_{2} (0.25) &= \pause 1.75 (1.75) - 1 =  \pause 2.0625 \\ \pause 
P_{3} (0.25) &= \pause 1.75 (2.0625) - 1.5 = \pause 1.8594 \\ \pause 
P_{4} (0.25) &= \pause 1.75 (1.8594) - 2.0625 = \pause 1.1915
\end{aligned}
\end{eqnarray*}
\pause
\end{frame}    
\begin{frame}
\frametitle{Resultado}
No tenemos cambio de signo en la sucesión, por lo que todos los eigenvalores son mayores que $0.25$
\\
\bigskip
\pause
Concluimos que:
\begin{align*}
0.25 < \lambda < 0.5
\end{align*}
\end{frame}
\begin{frame}
\frametitle{Ahora con el código}
Genera un archivo nuevo con el siguiente nombre:
\medskip
\texttt{Tema-04-Eigenvalores-Ejercicio-01.py}
\\
\medskip
\pause
Los archivos de la clase, será el ejercicio a cuenta de la sesión.
\end{frame}
\begin{frame}[allowframebreaks, fragile]
\frametitle{Parte del código}
\begin{lstlisting}[caption=Parte del código para la sucesión de Sturm]
from moduloMatrices import sturmSuc, numLambdas
import numpy as np

d = #
c = #

lambda1 = 0.25
lambda2 = 0.5

p1 = #
p2 = #

print('Sucesion Sturm con {0:}'.format(lambda1))
print(p1)
print('\nNumero de cambios de signo = {0:}'.format(numLambdas(p1)))
print()
print('Sucesion Sturm con {0:}'.format(lambda2))
print(p2)
print('\nNumero de cambios de signo = {0:}'.format(numLambdas(p2)))
\end{lstlisting}
\end{frame}

\section{Teorema de Gerschgorin}
\frame[allowframebreaks]{\frametitle{Temas} \tableofcontents[currentsection, hideothersubsections]}
\subsection{Descripción}

\begin{frame}
\frametitle{Utilidad del teorema}
El \textbf{\textcolor{armygreen}{teorema de Gerschgorin}} es útil para determinar los \emph{límites globales} de los eigenvalores de una matriz $\vb{A}$ de $n \times n$.
\end{frame}
\begin{frame}
\frametitle{¿Qué es lo global?}
El término \enquote{global} se refiere a los límites que acotan todos los eigenvalores.
\\
\bigskip
\pause
Veremos aquí una versión simplificada para una matriz simétrica.
\end{frame}
\begin{frame}
\frametitle{Comentando el teorema}
Si $\lambda$ es un valor propio de $\vb{A}$, entonces:
\pause
\begin{align*}
a_{i} - r_{i} \leq \lambda \leq a_{i} + r_{i}, \hspace{1.3cm} i = 1, 2, \ldots, n
\end{align*}
\pause
donde:
\pause
\begin{align}
a_{i} = A_{ii} \hspace{1.3cm} r_{i} = \nsum_{\substack{j=1 \\ j \neq i}}^{n} \abs{A_{ij}}
\label{eq:ecuacion_09_50}
\end{align}
\end{frame}
\begin{frame}
\frametitle{Cota para los eigenvalores}
Se sigue entonces que los límites para los eigenvalores más pequeños y más grandes están dados por:
\pause
\begin{align}
\lambda_{\min} \geq \min_{i} (a_{i} - r_{i}) \hspace{1.2cm} \lambda_{\max} \leq \max_{i} (a_{i} + r_{i})
\end{align}
\end{frame}

\subsection{Ejercicio 2}

\begin{frame}
\frametitle{Ejercicio 2}
Con el teorema de Gerschgorin determina la cota para los eigenvalores de la matriz:
\pause
\begin{align*}
\vb{A} = 
\begin{bmatrix}
4 & 2 & 0 \\
-2 & 4 & -2 \\
0 & -2 & 5
\end{bmatrix}
\end{align*}
\end{frame}
\begin{frame}
\frametitle{Solución}
De acuerdo con la ec. (\ref{eq:ecuacion_09_50}), se tiene que:
\pause
\begin{eqnarray*}
\begin{aligned}
a_{1} &= \pause 4 \hspace{1.2cm} \pause a_{2} = \pause 4 \hspace{1.2cm} \pause a_{3} = \pause 5 \\
r_{1} &= \pause 2 \hspace{1.2cm} \pause r_{2} = \pause 4 \hspace{1.2cm} \pause r_{3} = \pause 2
\end{aligned}
\end{eqnarray*}
\pause
Por lo que:
\pause
\begin{eqnarray*}
\begin{aligned}
\lambda_{\min} &\geq \min (a_{i} - r_{i}) = \pause 4 - 4 = 0 \\ \pause
\lambda_{\max} &\geq \max (a_{i} + r_{i}) = \pause 4 + 4 = 8
\end{aligned}
\end{eqnarray*}
\end{frame}
\begin{frame}[allowframebreaks, fragile]
\frametitle{El código de la función}
\begin{lstlisting}[caption=Función para establecer los paréntesis de los eigenvalores]
def gerschgorin(d, c):
    n = len(d)
    
    lamMin = d[0] - abs(c[0])
    lamMax = d[0] + abs(c[0])
    
    for i in range(1, n-1):
        lam = d[i] - abs(c[i]) - abs(c[i-1])
        if lam < lamMin: lamMin = lam
        lam = d[i] + abs(c[i]) + abs(c[i-1])
        if lam > lamMax: lamMax = lam
        
    lam = d[n-1] - abs(c[n-2])
    if lam < lamMin: lamMin = lam
    lam = d[n-1] + abs(c[n-2])
    if lam > lamMax: lamMax = lam
    
    return lamMin, lamMax
\end{lstlisting}
\end{frame}
\begin{frame}
\frametitle{Ahora con el código}
Usaremos la función para resolver el ejercicio, habrá que completar las instrucciones para obtener la solución.
\end{frame}
\begin{frame}[allowframebreaks, fragile]
\frametitle{Parte del código}
\begin{lstlisting}[caption=Parte del código para poner paréntesis a los eigenvalores]
from moduloMatrices import gerschgorin
import numpy as np

d = #
c = #
d[-1] = #
a, b = #

print('El intevalo para los eigenvalores es: ({0:}, {1:})'.format(a, b))
\end{lstlisting}
\end{frame}

\section{Eigenvalores entre paréntesis}
\frame{\tableofcontents[currentsection, hideothersubsections]}
\subsection{Combinando funciones}

\begin{frame}
\frametitle{Usando lo anterior}
La propiedad de la sucesión de Sturm utilizada junto con el teorema de Gerschgorin nos brindan una herramienta conveniente \textbf{\textcolor{lava}{para poner entre paréntesis}} cada eigenvalor de una matriz tridiagonal simétrica.
\end{frame}

\subsection{Ejercicio}

\begin{frame}
\frametitle{Enunciado del ejercicio}
Establece los paréntesis de los eigenvalores de la matrix $\vb{A}$:
\begin{align*}
\vb{A} = 
\begin{bmatrix}
4 & 2 & 0 \\
-2 & 4 & -2 \\
0 & -2 & 5
\end{bmatrix}
\end{align*}
\end{frame}
\begin{frame}
\frametitle{El procedimiento para resolver}
En el ejercicio anterior encontramos que todos los eigenvalores se encuentran en $(0, 8)$.
\\
\bigskip
\pause
Ahora dividimos este intervalo en dos y usamos la sucesión de Sturm para determinar el número de valores propios en $(0, 4)$.
\end{frame}
\begin{frame}
\frametitle{Bisectando el intervalo}
Con $\lambda = 4$, la secuencia es — ver las ecs. (\ref{eq:ecuacion_09_49})—:
\pause
\begin{eqnarray*}
\begin{aligned}
P_{0} (4) &= 1 \\[0.5em] \pause
P_{1} (4) &= 4 - 4 = 0 \\[0.5em] \pause
P_{2} (4) &= (4 - 4)(0) - 2^{2} (1) = -4 \\[0.5em] \pause
P_{3} (4) &= (5 - 4)(-4) - 2^{2} (0) = -4
\end{aligned}
\end{eqnarray*}
\end{frame}
\begin{frame}
\frametitle{Cambios en el signo}
Debido a que se asigna un valor cero al signo opuesto al del miembro anterior, \pause los signos en esta secuencia son (+, -, -, -).
\\
\bigskip
\pause
El cambio de signo demuestra la presencia de un eigenvalor en $(0, 4)$.
\end{frame}
\begin{frame}
\frametitle{Siguiente bisección}
Ahora hacemos la bisección en el intervalo $(4, 8)$ y calculamos la sucesión de Sturm con $\lambda = 6$:
\pause
\begin{eqnarray*}
\begin{aligned}
P_{0} (6) &= 1 \\[0.5em] \pause
P_{1} (6) &= 4 - 6 = -2 \\[0.5em] \pause
P_{2} (6) &= (4 - 6)(-2) - 2^{2} (1) = 0 \\[0.5em] \pause
P_{3} (6) &= (5 - 6)(0) - 2^{2} (-2) = 8
\end{aligned}
\end{eqnarray*}    
\end{frame}
\begin{frame}
\frametitle{Cambios de signo}
En la secuencia los signos son $(+, -, +, +)$, \pause lo que nos dice que hay dos eigenvalores en $(0, 6)$
\end{frame}
\begin{frame}
\frametitle{Conclusión}
Por lo tanto:
\begin{eqnarray*}
\begin{aligned}
0 \leq \lambda_{1} \leq 4 \\[0.5em] \pause
4 \leq \lambda_{2} \leq 6 \\[0.5em] \pause
6 \leq \lambda_{3} \leq 8
\end{aligned}
\end{eqnarray*}
\end{frame}

\subsection{\texttt{lambdaRango}}

\begin{frame}
\frametitle{Una función más}
La función \textoazul{lambdaRango} pone entre paréntesis los $N$ valores propios más pequeños de una matriz tridiagonal simétrica $\vb{A} = [ \vb{c} \backslash \vb{d} \backslash \vb{c}]$.
\\
\bigskip
\pause
Devuelve la secuencia $r_{0}, r_{1}, \ldots, r_{N}$, donde cada intervalo $(r_{i-1}, r_{1})$ contiene exactamente un eigenvalor.
\end{frame}
\begin{frame}
\frametitle{El algoritmo}
El algoritmo primero encuentra los límites de todos los eigenvalores por el teorema de Gerschgorin
\\
\bigskip
\pause
Entonces se usa el método de bisección junto con la propiedad de la sucesión de Sturm para determinar $r_{N}, r_{N-1}, \ldots, r_{0}$ en ese orden.
\end{frame}
\begin{frame}[allowframebreaks, fragile]
\frametitle{El código}
\begin{lstlisting}[caption=Código para poner en paréntesis los eigenvalores]
def lamRango(d, c, N):
    
    lamMin, lamMax = gerschgorin(d, c)
    r = np.ones(N+1)
    r[0] = lamMin
    
    for k in range(N, 0, -1):
        lam = (lamMax + lamMin)/2.0
        h = (lamMax - lamMin)/2.0
        
        for i in range(1000):
            p = sturmSuc(d, c, lam)
            numLam = numLambdas(p)
            h = h/2.0
            
            if numLam < k: lam = lam + h
            elif numLam > k: lam = lam - h
            else: break
        
        lamMax = lam
        r[k] = lam
    
    return r
\end{lstlisting}
\end{frame}
\begin{frame}[fragile]
\frametitle{Ejercicio con el código}
\begin{lstlisting}[caption=Parte del código para resolver el ejercicio]
from moduloMatrices import lamRango
import numpy as np

d = #
c = #
d[-1] = #
#
print(x)
\end{lstlisting}
\end{frame}


\section{Cálculo de eigenvalores}
\frame[allowframebreaks]{\tableofcontents[currentsection, hideothersubsections]}
\subsection{Juntando las funciones}

\begin{frame}
\frametitle{Procedimiento en conjunto}
Una vez que los eigenvalores deseados están entre paréntesis, \pause se pueden encontrar calculando las raíces de $P_{n} (\lambda) = 0$ con la bisección o con el método de Ridder.
\end{frame}
\begin{frame}[allowframebreaks, fragile]
\frametitle{La función para obtener los eigenvalores}
\begin{lstlisting}[caption=La función para calcular los eigenvalores de una matriz en banda 3]
def eigenvalores3(d, c, N):
    
    def f(x):
        p = sturmSuc(d, c, x)
        return p[len(p)-1]
    
    lam = np.zeros(N)
    r = lamRango(d, c, N)
    
    for i in range(N):
        lam[i] = ridder(f,r[i],r[i+1])
    
    return lam
\end{lstlisting}
\end{frame}
\begin{frame}
\frametitle{Ejercicio}
Con la función \funcionazul{eigenvalores3} calcula los tres eigenvalores más pequeños de la matriz de $100 \times 100$:
\pause
\begin{align*}
\vb{A} =
\begin{bmatrix}
2 & -1 & 0 & \ldots & 0 \\
-1 & 2 & -1 & \ldots & 0 \\
\vdots & \vdots & \ddots & \ddots & \vdots \\
0 & 0 & \ldots & -1 & 2
\end{bmatrix}
\end{align*}
\end{frame}
\begin{frame}[fragile]
\frametitle{Código para el ejercicio}
\begin{lstlisting}[caption=Código para el ejercicio]

# Aqui va el codigo ;-)

\end{lstlisting}
Agrega al inicio de \funcionazul{Ludescomp3} la línea:
\begin{verbatim}
c.setflags(write=True)
\end{verbatim}
\end{frame}

\subsection{Cálculo de eigenvectores}

\begin{frame}
\frametitle{Cuando se conoce una aproximación}
Si se conocen los eigenvalores (los valores aproximados serán lo suficientemente buenos), \pause la mejor manera de calcular los eigenvectores correspondientes es el método de potencia inversa con desplazamiento de valores propios.
\end{frame}
\begin{frame}
\frametitle{El método que ya revisamos}
Este método lo presentamos previamente, pero el algoritmo discutido no aprovechó las bandas.
\\
\bigskip
\pause
Presentamos una versión del método escrito para matrices tridiagonales simétricas.
\end{frame}

\subsection{\texttt{potenciaInversa3}}

\begin{frame}
\frametitle{La función para casos en banda 3}
Esta función \funcionazul{potenciaInversa3} es muy similar a \funcionazul{potenciaInversa} que trabajamos anterioemente, pero se ejecuta mucho más rápido porque aprovecha la estructura tridiagonal de la matriz.
\end{frame}
\begin{frame}[allowframebreaks, fragile]
\frametitle{La función}
\begin{lstlisting}[caption=Código para el cálculo de eigenvectores en 3 banda]
def potenciaInversa3(d, c, s, tol=1.0e-6):
    n = len(d)
    
    e = c.copy()
    dAst = d - s
    
    LUdescomp3(c, dAst, e)
    x = np.random.rand(n)
    xMag = np.sqrt(np.dot(x,x))
    x =x/xMag
    bandera = 0
    
    for i in range(30):
        xAnterior = x.copy()
        LUsoluc3(c, dAst, e, x)
        xMag = np.sqrt(np.dot(x,x))
        x = x/xMag
        
        if np.dot(xAnterior,x) < 0.0:
            sign = -1.0
            x = -x
        else: sign = 1.0
        
        if np.sqrt(np.dot(xAnterior - x,xAnterior - x)) < tol:
            return s + sign/xMag, x
    
    print('El método de la potencia inversa no converge')
\end{lstlisting}
\end{frame}

\subsection{Ejercicios}

\begin{frame}
\frametitle{Enunciado del ejercicio}
Calcula el décimo eigenvalor más pequeño de la matriz del ejercicio anterior:
\begin{align*}
\vb{A} =
\begin{bmatrix}
2 & -1 & 0 & \ldots & 0 \\
-1 & 2 & -1 & \ldots & 0 \\
\vdots & \vdots & \ddots & \ddots & \vdots \\
0 & 0 & \ldots & -1 & 2
\end{bmatrix}
\end{align*}
\end{frame}
\begin{frame}
\frametitle{Solución del ejercicio}
El código que mostramos calcula el n-ésimo eigenvalor de $\vb{A}$ con el método de la potencia inversa con desplazamiento de eigenvalor.
\end{frame}
\begin{frame}[allowframebreaks, fragile]
\frametitle{Parte de la solución}
\begin{lstlisting}[caption=Parte del código que resuelve el ejercicio]
from moduloMatrices import lamRango, potenciaInversa3
import numpy as np

N = 10
n = 100
d = #
c = #
r = #
s = (r[N-1] + r[N])/2.0
#  = potenciaInversa3(#)

print('Eigenvalor No. {0:} = {1:}'.format(N, lam))
\end{lstlisting}
\end{frame}
\begin{frame}
\frametitle{Ejercicio}
Calcula los tres eigenvalores más pequeños y los correspondientes eigenvectores de la matriz:
\renewcommand{\arraystretch}{0.9}\begin{align*}
\vb{A} = 
\begin{bmatrix}
11.0 & 2.0 & 3.0 & 1.0 & 4.0 \\
2.0 & 9.0 & 3.0 & 5.0 & 2.0 \\
3.0 & 3.0 & 15.0 & 4.0 & 3.0 \\
1.0 & 5.0 & 4.0 & 12.0 & 4.0 \\
4.0 & 2.0 & 3.0 & 4.0 & 17.0
\end{bmatrix}
\end{align*}
\end{frame}
\begin{frame}[allowframebreaks, fragile]
\frametitle{Parte del código}
\begin{lstlisting}[caption=Parte del código que resuelve el ejercicio]
from moduloMatrices import householder, calculaP, eigenvalores3, potenciaInversa3
import numpy as np

N = 3
a = np.array

xx = np.zeros((len(a), N))

d, c = 
p = 
lambdas = 

for i in range(N):
    s = lambdas[i] * 1.0000001
    lam, x = potenciaInversa3(d, c, s)
    xx[:,i] = x

xx = np.dot(p,xx)

print('Eigenvalores {0:}'.format(lambdas))
print('Eigenvectores: {0:}'.format(xx))
\end{lstlisting}
\end{frame}

\section{Ejercicios a cuenta}
\frame{\tableofcontents[currentsection, hideothersubsections]}
\subsection{Lista de ejercicios}

\begin{frame}
\frametitle{Ejercicios a cuenta - $0.5$ puntos}
Con el teorema de Gerschgorin determina los límites de los eigenvalores de:
\begin{align*}
\vb{A} = 
\begin{bmatrix}
10 & 4 & -1 \\
4 & 2 & 3 \\
-1 & 3 & 6
\end{bmatrix}
\hspace{1.5cm}
\vb{B} = 
\begin{bmatrix}
4 & 2 & -2 \\
2 & 5 & 3 \\
-2 & 3 & 4
\end{bmatrix}
\end{align*}
\end{frame}
\begin{frame}
\frametitle{Ejercicio - $0.5$ puntos}
Con la sucesión de Sturm demuestra que:
\begin{align*}
\vb{A} = 
\begin{bmatrix}
5 & -2 & 0 & 0 \\
-2 & 4 & -1 & 0 \\
0 & -1 & 4 & -2 \\
0 & 0 & -2 & 5
\end{bmatrix}
\end{align*}
\end{frame}
\begin{frame}
\frametitle{Ejercicio - $0.5$ puntos}
Indica el intervalo de los eigenvalores de:
\begin{align*}
\vb{A} = 
\begin{bmatrix}
2 & -1 & 0 & 0 \\
-1 & -2 & -1 & 0 \\
0 & -1 & 2 & -1 \\
0 & 0 & -1 & 1
\end{bmatrix}
\end{align*}
\end{frame}
\begin{frame}
\frametitle{Ejercicio - $1.5$ puntos}
Reescribe la función \funcionazul{lamRango(d, c, N)} de modo que ponga entre paréntesis los $N$ eigenvalores más grandes de una matriz tridiagonal.
\\
\bigskip
Use esta función para calcular los dos eigenvalores más grandes de la matriz de Hilbert de $6 \times 6$.
\end{frame}
\begin{frame}
\frametitle{Matriz para el ejercicio anterior}
\renewcommand{\arraystretch}{1.5}
\begin{align*}
\vb{A} =
\begin{bmatrix}
1 & \dfrac{1}{2} & \dfrac{1}{3} & \ldots & \dfrac{1}{6} \\
\dfrac{1}{2} & \dfrac{1}{3} & \dfrac{1}{4} & \ldots & \dfrac{1}{7} \\
\dfrac{1}{3} & \dfrac{1}{4} & \dfrac{1}{5} & \ldots & \dfrac{1}{8} \\
\vdots & \vdots & \vdots & \ddots & \vdots \\
\dfrac{1}{6} & \dfrac{1}{7} & \dfrac{1}{8} & \ldots & \dfrac{1}{11}
\end{bmatrix}
\end{align*}
\end{frame}
\end{document}