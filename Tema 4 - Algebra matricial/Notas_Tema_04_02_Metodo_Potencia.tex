\documentclass[12pt]{beamer}
\usepackage{../Estilos/BeamerFC}
\usepackage{../Estilos/ColoresLatex}
\usepackage{courier}
\usepackage{listingsutf8}
\usepackage{listings}
\usepackage{xcolor}
\usepackage{textcomp}
\usepackage{color}
\definecolor{deepblue}{rgb}{0,0,0.5}
\definecolor{brown}{rgb}{0.59, 0.29, 0.0}
\definecolor{OliveGreen}{rgb}{0,0.25,0}
% \usepackage{minted}

\DeclareCaptionFont{white}{\color{white}}
\DeclareCaptionFormat{listing}{\colorbox{gray}{\parbox{0.98\textwidth}{#1#2#3}}}
\captionsetup[lstlisting]{format=listing,labelfont=white,textfont=white}
\renewcommand{\lstlistingname}{Código}


\definecolor{Code}{rgb}{0,0,0}
\definecolor{Keywords}{rgb}{255,0,0}
\definecolor{Strings}{rgb}{255,0,255}
\definecolor{Comments}{rgb}{0,0,255}
\definecolor{Numbers}{rgb}{255,128,0}

\makeatletter

\newif\iffirstchar\firstchartrue
\newif\ifstartedbyadigit
\newif\ifprecededbyequalsign

\newcommand\processletter
{%
  \ifnum\lst@mode=\lst@Pmode%
    \iffirstchar%
        \global\startedbyadigitfalse%
      \fi
      \global\firstcharfalse%
    \fi
}

\newcommand\processdigit
{%
  \ifnum\lst@mode=\lst@Pmode%
      \iffirstchar%
        \global\startedbyadigittrue%
      \fi
      \global\firstcharfalse%
  \fi
}

\lst@AddToHook{OutputOther}%
{%
  \lst@IfLastOtherOneOf{=}
    {\global\precededbyequalsigntrue}
    {}%
}

\lst@AddToHook{Output}%
{%
  \ifprecededbyequalsign%
      \ifstartedbyadigit%
        \def\lst@thestyle{\color{orange}}%
      \fi
    \fi
  \global\firstchartrue%
  \global\startedbyadigitfalse%
  \global\precededbyequalsignfalse%
}

\lstset{ 
language=Python,                % choose the language of the code
basicstyle=\footnotesize\ttfamily,       % the size of the fonts that are used for the code
numbers=left,                   % where to put the line-numbers
numberstyle=\scriptsize,      % the size of the fonts that are used for the line-numbers
stepnumber=1,                   % the step between two line-numbers. If it is 1 each line will be numbered
numbersep=5pt,                  % how far the line-numbers are from the code
backgroundcolor=\color{white},  % choose the background color. You must add \usepackage{color}
showspaces=false,               % show spaces adding particular underscores
showstringspaces=false,         % underline spaces within strings
showtabs=false,                 % show tabs within strings adding particular underscores
frame=single,   		% adds a frame around the code
tabsize=2,  		% sets default tabsize to 2 spaces
captionpos=t,   		% sets the caption-position to bottom
breaklines=true,    	% sets automatic line breaking
breakatwhitespace=false,    % sets if automatic breaks should only happen at whitespace
escapeinside={| |},  % if you want to add a comment within your code
stringstyle =\color{OliveGreen},
otherkeywords={as, np.array, np.concatenate, np.linspace, linspace, interpolate.interp1d, kind, plt.plot, .copy, np.arange, np.cos, np.pi, lw, ls, label, splrep, splev, plt.legend, loc, plt.title, plt.ylim, plt.show, sign, math.ceil, math.log, np.sqrt, np.exp, np.zeros, plt.xlabel, plt.ylabel, plt.xlim, np.identity, random, np.dot, np.outer, np.diagonal },             % Add keywords here
keywordstyle = \color{blue},
commentstyle = \color{darkcerulean},
identifierstyle = \color{black},
literate=%
         {á}{{\'a}}1
         {é}{{\'e}}1
         {í}{{\'i}}1
         {ó}{{\'o}}1
         {ú}{{\'u}}1
%
%keywordstyle=\ttb\color{deepblue}
%fancyvrb = true,
}

\lstdefinestyle{FormattedNumber}{%
    literate={0}{{\textcolor{red}{0}}}{1}%
             {1}{{\textcolor{red}{1}}}{1}%
             {2}{{\textcolor{red}{2}}}{1}%
             {3}{{\textcolor{red}{3}}}{1}%
             {4}{{\textcolor{red}{4}}}{1}%
             {5}{{\textcolor{red}{5}}}{1}%
             {6}{{\textcolor{red}{6}}}{1}%
             {7}{{\textcolor{red}{7}}}{1}%
             {8}{{\textcolor{red}{8}}}{1}%
             {9}{{\textcolor{red}{9}}}{1}%
             {.0}{{\textcolor{red}{.0}}}{2}% Following is to ensure that only periods
             {.1}{{\textcolor{red}{.1}}}{2}% followed by a digit are changed.
             {.2}{{\textcolor{red}{.2}}}{2}%
             {.3}{{\textcolor{red}{.3}}}{2}%
             {.4}{{\textcolor{red}{.4}}}{2}%
             {.5}{{\textcolor{red}{.5}}}{2}%
             {.6}{{\textcolor{red}{.6}}}{2}%
             {.7}{{\textcolor{red}{.7}}}{2}%
             {.8}{{\textcolor{red}{.8}}}{2}%
             {.9}{{\textcolor{red}{.9}}}{2}%
             {\ }{{ }}{1}% handle the space
         ,%
          %mathescape=true
          escapeinside={__}
          }



\usepackage[siunitx]{circuitikz}
\usepackage{parskip}
\usetikzlibrary{arrows,patterns,shapes}
\usetikzlibrary{decorations.markings}
\usetikzlibrary{arrows}
\usepackage{blkarray}
\usetheme{Madrid}
\usecolortheme{whale}
%\useoutertheme{default}
\setbeamercovered{invisible}
% or whatever (possibly just delete it)
\setbeamertemplate{section in toc}[sections numbered]
\setbeamertemplate{subsection in toc}[subsections numbered]
\setbeamertemplate{subsection in toc}{\leavevmode\leftskip=3.2em\rlap{\hskip-2em\inserttocsectionnumber.\inserttocsubsectionnumber}\inserttocsubsection\par}
% \setbeamercolor{section in toc}{fg=blue}
% \setbeamercolor{subsection in toc}{fg=blue}
% \setbeamercolor{frametitle}{fg=blue}
\setbeamertemplate{caption}[numbered]

\setbeamertemplate{footline}
\beamertemplatenavigationsymbolsempty
\setbeamertemplate{headline}{}


\makeatletter
% \setbeamercolor{section in foot}{bg=gray!30, fg=black!90!orange}
% \setbeamercolor{subsection in foot}{bg=blue!30}
% \setbeamercolor{date in foot}{bg=black}
\setbeamertemplate{footline}
{
  \leavevmode%
  \hbox{%
  \begin{beamercolorbox}[wd=.333333\paperwidth,ht=2.25ex,dp=1ex,center]{section in foot}%
    \usebeamerfont{section in foot} \insertsection
  \end{beamercolorbox}%
  \begin{beamercolorbox}[wd=.333333\paperwidth,ht=2.25ex,dp=1ex,center]{subsection in foot}%
    \usebeamerfont{subsection in foot}  \insertsubsection
  \end{beamercolorbox}%
  \begin{beamercolorbox}[wd=.333333\paperwidth,ht=2.25ex,dp=1ex,right]{date in head/foot}%
    \usebeamerfont{date in head/foot} \insertshortdate{} \hspace*{2em}
    \insertframenumber{} / \inserttotalframenumber \hspace*{2ex} 
  \end{beamercolorbox}}%
  \vskip0pt%
}
\makeatother

\makeatletter
\patchcmd{\beamer@sectionintoc}{\vskip1.5em}{\vskip0.8em}{}{}
\makeatother

\usefonttheme{serif}
\DeclareSIUnit\megapascal{\mega\pascal}

\title{\large{Método de la potencia y potencia inversa}}
\subtitle{Tema 4 - Álgebra matricial}
\author{M. en C. Gustavo Contreras Mayén}
\date{}

\begin{document}
\maketitle

\section*{Contenido}
\frame{\tableofcontents[currentsection, hideallsubsections]}

\section{Potencia y potencia inversa}
\frame{\tableofcontents[currentsection, hideothersubsections]}
\subsection{Introducción}

\begin{frame}
\frametitle{Utilidad de los métodos de potencias}
Los métodos de potencias son importantes ya que son un medio sencillo para calcular los eigenvalores.
\\
\bigskip
\pause
Hay tres versiones de los métodos:
\end{frame}
\begin{frame}
\frametitle{Versiones de los métodos}
\setbeamercolor{item projected}{bg=blue-green,fg=blue(pigment)}
\setbeamertemplate{enumerate items}{%
\usebeamercolor[bg]{item projected}%
\raisebox{1.5pt}{\colorbox{bg}{\color{fg}\footnotesize\insertenumlabel}}%
}
\begin{enumerate}[<+->]
\item El método de la potencia: es una técnica iterativa que determina el eigenvalor dominante de una matriz (el eigenvalor con la mayor magnitud).

Adicionalmente, no solo se obtiene el eigenvalor, sino también un eigenvector asociado.
\seti
\end{enumerate}
\end{frame}
\begin{frame}
\frametitle{Versiones de los métodos}
\setbeamercolor{item projected}{bg=blue-green,fg=blue(pigment)}
\setbeamertemplate{enumerate items}{%
\usebeamercolor[bg]{item projected}%
\raisebox{1.5pt}{\colorbox{bg}{\color{fg}\footnotesize\insertenumlabel}}%
}
\begin{enumerate}[<+->]
\seti
\item El método de la potencia inversa: ocupa la potencia inversa de la matriz y encuentra el mínimo eigenvalor.
\item El método de la potencia inversa con desplazamiento.
\end{enumerate}
\end{frame}


\subsection{Método de potencia inversa}

\begin{frame}
\frametitle{Definición del método}
El método de potencia inversa es un algoritmo simple y eficiente que encuentra el eigenvalor más pequeño $\lambda_{1}$ y el eigenvector correspondiente $\vb{x}_{1}$ del problema de eigenvalores estándar:
\pause
\begin{align}
\vb{A \, x} = \lambda \, \vb{x}
\label{eq:ecuacion_09_27}
\end{align}
\end{frame}
\begin{frame}
\frametitle{Operación del método}
El método funciona así:
\pause
\begin{tcolorbox}[colback=bisque, title=Resumen del procedimiento]
Sea $\vb{v}$ un vector unitario (un vector aleatorio servirá). \\ \pause
Hagamos lo siguiente hasta que el cambio en $\vb{v}$ sea insignificante: \\ \pause
Resuelve $\vb{A \, z} = \vb{v}$ para el vector $\vb{z}$. \\ \pause
Calcula $\abs{\vb{z}}$. \\ \pause 
Calcula $\vb{v} = \vb{z} / \abs{\vb{z}}$.
\end{tcolorbox}
\end{frame}
\begin{frame}
\frametitle{Operación del método}
Al final del procedimiento tenemos:
\pause
\begin{align*}
\abs{z} = \pm \dfrac{1}{\lambda_{1}} \hspace{0.7cm} \text{y} \hspace{0.7cm} \vb{v} = \vb{x}_{1}
\end{align*}
\pause
El signo de $\lambda_{1}$ se determina de la siguiente manera:
\pause
\setbeamercolor{item projected}{bg=armygreen,fg=aquamarine}
\setbeamertemplate{enumerate items}{%
\usebeamercolor[bg]{item projected}%
\raisebox{1.5pt}{\colorbox{bg}{\color{fg}\footnotesize\insertenumlabel}}%
}
\begin{enumerate}[<+->]
\item Si $\vb{z}$ cambia de signo entre iteraciones sucesivas, $\lambda_{1}$ es negativo.
\item De lo contrario, utiliza el signo más.
\end{enumerate}
\end{frame}
\begin{frame}
\frametitle{¿Por qué funciona el método?}
Veamos ahora por qué funciona el método.
\\
\bigskip
\pause
Dado que los eigenvectores $\vb{x}_{i}$ de $\vb{A}$ son ortonormales (linealmente independientes), pueden usarse como base para cualquier vector n-dimensional.
\end{frame}
\begin{frame}
\frametitle{¿Por qué funciona el método?}
 Así, $\vb{v}$ y $\vb{z}$ admiten las representaciones únicas:
 \pause
\begin{align}
\vb{v} = \nsum_{i=1}^{n} v_{i} \, \vb{x}_{i} \hspace{1cm} \vb{z} = \nsum_{i=1}^{n} z_{i} \, \vb{x}_{i}
\label{eq:ecuacion_09_28}
\end{align}
donde $v_{i}$ y $z_{i}$ son las componentes de $\vb{v}$ y $\vb{z}$ con respecto a los eigenvectores $\vb{x}_{i}$.
\end{frame}
\begin{frame}
\frametitle{Sustituyendo las expresiones}
Sustituyendo en $\vb{A \, z} = \vb{v}$, se obtiene:
\pause
\begin{align*}
\vb{A} \, \nsum_{i=i}^{n} z_{i} \, \vb{x}_{i} - \nsum_{i=i}^{n} v_{i} \, \vb{x}_{i} = \vb{0}
\end{align*}
\pause
Pero $\vb{A} \,\vb{x}_{i} = \lambda \, \vb{x}_{i}$, por lo que:
\pause
\begin{align*}
\nsum_{i=1}^{n} \big(z_{i} \, \lambda_{i} - v_{i} \big) \, \vb{x}_{i} = \vb{0}
\end{align*}
\end{frame}
\begin{frame}
\frametitle{Resultado importante}
Entonces se tiene:
\pause
\begin{align*}
z_{i} = \dfrac{v_{i}}{\lambda_{i}}
\end{align*}
\pause
Se sigue de la ec. (\ref{eq:ecuacion_09_28}) que:
\pause
\begin{eqnarray}
\begin{aligned}[b]
\vb{z} &= \nsum_{i=1}^{n} \dfrac{v_{i}}{\lambda_{i}} \, \vb{x}_{i} = \pause \dfrac{1}{\lambda_{1}} \, \nsum_{i=1}^{n} v_{i} \, \dfrac{\lambda_{1}}{\lambda_{i}} \, \vb{x}_{i} = \\ \pause
&= \dfrac{1}{\lambda_{1}} \bigg( v_{1} \vb{x}_{1} + v_{2} \dfrac{\lambda_{1}}{\lambda_{2}} \vb{x}_{2} + v_{3} \dfrac{\lambda_{1}}{\lambda_{3}} \vb{x}_{3} + \ldots \bigg)
\end{aligned}
\label{eq:ecuacion_09_29}
\end{eqnarray}
\end{frame}
\begin{frame}
\frametitle{Sobre el coeficiente}
Como $\lambda_{1} / \lambda_{i} < 1 \, (i \neq 1)$, observamos que el coeficiente de $\vb{x}_{1}$ se ha vuelto más prominente en $\vb{z}$ que en $\vb{v}$, \pause por tanto, $\vb{z}$ \textbf{\textcolor{ao}{es una mejor aproximación}} a $\vb{x}_{1}$. \pause Esto completa el primer ciclo iterativo.
\end{frame}
\begin{frame}
\frametitle{Avanzando en las siguientes iteraciones}
En ciclos posteriores establecemos $\vb{v} = \vb{z}/ \abs{\vb{z}}$ y se repite el proceso.
\\
\bigskip
\pause
Cada iteración aumenta la dominancia del primer término en la ec. (\ref{eq:ecuacion_09_29}) para que el proceso converja a:
\pause
\begin{align*}
\vb{z} = \dfrac{1}{\lambda_{1}} v_{1} \vb{x}_{1} = \dfrac{1}{\lambda_{1}} \vb{x}_{1}
\end{align*}
(en este paso $v_{1} = 1$ ya que $\vb{v} =  \vb{x}_{1}$, por lo que $v_{1} = 1, v_{2} = v_{3} = \ldots = 0$)
\end{frame}
\begin{frame}
\frametitle{Forma no estándar}
El método de la potencia inversa también funciona con problemas no estándar de eigenvalores:
\pause
\begin{align}
\vb{A \, x} = \lambda \, \vb{B \, x}
\label{eq:ecuacion_09_30}
\end{align}
siempre que $\vb{A \, z} = \vb{v}$ en el algoritmo se reemplace por:
\pause
\begin{align}
\vb{A \, z} = \vb{B \, v}
\label{eq:ecuacion_09_31}
\end{align}
\end{frame}
\begin{frame}
\frametitle{Alternativa a problemas no estándar}
La alternativa es, por supuesto, transformar el problema a la forma estándar antes de utilizar el método de la potencia.
\end{frame}

\subsection{Potencia con desplazamiento}

\begin{frame}
\frametitle{Desplazamiento de eigenvalores}
Por inspección de la ec. (\ref{eq:ecuacion_09_29}) vemos que la velocidad de convergencia está determinada por la fuerza de la desigualdad $\abs{\lambda_{1}/\lambda_{2}} < 1$ (el segundo término de la ecuación).
\end{frame}
\begin{frame}
\frametitle{Condición de la convergencia}
Si $\abs{\lambda_{2}}$ está bien separado de $\abs{\lambda_{1}}$, la desigualdad es fuerte y la convergencia es rápida.
\\
\bigskip
\pause
Por el contrario, la proximidad de estos dos valores propios da como resultado una convergencia muy lenta.
\end{frame}
\begin{frame}
\frametitle{Mejorando la convergencia}
La razón de convergencia se puede mejorar mediante una técnica llamada \textbf{\textcolor{cinnamon}{desplazamiento de eigenvalores}}.
\\
\bigskip
\pause
Haciendo:
\pause
\begin{align}
\lambda = \lambda^{*} + s
\label{eq:ecuacion_09_32}
\end{align}
donde $s$ es el \enquote{desplazamiento}.
\end{frame}
\begin{frame}
\frametitle{Cambiando el problema}
El problema de eigenvalores ec. (\ref{eq:ecuacion_09_27}) se transforma a:
\pause
\begin{align*}
\vb{ A \, x} = (\lambda^{*} + s) \, \vb{x}
\end{align*}
\pause
de manera equivalente:
\begin{align}
\vb{A^{*} \, x} = \lambda^{*} \, \vb{x}
\label{eq:ecuacion_09_33}
\end{align}
donde
\begin{align}
\vb{A}^{*} = \vb{A} - s \, \vb{I}
\label{eq:ecuacion_09_34}
\end{align}
\end{frame}
\begin{frame}
\frametitle{Resolviendo el problema}
Resolviendo el problema transformado en la ec. (\ref{eq:ecuacion_09_33}) por el método de la potencia inversa, se obtiene $\lambda^{*}_{1}$ y $\vb{x}_{1}$, \pause donde $\lambda^{*}_{1}$ es el valor propio más pequeño de $\vb{A}^{*}$.
\\
\bigskip
\pause
El valor propio correspondiente del problema original, $\lambda = \lambda_{1}^{*} + s$, es por lo tanto, el \textbf{\textcolor{darkred}{eigenvalor más cercano a}} $s$.
\end{frame}
\begin{frame}
\frametitle{Utilidad del desplazamiento}
El cambio de eigenvalores tiene dos aplicaciones.
\pause
\setbeamercolor{item projected}{bg=electricblue,fg=eggplant}
\setbeamertemplate{enumerate items}{%
\usebeamercolor[bg]{item projected}%
\raisebox{1.5pt}{\colorbox{bg}{\color{fg}\footnotesize\insertenumlabel}}%
}
\begin{enumerate}
\item Una obvia es la determinación del valor propio más cercano a un cierto valor $s$.
\pause
Por ejemplo, si la velocidad de trabajo de un eje es de $s$ rpm, es imperativo asegurarse de que no existan frecuencias naturales (que están relacionadas con los eigenvalores) cercanas a esa velocidad.
\seti
\end{enumerate}
\end{frame}
\begin{frame}
\frametitle{Utilidad del desplazamiento}
\setbeamercolor{item projected}{bg=electricblue,fg=eggplant}
\setbeamertemplate{enumerate items}{%
\usebeamercolor[bg]{item projected}%
\raisebox{1.5pt}{\colorbox{bg}{\color{fg}\footnotesize\insertenumlabel}}%
}
\begin{enumerate}
\conti 
\item El cambio de eigenvalores también se utiliza para acelerar la convergencia.

Supongamos que estamos calculando el eigenvalor más pequeño $\lambda_{1}$ de la matriz $\vb{A}$. \pause La idea es introducir un desplazamiento $s$ que haga que $\lambda_{1}^{*}/\lambda_{2}^{*}$ sea lo más pequeño posible.
\end{enumerate}
\end{frame}
\begin{frame}
\frametitle{Revisando el segundo punto}
Como $\lambda_{1}^{*} = \lambda_{1} - s$, debemos elegir $s \approx \lambda_{1}$ (debe evitarse $s = \lambda_{1}$ para no tener una división por cero). 
\\
\bigskip
\pause
Por supuesto, este método solo funciona si tenemos una estimación previa de $\lambda_{1}$.
\end{frame}
\begin{frame}
\frametitle{Utilidad del método}
El método de potencia inversa con desplazamiento de eigenvalores es una herramienta particularmente poderosa para encontrar eigenvectores si se conocen los eigenvalores.
\\
\bigskip
\pause
Al desplazarse muy cerca de un eigenvalor, el eigenvector correspondiente se puede calcular en una o dos iteraciones.
\end{frame}

\subsection{Método de la potencia}

\begin{frame}
\frametitle{Definiendo el método}
El método de potencia converge al eigenvalor más alejado de cero y al eigenvector asociado.
\\
\bigskip
\pause
Es muy similar al método de potencia inversa, la única diferencia entre los dos métodos es el intercambio de $\mathbf{v}$ y $\mathbf{z}$ en la ec. (\ref{eq:ecuacion_09_28}).
\end{frame}
\begin{frame}
\frametitle{Esquema del procedimiento}
\begin{tcolorbox}[colback=lavendermist, title=Método de la potencia]
Sea $\vb{v}$ un vector unitario (un vector aleatorio servirá). \\ \pause
Hagamos lo siguiente hasta que el cambio en $\vb{v}$ sea insignificante: \\ \pause
\hspace{1cm} Calcula el vector $\vb{z} = \vb{A v}$. \\ \pause
\hspace{1cm} Calcula $\abs{\vb{z}}$. \\ \pause
\hspace{1cm} Calcula $\vb{v} = \vb{z}/\abs{\vb{z}}$.
\end{tcolorbox}
\end{frame}
\begin{frame}
\frametitle{Una vez completado el procedimiento}
Al final del procedimiento obtenemos:
\pause
\begin{align*}
\abs{z} = \pm \lambda_{n} \hspace{0.6cm} \vb{v} = \vb{x}_{n}
\end{align*}
el signo de $\lambda_{n}$ se determina de la misma forma que en el método de potencia inversa.
\end{frame}

\section{Funciones \texttt{python}}
\frame{\tableofcontents[currentsection, hideothersubsections]}
\subsection{\texttt{potenciaInversa}}

\begin{frame}
\frametitle{Lo que hace el código}
Dada la matriz $\vb{A}$ y el desplazamiento $s$, la función \funcionazul{potenciaInversa} devuelve el eigenvalor de $\vb{A}$ más cercano a $s$ y el eigenvector correspondiente.
\\
\bigskip
\pause
La matriz $\vb{A}^{*} = \vb{A} - s \, \vb{I}$ se descompone tan pronto como se forma, de modo que solo se necesita la fase de solución (sustitución hacia adelante y hacia atrás) en el ciclo iterativo.
\end{frame}
\begin{frame}
\frametitle{Para cierto tipo de matrices}
Si $\vb{A}$ tiene bandas, la eficiencia del programa podría mejorarse reemplazando \funcionazul{LUdescomp} y \funcionazul{LUsoluc} por funciones para matrices con bandas (\funcionazul{LUdescomp5} y \funcionazul{LUsoluc5}).
\end{frame}
\begin{frame}
\frametitle{Modificación en una línea}
La línea de programa que forma $\vb{A}^{*}$ también debe modificarse para que sea compatible con el esquema de almacenamiento utilizado para $\vb{A}$.
\end{frame}
\begin{frame}[allowframebreaks, fragile]
\frametitle{Código de la función de la potencia inversa}
\begin{lstlisting}[caption=Código completo para la potencia inversa]
def potenciaInversa(a, s, tol=1.0e-6):
    n = len(a)
    aAst = a - np.identity(n) * s
    aAst = LUdescomp(aAst)
    x = np.zeros(n)
    
    for i in range(n):
        x[i] = random()
        
    xMag = np.sqrt(np.dot(x,x))
    x =x/xMag
    
    for i in range(50):
        xAnterior = x.copy()
        x = LUsoluc(aAst, x)
        xMag = np.sqrt(np.dot(x,x))
        x = x/xMag
        
        if np.dot(xAnterior,x) < 0.0:
            sign = -1.0
            x = -x
        else: sign = 1.0
        
        if np.sqrt(np.dot(xAnterior - x, xAnterior - x)) < tol:
            return s + sign/xMag, x
    
    print('El método de la potencia inversa no converge')  
\end{lstlisting}
\end{frame}

\section{Ejercicios}
\frame{\tableofcontents[currentsection, hideothersubsections]}
\subsection{Ejercicio 1}

\begin{frame}
\frametitle{Enunciado del problema}
La matriz de esfuerzo que describe el estado de esfuerzo en un punto es:
\pause
\renewcommand{\arraystretch}{1}
\begin{align*}
\vb{S} =
\begin{bmatrix}
- 30 & 10 & 20 \\
10 & 40 & -50 \\
20 & - 50 & -10
\end{bmatrix} \, \si{\megapascal}
\end{align*}
Utiliza $\vb{v} = [1 \quad 0 \quad 0]^{T}$ como suposición inicial para el eigenvector.
\end{frame}
\begin{frame}[allowframebreaks, fragile]
\frametitle{Resolviendo el ejercicio}
\begin{lstlisting}[caption=Ejercicio para el método de la potencia]
import numpy as np

s = # aqui va la matriz

v = # aqui va el vector

for i in range(100):
    vAnterior = v.copy()
    z = np.dot(s,v)
    zMag = np.sqrt(np.dot(z,z))
    v = z/zMag
    
    if np.dot(vAnterior,v) < 0.0:
        sign = -1.0
        v = -v
    else: sign = 1.0
    
    if np.sqrt(np.dot(vAnterior - v, vAnterior - v)) < 1.0e-6: break

lam = sign*zMag

print('Total de iteraciones es: {0:}'.format(i))

print('El eigenvalor es: {0:}'.format(lam))
\end{lstlisting}
\end{frame}

\subsection{Ejercicio 2}

\begin{frame}
\frametitle{Enunciado del ejercicio 2}
Calcula el eigenvalor más pequeño $\lambda_{1}$ y el correspondiente eigenvector de la matriz:
\pause
\renewcommand{\arraystretch}{1}
\begin{align*}
\vb{A} = 
\begin{bmatrix}
11 & 2 & 3 & 1 & 4 \\
2 & 9 & 3 & 5 & 2 \\
3 & 3 & 15 & 4 & 3 \\
1 & 5 & 4 & 12 & 4 \\
4 & 2 & 3 & 4 & 17 \\
\end{bmatrix}
\end{align*}
\end{frame}
\begin{frame}
\frametitle{Revisando las iteraciones}
Utiliza el método de la potencia inversa:
\pause
\setbeamercolor{item projected}{bg=mahogany,fg=maize}
\setbeamertemplate{enumerate items}{%
\usebeamercolor[bg]{item projected}%
\raisebox{1.5pt}{\colorbox{bg}{\color{fg}\footnotesize\insertenumlabel}}%
}
\begin{enumerate}[<+->]
\item Determina cuántas iteraciones se ocuparon.
\item Con el método de la potencia inversa con desplazamiento, usando $\lambda_{1} \approx 5$, ¿cuántas iteraciones se realizaron? 
\end{enumerate}
\end{frame}
\begin{frame}[fragile]
\frametitle{Código para el ejercicio 2}
\begin{lstlisting}[caption=Ejercicio para el método de la potencia inversa con desplazamiento]
a = # aqui va la matriz
s = # aqui va el desplazamiento

lam, x = potenciaInversa(a, s)

print('\nEl eigenvalor es: {0:}'.format(lam))
print('\nEl eigenvector es: \n {0:}'.format(x))
\end{lstlisting}
\end{frame}

\subsection{Ejercicio 3}

\begin{frame}
\frametitle{Enunciado del ejercicio}
Una viga apuntalada en voladizo soporta una carga axial de compresión $P$.
\begin{figure}
    \centering
    \begin{tikzpicture}
        \draw (0, 0) rectangle (6, 0.25) node [below, midway] {$L$};
        \draw (6.25, 0.125) -- (7.5, 0.125) node [near end, pos=1.2] {$x$};
        \draw [fill, color=beaublue] (6, -0.25) rectangle (6.25, 0.5);
        \draw (6, -0.25) -- (6, 0.5);
        \draw (0, -0.1) circle (3pt);
        \draw [fill, color=beaublue] (-0.2, -0.2) rectangle (0.2, -0.4);
        \draw (-0.2, -0.2) -- (0.2, -0.2);
        \draw [-stealth, thick] (-1, 0.125) -- (0, 0.125);
        \node at (-1.4, 0.125) {$P$};
    \end{tikzpicture}
\end{figure}
\end{frame}
\begin{frame}
\frametitle{Enunciado del ejercicio}
Puede demostrarse que el desplazamiento lateral $u (x)$ de la viga satisface la ecuación diferencial:
\pause
\begin{align}
\nderivada{u}{4} + \dfrac{P}{E I} \, \sderivada{u} &= 0 \label{eq:ecuacion_09_06_a} \\[0.5em]
u (0) = \sderivada{u} &= 0 \hspace{0.7cm} u (L) = \pderivada{L} = 0 \label{eq:ecuacion_09_06_b}
\end{align}
\end{frame}
\begin{frame}
\frametitle{Por resolver}
\setbeamercolor{item projected}{bg=awesome,fg=aureolin}
\setbeamertemplate{enumerate items}{%
\usebeamercolor[bg]{item projected}%
\raisebox{1.5pt}{\colorbox{bg}{\color{fg}\footnotesize\insertenumlabel}}%
}
\begin{enumerate}
\item Demuestra que el análisis de pandeo de la viga da como resultado un problema de eigenvalores de la matriz si las derivadas se aproximan por diferencias finitas.
\conti
\end{enumerate}
\end{frame}
\begin{frame}
\frametitle{Por resolver}
\setbeamercolor{item projected}{bg=awesome,fg=aureolin}
\setbeamertemplate{enumerate items}{%
\usebeamercolor[bg]{item projected}%
\raisebox{1.5pt}{\colorbox{bg}{\color{fg}\footnotesize\insertenumlabel}}%
}
\begin{enumerate}
\seti
\item Calcula la carga de pandeo más pequeña de la viga, haciendo pleno uso de las matrices en bandas. Ejecuta el programa con $100$ nodos interiores ($n = 100)$.
\end{enumerate}
\end{frame}
\begin{frame}
\frametitle{Solución inciso 1}
Dividimos la viga en $n + 1$ segmentos de longitud $L/(n + 1)$ cada uno como se muestra.
\\
\bigskip
\pause
\begin{figure}
    \centering
    \begin{tikzpicture}
        \draw (0, 0) rectangle (6, 0.25) node [below, midway] {$L$};
        \draw (6.25, 0.125) -- (7.5, 0.125) node [near end, pos=1.2] {$x$};
        \draw [fill, color=beaublue] (6, -0.25) rectangle (6.25, 0.5);
        \draw (6, -0.25) -- (6, 0.5);
        \draw (0, -0.1) circle (3pt);
        \draw [fill, color=beaublue] (-0.2, -0.2) rectangle (0.2, -0.4);
        \draw (-0.2, -0.2) -- (0.2, -0.2);
        \draw [-stealth, thick] (-1.4, 0.125) -- (0, 0.125);
        \node at (-1.7, 0.125) {$P$};

        \foreach \x in {-1, 0, 1, 2}
            \node at (\x, 0.6) {\footnotesize $\x$};
        
        \node at (4, 0.6) {\footnotesize $n-1$};
        \node at (5, 0.6) {\footnotesize $n$};
        \node at (6, 0.8) {\footnotesize $n+1$};
        \node at (7, 0.6) {\footnotesize $n+2$};

        
        \draw [dashed] (0, 0) rectangle (-1, 0.25);
        \draw [dashed] (6, 0) rectangle (7, 0.25);

        \foreach \x in {1, 2, 4, 5}
            \draw (\x, 0) -- (\x, 0.25);
    \end{tikzpicture}
\end{figure}

\end{frame}
\begin{frame}
\frametitle{Solución inciso 1}
Reemplazando las derivadas de $u$ en la ec. (\ref{eq:ecuacion_09_06_a}) por diferencias finitas centrales de $\order{h^{2}}$ en los nodos interiores (nodos $1$ a $n$), obtenemos:
\pause
\begin{align*}
\dfrac{u_{i-2} - 4 u_{i-1} + 6 u_{i} - 4 u_{i+1} + u_{i+2}}{h^{4}} = \dfrac{P}{E I} \dfrac{-u_{i-1} + 2 u_{i} - u_{i-1}}{h^{2}} \\
i = 1, 2, \ldots, n
\end{align*}
\end{frame}
\begin{frame}
\frametitle{Avanzando en la solución}
Luego de multiplicar por $h^{4}$, las ecuaciones son:
\pause
\begin{eqnarray}
\begin{aligned}
u_{-1} - 4 u_{0} + 6 u_{1} - 4 u_{2} + u_{3} &= \lambda (-u_{0} + 2 u_{1} - u_{2}) \\ \pause
u_{0} - 4 u_{1} + 6 u_{2} - 4 u_{3} + u_{4} &= \lambda (-u_{1} + 2 u_{2} - u_{3}) \\ \pause 
\vdots \hspace{3cm} &{} \\ \pause
u_{n-3} - 4 u_{n-2} + 6 u_{n-1} - 4 u_{n} + u_{n+1} &= \lambda (-u_{n-2} + 2 u_{n-1} - u_{n}) \\ \pause 
u_{n-2} - 4 u_{n-1} + 6 u_{n} - 4 u_{n+1} + u_{n+2} &= \lambda (-u_{n-1} + 2 u_{n} - u_{n+1})
\end{aligned}
\label{eq:ecuacion_09_06_c}
\end{eqnarray}
\pause
con
\begin{align*}
\lambda = \dfrac{P h^{2}}{E I} = \dfrac{P L^{2}}{(n + 1)^{2} E I}
\end{align*}
\end{frame}
\begin{frame}
\frametitle{Ocupando las condiciones iniciales}
Los desplazamientos $u_{-1}, u_{0}, u_{n+1}$ y $u_{n+2}$ se cancelan utilizando las CI del enunciado.
\\
\bigskip\
\pause
Con la tabla que trabajamos en el Tema 3, las aproximaciones por diferencias finitas para las CI, se tiene:
\begin{align*}
u_{0} = 0\hspace{0.5cm} u_{1} = - u_{1} \hspace{0.5cm} u_{n+1} = 0 \hspace{0.5cm} u_{n+2} = u_{n}
\end{align*}
\end{frame}
\begin{frame}
\frametitle{Obteniendo una expresión}
Al sustituir las ecs. (\ref{eq:ecuacion_09_06_c}), nos lleva a un problema de eigenvalores de la forma $\vb{A x} = \lambda \vb{B x}$, donde:
\end{frame}
\begin{frame}
\frametitle{Matriz de coeficientes $\vb{A}$}
\renewcommand{\arraystretch}{1}
\begin{align*}
\vb{A} =
\begin{bmatrix}
5 & -4 & 1 & 0 & 0 & \ldots & 0 \\
-4 & 6 & -4 & 1 & 0 & \ldots & 0 \\
1 & -4 & 6 & -4 & 1 & \ldots & 0 \\
\vdots & \ddots & \ddots & \ddots & \ddots & \ddots & \vdots \\
0 & \ldots & 1 & -4 & 6 & -4 & 1 \\    
0 & \ldots & 0 & 0 & 1 & -4 & 7    
\end{bmatrix}
\end{align*}
\end{frame}
\begin{frame}
\frametitle{Matriz de coeficientes $\vb{B}$}
\renewcommand{\arraystretch}{1}
\begin{align*}
\vb{B} =
\begin{bmatrix}
2 & -1 & 0 & 0 & 0 & \ldots & 0 \\
-1 & 2 & -1 & 0 & 0 & \ldots & 0 \\
0 & -1 & 2 & -1 & 0 & \ldots & 0 \\
\vdots & \ddots & \ddots & \ddots & \ddots & \ddots & \vdots \\
0 & \ldots & 0 & -1 & 2 & -1 & 0 \\
0 & \ldots & 0 & 0 & 0 & -1 & 2    
\end{bmatrix}
\end{align*}
\end{frame}
\begin{frame}
\frametitle{Solución inciso 2}
La función \funcionazul{potenciaInversa5} que mostraremos a continuación devuelve el eigenvalor más pequeño y el eigenvector correspondiente de $\vb{A x} = \lambda \vb{B x}$, \pause donde $\vb{A}$ es una matriz pentadiagonal y $\vb{B}$ es una matriz dispersa (en este problema es tridiagonal).
\end{frame}
\begin{frame}
\frametitle{Solución inciso 2}
La matriz $\vb{A}$ se ingresa por sus diagonales $\vb{d}$, $\vb{e}$ y $\vb{f}$ como se hizo en el Tema 3 junto con la descomposición $\vb{LU}$.
\\
\bigskip
\pause
El algoritmo no usa $\vb{B}$ directamente, sino que llama a la función \funcionazul{Bv(v)} que proporciona el producto $\vb{B v}$. \pause No se utiliza el desplazamiento de eigenvalores.
\end{frame}
\begin{frame}
\frametitle{La función \texttt{potenciaInversa5}}
El código para la función \funcionazul{potenciaInversa} se encuentra dentro del módulo \funcionazul{moduloMatrices}.
\\
\bigskip
Mientras que la función \funcionazul{Bv(v)} está en el código principal .
\end{frame}
\begin{frame}[allowframebreaks, fragile]
\frametitle{Código función para matriz pentadiagonal}
\begin{lstlisting}[caption=Código para el método de la potencia inversa con matriz en banda 
n=5]
def potenciaInversa5(Bv, d, e, f, tol=1.0e-6):
    
    n = len(d)
    d, e, f = LUdescomp5(d, e, f)
    x = np.random.rand(n)
    xMag = np.sqrt(np.dot(x,x))
    x = x/xMag
    
    for i in range(30):
        xAnterior = x.copy()
        x = Bv(xAnterior)
        x = LUsoluc5(d, e, f, x)
        xMag = np.sqrt(np.dot(x,x))
        x = x/xMag
        
        if np.dot(xAnterior,x) < 0.0:
            sign = -1.0
            x = -x
        else: sign = 1.0
        
        if np.sqrt(np.dot(xAnterior - x,xAnterior - x)) < tol:
            return sign/xMag, x
    
    print('El metodo de la potencia inversa no converge')
\end{lstlisting}
\end{frame}
\begin{frame}[allowframebreaks, fragile]
\frametitle{Código principal}
\begin{lstlisting}[caption=Código para resolver el ejercicio]
from moduloMatrices import potenciaInversa5
import numpy as np

def Bv(v):
    n = len(v)
    z = np.zeros(n)
    z[0] = 2.0 * v[0] - v[1]
    
    for i in range(1, n-1):
        z[i] = -v[i-1] + 2.0 * v[i] - v[i+1]
        z[n-1] = -v[n-2] + 2.0 * v[n-1]
    
    return z

n = 100
d = np.ones(n) * 6.0
d[0] = 5.0
d[n-1] = 7.0
e = np.ones(n-1) * (-4.0)
f = np.ones(n-2) * 1.0

lam, x = potenciaInversa5(Bv, d, e, f)

print('PL**2/EI = {0:}'.format(lam*(n+1)**2))
\end{lstlisting}
\end{frame}
\begin{frame}
\frametitle{Resultado consistente}
El resultado que nos devuelve el código, concuerda con la solución analítica:
\pause
\begin{align*}
\dfrac{P L^{2}}{E I} = 20.186734173260856
\end{align*}
\end{frame}

\section{Reducción de Householder}
\frame{\tableofcontents[currentsection, hideothersubsections]}
\subsection{Definición}

\begin{frame}
\frametitle{Transformación más conveniente}
Hemos mencionado que las transformaciones de similitud se pueden usar para transformar un problema de eigenvalores a una forma que sea más fácil de resolver.
\\
\bigskip
\pause
La más deseable de las formas \enquote{fáciles} es, por supuesto, la forma diagonal que resulta del método de Jacobi.
\end{frame}
\begin{frame}
\frametitle{Detalle con el método}
Sin embargo, el método de Jacobi requiere alrededor de $10 n^{3}$ a $20 n^{3}$ multiplicaciones, por lo que la cantidad de cálculo aumenta muy rápidamente con $n$.
\end{frame}
\begin{frame}
\frametitle{Transformación más conveniente}
Por lo general, estaremos mejor si reducimos la matriz a la forma tridiagonal, \pause lo que se puede hacer precisamente en $n - 2$ transformaciones mediante el \textbf{\textcolor{darkmagenta}{método de Householder}}.
\end{frame}
\begin{frame}
\frametitle{Transformación más conveniente}
Una vez que se logra la forma tridiagonal, todavía tenemos que obtener los eigenvalores y los eigenvectores, pero existen medios efectivos para lidiar con eso, que veremos a continuación.
\end{frame}

\subsection{Matriz de Householder}

\begin{frame}
\frametitle{El procedimiento}
Caqda transformación de Householder utiliza la \textbf{\textcolor{drab}{matriz de Houlseholder}}:
\pause
\begin{align}
\vb{Q} = \vb{I} - \dfrac{\vb{u} \, \vb{U}^{T}}{H}
\label{eq:ecuacion_09_36}
\end{align}
\pause
donde 
$\vb{u}$ es un vector y 
\begin{align}
H = \dfrac{1}{2} \vb{u}^{T} \, \vb{u} = \dfrac{1}{2} \abs{\vb{u}}
\label{eq:ecuacion_09_37}
\end{align}
\pause
Nótese que $\vb{u} \, \vb{U}^{T}$ en la ec. (\ref{eq:ecuacion_09_36}) es el producto exterior, es decir, una matriz con los elementos $(\vb{u} \vb{u}^{T})_{ij} = u_{i} u_{j}$.
\end{frame}
\begin{frame}
\frametitle{De la matriz $\vb{Q}$}
Dado que la matriz $\vb{Q}$ es claramente simétrica $(\vb{Q}^{T} = \vb{Q})$, se puede escribir:
\pause
\begin{eqnarray*}
\begin{aligned}
\vb{Q}^{T} \vb{Q} &= \pause \vb{Q Q} = \pause \left( \vb{I} - \dfrac{\vb{u} \vb{u}^{T}}{H} \right) \left( \vb{I} - \dfrac{\vb{u} \vb{u}^{T}}{H} \right) = \\ \pause
&= \vb{I} - 2 \, \dfrac{\vb{u} \vb{u}^{T}}{H} + \dfrac{\vb{u} (\vb{u}^{T} \vb{u}) \vb{u}^{T}}{H} = \\ \pause
&= \vb{I} - 2 \, \dfrac{\vb{u} \vb{u}^{T}}{H} + \dfrac{\vb{u} (2 H) \vb{u}^{T}}{H^{2}} = \\ \pause
&= \vb{I}
\end{aligned}
\end{eqnarray*}
Lo que nos muestra que $\vb{Q}$ es también ortogonal.
\end{frame}
\begin{frame}
\frametitle{Eligiendo un vector}
Sea $\vb{x}$ un vector arbitrario y consideremos la transformación $\vb{Q x}$.
\\
\bigskip
\pause
Eligiendo:
\pause
\begin{align}
\vb{u} = x + k \, \vb{e}_{1}
\label{eq:ecuacion_09_38}
\end{align}
\pause
donde:
\begin{align*}
k = \pm \, \abs{x} \hspace{1.5cm} \vb{e}_{1} = \big[ 1 \quad 0 \quad 0 \quad \ldots \quad 0 \big]^{T}
\end{align*}
\end{frame}
\begin{frame}
\frametitle{Resultado obtenido}
Se obtiene entonces:
\pause
\begin{eqnarray*}
\begin{aligned}
\vb{Q x} &= \left( \vb{I} - \dfrac{\vb{u} \vb{u}^{T}}{H} \right) \, \vb{x} = \pause \left[ \vb{I} - \dfrac{\vb{x} (\vb{x} + k \vb{e}_{1})^{T}}{H} \right] \, \vb{x} = \\[0.5em] \pause
&= \vb{x} - \dfrac{\vb{u} (\vb{x}^{T} \vb{x} + k \, \vb{e}_{1}^{T} \, \vb{x})}{H} = \\[0.5em] \pause
&= \vb{x} - \dfrac{\vb{u} (k^{2} + k \,x_{1})}{H}
\end{aligned}
\end{eqnarray*}
\end{frame}
\begin{frame}
\frametitle{Usando un valor}
Pero:
\pause
\begin{eqnarray*}
\begin{aligned}
2 \, H &= \pause (\vb{x} + k \, \vb{e}_{1})^{T} \, (\vb{x} + k \, \vb{e}_{1}) = \\[0.5em] \pause
&= \abs{\vb{x}}^{2} + k (\vb{x}^{T} \, \vb{e}_{1} + \vb{e}_{1}^{T} \, \vb{x}) + k^{2} \, \vb{e}_{1}^{2} \, \vb{e}_{1} = \\[0.5em] \pause
&= k^{2} + 2 \, k \, x_{1} + k^{2} = \\[0.5em] \pause
&= 2 (k^{2} + k \, x_{1})
\end{aligned}
\end{eqnarray*}
\end{frame}
\begin{frame}
\frametitle{Resultado de la transformación}
Por lo que:
\pause
\begin{eqnarray}
\begin{aligned}
\vb{Q x} = \pause \vb{x} - \vb{u} = \pause -k \, \vb{e}_{1} = [ -k \quad 0 \quad 0 \quad \ldots \quad 0]^{T}
\label{eq:ecuacion_09_39}
\end{aligned}
\end{eqnarray}
\pause
La transformación elimina todos los elementos de $\vb{x}$ excepto el primero.
\end{frame}

\subsection{Reducción de Householder}

\begin{frame}
\frametitle{Transformando una matriz simétrica}
Apliquemos ahora la transformación a una matriz simétrica $\vb{A}$ de $n \times n$:
\pause
\begin{eqnarray}
\begin{aligned}
\vb{P}_{1} \, \vb{A} = 
\begin{bmatrix}
1 & \vb{0}^{T} \\
\vb{0} & \vb{Q}
\end{bmatrix}
\begin{bmatrix}
A_{11} & \vb{x}^{T} \\
\vb{x} & \pderivada{\vb{A}}
\end{bmatrix}
= \pause
\begin{bmatrix}
A_{11} & \vb{x}^{T} \\
\vb{Q x} & \vb{Q} \, \pderivada{\vb{A}}
\end{bmatrix}
\end{aligned}
\label{eq:ecuacion_09_40}
\end{eqnarray}
\end{frame}
\begin{frame}
\frametitle{Elementos de la transformación}
Aquí $\vb{x}$ representa la primera columna de $\vb{A}$ con el primer elemento omitido, \pause y $\pderivada{\vb{A}}$ es simplemente $\vb{A}$ con su primera fila y columna eliminadas.
\end{frame}
\begin{frame}
\frametitle{Elementos de la transformación}
La matriz $\vb{Q}$ de dimensiones $(n - 1) \times (n - 1)$ se construye utilizando las ecs. (\ref{eq:ecuacion_09_36}) - (\ref{eq:ecuacion_09_38}). 
\end{frame}
\begin{frame}
\frametitle{Reducción de elementos}
Haciendo referencia a la ec. (\ref{eq:ecuacion_09_39}), vemos que la transformación reduce la primera columna de $\vb{A}$ a:
\pause
\renewcommand{\arraystretch}{1}
\begin{align*}
\begin{bmatrix}
A_{11} \\
\vb{Q x}
\end{bmatrix}
=
\begin{bmatrix}
A_{11} \\
-k \\
0 \\
\vdots \\
0
\end{bmatrix}
\end{align*}
\end{frame}
\begin{frame}
\frametitle{Resultado de la transformación}
La transformación:
\pause
\begin{align}
\vb{A} \leftarrow \vb{P}_{1} \, \vb{A} \, \vb{P}_{1} =
\begin{bmatrix}
A_{11} & (\vb{Q x})^{T} \\
\vb{Q x} & \vb{Q \pderivada{A} Q}
\end{bmatrix}
\label{eq:ecuacion_09_41}
\end{align}
por lo tanto, tridiagonaliza la primera fila y la primera columna de $\vb{A}$.
\end{frame}
\begin{frame}
\frametitle{Transformación para otra matriz}
Aquí hay un diagrama de la transformación para una matriz de $4 \times 4$:
\pause
\scalebox{0.9}{
\begin{tabular}{c c c c c}
\renewcommand{\arraystretch}{1}
\begin{tabular}{| c | c c c |} \hline
$1$ & $0$ & $0$ & $0$ \\ \hline
$0$ & & & \\
$0$ & & $Q$ & \\
$0$ & & & \\ \hline
\end{tabular}
& $\cdot$ &
\begin{tabular}{| c | c c c |} \hline
$A_{11}$ & $A_{12}$ & $A_{13}$ & $A_{14}$ \\ \hline
$A_{21}$ & & & \\
$A_{31}$ & & $\pderivada{\vb{A}}$ & \\
$A_{41}$ & & & \\ \hline
\end{tabular}
& $\cdot$ &
\begin{tabular}{| c | c c c |} \hline
$1$ & $0$ & $0$ & $0$ \\ \hline
$0$ & & & \\
$0$ & & $Q$ & \\
$0$ & & & \\ \hline
\end{tabular} = 
\end{tabular} }
\end{frame}
\begin{frame}
\frametitle{Transformación para otra matriz}
\scalebox{0.9}{
= \begin{tabular}{| c | c c c |} \hline
$A_{11}$ & $-k$ & $0$ & $0$ \\ \hline
$-k$ & & & \\
$0$ & & $\vb{Q} \pderivada{\vb{A}} \vb{Q}$ & \\
$0$ & & & \\ \hline
\end{tabular}}
\\
La segunda fila y columna de $\vb{A}$ se reducen a continuación aplicando la transformación a la parte inferior derecha de $3 \times 3$ de la matriz.
\end{frame}
\begin{frame}
\frametitle{Denotando la transformación}
Esta transformación puede expresarse como $\vb{A} \leftarrow \vb{P}_{2} \, \vb{A} \, \vb{P}_{2}$, donde ahora:
\pause
\begin{align}
\vb{P}_{2} =
\begin{bmatrix}
\vb{I}_{2} & \vb{0}^{T} \\
\vb{0} & \vb{Q}
\end{bmatrix}
\label{eq:ecuacion_09_42}
\end{align}
\end{frame}
\begin{frame}
\frametitle{Identificando un componente}
En la ec. (\ref{eq:ecuacion_09_42}) $\vb{I}_{2}$ es una matriz identidad de $2 \times 2$ y $\vb{Q}$ es una matriz de $(n - 2) \times (n - 2)$ construida eligiendo para $\vb{x}$ los $n - 2$ elementos inferiores de la segunda columna de $\vb{A}$.
\end{frame}
\begin{frame}
\frametitle{Número de transformaciones}
Se necesita un total de $n - 2$ transformaciones con:
\pause
\begin{align*}
\vb{P}_{i} =
\begin{bmatrix}
\vb{I}_{i} & \vb{0}^{T} \\
\vb{0} & \vb{Q}
\end{bmatrix} \hspace{1cm} i = 1, 2, \ldots, n - 2
\end{align*}
para alcanzar la forma tridiagonal.
\end{frame}
\begin{frame}
\frametitle{Desventaja en el procedimiento}
Es un desperdicio formar $\vb{P}_{i}$ y luego realizar la multiplicación de matrices $\vb{P}_{i} \vb{A} \vb{P}_{i}$.
\end{frame}
\begin{frame}
\frametitle{titulo}
Notamos que:
\pause
\begin{eqnarray*}
\begin{aligned}
\pderivada{\vb{A}} \, \vb{Q} &= \pderivada{\vb{A}} \left( \vb{I} - \dfrac{\vb{u} \vb{u}^{T}}{H} \right) = \\ \pause
&= \pderivada{\vb{A}} - \dfrac{\pderivada{\vb{A}} \vb{u}}{H} \, \vb{u}^{T} = \\ \pause
&= \pderivada{\vb{A}} - \vb{v} \vb{u}^{T}
\end{aligned}
\end{eqnarray*}
\pause
donde:
\begin{align}
\vb{v} = \dfrac{\pderivada{\vb{A}} \vb{u}}{H}
\label{eq:ecuacion_09_43}
\end{align}
\end{frame}
\begin{frame}
\frametitle{Resultado importante}
Por lo que:
\begin{eqnarray*}
\begin{aligned}
\vb{Q} \pderivada{\vb{A}} \vb{Q} &= \left( \vb{I} - \dfrac{\vb{u} \vb{u}^{T}}{H} \right) \big( \pderivada{\vb{A}} - \vb{v} \vb{u}^{T} \big) = \\[0.3em] \pause
&= \pderivada{\vb{A}} - \vb{v} \vb{u}^{T} - \dfrac{\vb{u} \vb{u}^{T}}{H} \left( \pderivada{\vb{A}} - \vb{v} \vb{u}^{T} \right) = \\[0.3em] \pause
&= \pderivada{\vb{A}} - \vb{v} \vb{u}^{T} - \dfrac{\vb{u} ( \vb{u}^{T} \pderivada{\vb{A}})}{H} + \dfrac{\vb{u} ( \vb{u}^{T} \vb{v}) \vb{u}^{T}}{H} = \\[0.35em] \pause
&= \pderivada{\vb{A}} - \vb{v} \vb{u}^{T} - \vb{u} \vb{v}^{T} + 2 \, g \vb{u} \vb{u}^{T}
\end{aligned}
\end{eqnarray*}
\pause
Donde:
\pause
\begin{align}
g = \dfrac{\vb{u}^{T} \vb{v}}{2 \, H}
\label{eq:ecuacion_09_44}
\end{align}
\end{frame}
\begin{frame}
\frametitle{Nuevos términos}
Haciendo que:
\pause
\begin{align}
\vb{w} =  \vb{v} - g \, \vb{u}
\label{eq:ecuacion_09_45}
\end{align}
\pause
se verifica fácilmente que la transformación se puede escribir como:
\pause
\begin{align}
\vb{Q} \, \pderivada{\vb{A}} \, \vb{Q} = \pderivada{\vb{A}} - \vb{W} \vb{u}^{T} - \vb{u} \vb{w}^{T}
\label{eq:ecuacion_09_46}
\end{align}
lo que nos da el siguiente procedimiento computacional que se llevará a cabo con $i = 1, 2, \ldots, n - 2$:
\end{frame}
\begin{frame}
\frametitle{Pseudocódigo para la transformación}
\begin{tcolorbox}[colback=melon, title=Procedimiento para la reducción]
Hacer con $i = 1, 2, \ldots, n - 2$: \\ \pause
\hspace{1cm} Sea $\pderivada{\vb{A}}$ la $(n - i) \times (n - i)$ parte inferior derecha de $\vb{A}$. \\ \pause
\hspace{1cm} Sea 
\[ \vb{x} = \big[ A_{i+1, i} \quad A_{i+2, i} \quad \ldots A_{n, i} \big]^{T} \]
la columna de longitud $n-1$ a la izquierda de $\pderivada{\vb{A}}$. \\ \pause
\hspace{1cm} Calcular $\abs{\vb{x}}$. Sea $k = \abs{\vb{x}}$ si $x_{1} > 0$ y $k = - \abs{\vb{x}}$ si $x_{1} < 0$.

Esta elección de signo minimiza el error por redondeo.
\end{tcolorbox}
\end{frame}
\begin{frame}
\frametitle{Pseudocódigo para la transformación}
\begin{tcolorbox}[colback=melon]
\hspace{1cm} Sea 
\[ \vb{u} = \big[ k + x_{1} \quad x_{2} \quad x_{3} \ldots x_{n-i} \big]^{T} \] \pause
\hspace{1cm} Calcular $H = \abs{x}/2$. \\ \pause
\hspace{1cm} Calcular $\vb{v} = \pderivada{\vb{A}} \, \vb{u} / H$. \\ \pause
\hspace{1cm} Calcular $g = \vb{u}^{T} \, \vb{v} / (2 \, H)$ 
\end{tcolorbox}
\end{frame}
\begin{frame}
\frametitle{Pseudocódigo para la transformación}
\begin{tcolorbox}[colback=melon]
\hspace{1cm} Calcular $\vb{w} = \vb{v} - g \, \vb{u}$ \\ \pause
\hspace{1cm} Calcular la transformación:
\[ \pderivada{\vb{A}} \leftarrow \pderivada{\vb{A}} - \vb{w}^{T} \, \vb{u} - \vb{u}^{T} \, \vb{w} \] \pause
\hspace{1cm} Se deja que:
\[ A_{i, i+1} = A_{i+1, i} = - k \]
\end{tcolorbox}
\end{frame}

\subsection{Transformación acumulada}

\begin{frame}
\frametitle{De los eigenvalores}
Debido a que usamos transformaciones de similitud, los eigenvalores de la matriz tridiagonal son los mismos que los de la matriz original.
\end{frame}
\begin{frame}
\frametitle{De los eigenvectores}
Sin embargo, para determinar los eigenvectores $\vb{X}$ de la matriz original $\vb{A}$ debemos usar la transformación:
\pause
\begin{align*}
\vb{X} = \vb{P} \, \vb{X}_{\text{tridiag}}
\end{align*}
\pause
donde $\vb{P}$ es la acumulación de las transformaciones individuales:
\pause
\begin{align*}
\vb{P} = \vb{P}_{1} \, \vb{P}_{2} \, \ldots , \vb{P}_{n-2}
\end{align*}
\end{frame}
\begin{frame}
\frametitle{La matriz acumulada}
Construimos la matriz de transformación acumulada inicializando $\vb{P}$ a una matriz identidad de $n \times n$ y luego aplicando la transformación:
\pause
\renewcommand{\arraystretch}{1}
\begin{align}
\vb{P} \leftarrow \vb{P} \, \vb{P}_{i} =
\begin{bmatrix}
P_{11} & P_{12} \\
P_{21} & P_{22}
\end{bmatrix}
\begin{bmatrix}
\vb{I}_{i} & \vb{0}^{T} \\
\vb{0} & \vb{Q}
\end{bmatrix} =
\begin{bmatrix}
P_{11} & P_{12} \vb{Q} \\
P_{21} & P_{22} \vb{Q}
\end{bmatrix}
\label{eq:ecuacion_b}
\end{align}
con $i = 1, 2, \ldots, n - 2$
\end{frame}
\begin{frame}
\frametitle{Multiplicando las columnas}
Se puede ver que cada multiplicación afecta solo a las $n - i$ columnas más a la derecha de $\vb{P}$ (dado que la primera fila de $\vb{P}_{12}$ contiene solo ceros, también se puede omitir en la multiplicación).
\end{frame}
\begin{frame}
\frametitle{Usando notación}
Usando la notación:
\pause
\renewcommand{\arraystretch}{1}
\begin{align*}
\pderivada{\vb{P}} =
\begin{bmatrix}
\vb{P}_{12} \\
\vb{P}_{22}
\end{bmatrix}
\end{align*}
\pause
Se tiene que:
\pause
\begin{eqnarray}
\begin{aligned}[b]
\begin{bmatrix}
\vb{P}_{12} \vb{Q}\\
\vb{P}_{22} \vb{Q}
\end{bmatrix}
&= \pderivada{\vb{P}} \vb{Q} = \pderivada{\vb{P}} \left( \vb{I} - \dfrac{\vb{u} \vb{u}^{T}}{H} \right) = \\[0.5em] \pause
&= \pderivada{\vb{P}} - \dfrac{\pderivada{\vb{P}} \vb{u}}{H} \, \vb{u}^{T} = \\[0.5em] \pause
&= \pderivada{\vb{P}} - \vb{y} \, \vb{u}^{T}
\end{aligned}
\label{eq:ecuacion_09_47}
\end{eqnarray}
\end{frame}
\begin{frame}
\frametitle{Usando notación}
Donde:
\pause
\begin{align}
\vb{y} = \dfrac{\pderivada{\vb{P}} \vb{u}}{H}
\label{eq:ecuacion_09_48}
\end{align}
\end{frame}
\begin{frame}
\frametitle{Pseudocódigo para multiplicar las matrices}
El procedimiento para realizar la multiplicación de matrices en la ec. (\ref{eq:ecuacion_b}) es como sigue:
\pause
\begin{tcolorbox}[colback=platinum, title=Procedimiento para la multiplicación de matrices]
Recupera $\vb{u}$ (las $\vb{u}$ se almacenan en columnas debajo de la diagonal principal de $\vb{A}$). \\ \pause
Calcula $H = \abs{\vb{u}} / 2$. \\ \pause
Calcula $\vb{y} = \pderivada{\vb{P}} \, \vb{u} / H$. \\ \pause
Calcula la tranformación: $\pderivada{\vb{P}} \leftarrow \pderivada{\vb{P}} - \vb{y} \, \vb{u}^{T}$
\end{tcolorbox}
\end{frame}

\subsection{Función \texttt{householder}}

\begin{frame}
\frametitle{Función en \python}
La función \funcionazul{householder} reduce una matriz a la forma tridiagonal.
\\
\bigskip
\pause
Devuelve (\funcionazul{d}, \funcionazul{c}), donde \funcionazul{d} y \funcionazul{c} son vectores que contienen los elementos de la diagonal principal y la subdiagonal, respectivamente.
\end{frame}
\begin{frame}
\frametitle{Cómo opera el código}
Solo la parte triangular superior se reduce a la forma triangular.
\\
\bigskip
\pause
La parte debajo de la diagonal principal se usa para almacenar los vectores $\vb{u}$.
\end{frame}
\begin{frame}
\frametitle{Cómo opera el código}
Esto se hace automáticamente mediante la instrucción \funcionazul{u = a[k+1:n,k]}, que no crea un nuevo objeto \funcionazul{u}, sino que simplemente establece una referencia a \funcionazul{a[k+1:n,k]} (hace una copia).
\\
\bigskip
\pause
Por lo tanto, cualquier cambio realizado en \funcionazul{u} se refleja en \funcionazul{a[k+1:n,k]}.
\end{frame}
\begin{frame}
\frametitle{Otra función necesaria}
La función \funcionazul{calculaP} devuelve la matriz de transformación acumulada $\vb{P}$.
\\
\bigskip
\pause
No es necesario llamarla si solo se van a calcular los eigenvalores.
\end{frame}
\begin{frame}[allowframebreaks, fragile]
\frametitle{Función para tridiagonalizar}
\begin{lstlisting}[caption=Función householder]
def householder(a):
    n = len(a)
    
    for k in range(n-2):
        u = a[k+1:n,k]
        uMag = np.sqrt(np.dot(u,u))
        
        if u[0] < 0.0: uMag = -uMag
        
        u[0] = u[0] + uMag
        h = np.dot(u,u)/2.0
        v = np.dot(a[k+1:n,k+1:n],u)/h
        g = np.dot(u,v)/(2.0*h)
        v = v - g * u
        a[k+1:n,k+1:n] = a[k+1:n,k+1:n] - np.outer(v,u)- np.outer(u,v)
        a[k,k+1] = -uMag
        
    return np.diagonal(a), np.diagonal(a,1)
\end{lstlisting}
\end{frame}
\begin{frame}[allowframebreaks, fragile]
\frametitle{Función para calcular la matriz acumulada}
\begin{lstlisting}[caption=Función calculaP]
def calculaP(a):
        
    n = len(a)
    p = np.identity(n) * 1.0
    
    for k in range(n-2):
        u = a[k+1:n,k]
        h = np.dot(u,u)/2.0
        v = np.dot(p[1:n,k+1:n],u)/h
        p[1:n,k+1:n] = p[1:n,k+1:n] - np.outer(v,u)
    
    return p
\end{lstlisting}
\end{frame}

\subsection{Ejercicios}

\begin{frame}
\frametitle{Ejercicio 1}
Transforma la matriz:
\renewcommand{\arraystretch}{1}
\begin{align*}
\vb{A} =
\begin{bmatrix}
7 & 2 & 3 & -1 \\
2 & 8 & 5 & 1 \\
3 & 5 & 12 & 9 \\
-1 & 1 & 9 & 7 \\
\end{bmatrix}
\end{align*}
a una forma tridiagonal.
\end{frame}
\begin{frame}
\frametitle{Incisos}
\setbeamercolor{item projected}{bg=pumpkin,fg=black}
\setbeamertemplate{enumerate items}{%
\usebeamercolor[bg]{item projected}%
\raisebox{1.5pt}{\colorbox{bg}{\color{fg}\footnotesize\insertenumlabel}}%
}
\begin{enumerate}[<+->]
\item \textbf{Ejercicio a cuenta:} Realiza las cuentas a mano.
\item Usa las funciones \funcionazul{householder} y \funcionazul{calculaP}.
\end{enumerate}
\end{frame}
\begin{frame}[allowframebreaks, fragile]
\frametitle{Inciso 2}
\begin{lstlisting}[caption=Código para resolver el ejercicio]
from moduloMatrices import householder, calculaP
import numpy as np

a = # Aqui va el arreglo

d, c = householder(a)

print('Diagonal principal d: \n {0:}'.format(d))
print('\nSubdiagonal c: \n {0:}'.format(c))
print('\nMatriz de Transformacion [P]: {0:}')
print(calculaP(a))
\end{lstlisting}
\end{frame}

\end{document}

