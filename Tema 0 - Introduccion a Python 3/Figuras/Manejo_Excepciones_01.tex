\documentclass[margin=3mm]{standalone}
\usepackage[utf8]{inputenc}
\usepackage[spanish]{babel}
\usepackage{multirow}
\usepackage{tikz}
\definecolor{darkspringgreen}{rgb}{0.09, 0.45, 0.27}
\begin{document}
\begin{tikzpicture}
    \node at (0, 0) { \begin{tabular}{l l}
        \texttt{try:} & \\
        \multicolumn{2}{l}{\hspace{0.6cm} \textcolor{darkspringgreen}{\# Código a ejecutar}} \\
         \multicolumn{2}{l}{\hspace{0.6cm} \textcolor{darkspringgreen}{\# Pero es posible que el error se presente en este bloque}} \\
         & \\
        \texttt{except <tipo de error>:} & \\
        \multicolumn{2}{l}{\hspace{0.6cm} \textcolor{darkspringgreen}{\# Instrucciones que manejarán la excepción}} \\
        \multicolumn{2}{l}{\hspace{0.6cm} \textcolor{darkspringgreen}{\# Este bloque se ejecuturá si en el bloque try hay un error}} \\
         & \\
        \texttt{else:} & \\
        \multicolumn{2}{l}{\hspace{0.6cm} \textcolor{darkspringgreen}{\# Instrucciones que se ejecutarán si el bloque try se ejecuta}} \\
        \multicolumn{2}{l}{\hspace{0.6cm} \textcolor{darkspringgreen}{\# sin errores}} \\
         & \\
        \texttt{finally:} & \\
        \multicolumn{2}{l}{\hspace{0.6cm} \textcolor{darkspringgreen}{\# Este bloque se ejecutará siempre}}
    \end{tabular}};
\end{tikzpicture}
\end{document}