\documentclass[12pt]{beamer}
\usepackage{../Estilos/BeamerFC}
\usepackage{../Estilos/ColoresLatex}
\usepackage{courier}
\usepackage{listingsutf8}
\usepackage{listings}
\usepackage{xcolor}
\usepackage{textcomp}
\usepackage{color}
\definecolor{deepblue}{rgb}{0,0,0.5}
\definecolor{brown}{rgb}{0.59, 0.29, 0.0}
\definecolor{OliveGreen}{rgb}{0,0.25,0}
% \usepackage{minted}

\DeclareCaptionFont{white}{\color{white}}
\DeclareCaptionFormat{listing}{\colorbox{gray}{\parbox{0.98\textwidth}{#1#2#3}}}
\captionsetup[lstlisting]{format=listing,labelfont=white,textfont=white}
\renewcommand{\lstlistingname}{Código}


\definecolor{Code}{rgb}{0,0,0}
\definecolor{Keywords}{rgb}{255,0,0}
\definecolor{Strings}{rgb}{255,0,255}
\definecolor{Comments}{rgb}{0,0,255}
\definecolor{Numbers}{rgb}{255,128,0}

\makeatletter

\newif\iffirstchar\firstchartrue
\newif\ifstartedbyadigit
\newif\ifprecededbyequalsign

\newcommand\processletter
{%
  \ifnum\lst@mode=\lst@Pmode%
    \iffirstchar%
        \global\startedbyadigitfalse%
      \fi
      \global\firstcharfalse%
    \fi
}

\newcommand\processdigit
{%
  \ifnum\lst@mode=\lst@Pmode%
      \iffirstchar%
        \global\startedbyadigittrue%
      \fi
      \global\firstcharfalse%
  \fi
}

\lst@AddToHook{OutputOther}%
{%
  \lst@IfLastOtherOneOf{=}
    {\global\precededbyequalsigntrue}
    {}%
}

\lst@AddToHook{Output}%
{%
  \ifprecededbyequalsign%
      \ifstartedbyadigit%
        \def\lst@thestyle{\color{orange}}%
      \fi
    \fi
  \global\firstchartrue%
  \global\startedbyadigitfalse%
  \global\precededbyequalsignfalse%
}

\lstset{ 
language=Python,                % choose the language of the code
basicstyle=\footnotesize\ttfamily,       % the size of the fonts that are used for the code
numbers=left,                   % where to put the line-numbers
numberstyle=\scriptsize,      % the size of the fonts that are used for the line-numbers
stepnumber=1,                   % the step between two line-numbers. If it is 1 each line will be numbered
numbersep=5pt,                  % how far the line-numbers are from the code
backgroundcolor=\color{white},  % choose the background color. You must add \usepackage{color}
showspaces=false,               % show spaces adding particular underscores
showstringspaces=false,         % underline spaces within strings
showtabs=false,                 % show tabs within strings adding particular underscores
frame=single,   		% adds a frame around the code
tabsize=2,  		% sets default tabsize to 2 spaces
captionpos=t,   		% sets the caption-position to bottom
breaklines=true,    	% sets automatic line breaking
breakatwhitespace=false,    % sets if automatic breaks should only happen at whitespace
escapeinside={| |},  % if you want to add a comment within your code
stringstyle =\color{OliveGreen},
otherkeywords={as, np.array, np.concatenate, np.linspace, linspace, interpolate.interp1d, kind, plt.plot, .copy, np.arange, np.cos, np.pi, lw, ls, label, splrep, splev, plt.legend, loc, plt.title, plt.ylim, plt.show, sign, math.ceil, math.log, np.sqrt, np.exp, np.zeros, plt.xlabel, plt.ylabel, plt.xlim, np.identity, random, np.dot, np.outer, np.diagonal },             % Add keywords here
keywordstyle = \color{blue},
commentstyle = \color{darkcerulean},
identifierstyle = \color{black},
literate=%
         {á}{{\'a}}1
         {é}{{\'e}}1
         {í}{{\'i}}1
         {ó}{{\'o}}1
         {ú}{{\'u}}1
%
%keywordstyle=\ttb\color{deepblue}
%fancyvrb = true,
}

\lstdefinestyle{FormattedNumber}{%
    literate={0}{{\textcolor{red}{0}}}{1}%
             {1}{{\textcolor{red}{1}}}{1}%
             {2}{{\textcolor{red}{2}}}{1}%
             {3}{{\textcolor{red}{3}}}{1}%
             {4}{{\textcolor{red}{4}}}{1}%
             {5}{{\textcolor{red}{5}}}{1}%
             {6}{{\textcolor{red}{6}}}{1}%
             {7}{{\textcolor{red}{7}}}{1}%
             {8}{{\textcolor{red}{8}}}{1}%
             {9}{{\textcolor{red}{9}}}{1}%
             {.0}{{\textcolor{red}{.0}}}{2}% Following is to ensure that only periods
             {.1}{{\textcolor{red}{.1}}}{2}% followed by a digit are changed.
             {.2}{{\textcolor{red}{.2}}}{2}%
             {.3}{{\textcolor{red}{.3}}}{2}%
             {.4}{{\textcolor{red}{.4}}}{2}%
             {.5}{{\textcolor{red}{.5}}}{2}%
             {.6}{{\textcolor{red}{.6}}}{2}%
             {.7}{{\textcolor{red}{.7}}}{2}%
             {.8}{{\textcolor{red}{.8}}}{2}%
             {.9}{{\textcolor{red}{.9}}}{2}%
             {\ }{{ }}{1}% handle the space
         ,%
          %mathescape=true
          escapeinside={__}
          }



\usetheme{Warsaw}
\usecolortheme{seahorse}
%\useoutertheme{default}
\setbeamercovered{invisible}
% or whatever (possibly just delete it)
\setbeamertemplate{section in toc}[sections numbered]
\setbeamertemplate{subsection in toc}[subsections numbered]
\setbeamertemplate{subsection in toc}{\leavevmode\leftskip=3.2em\rlap{\hskip-2em\inserttocsectionnumber.\inserttocsubsectionnumber}\inserttocsubsection\par}
\setbeamercolor{section in toc}{fg=blue}
\setbeamercolor{subsection in toc}{fg=blue}
\setbeamercolor{frametitle}{fg=blue}
\setbeamertemplate{caption}[numbered]

\setbeamertemplate{footline}
\beamertemplatenavigationsymbolsempty
\setbeamertemplate{headline}{}


\makeatletter
\setbeamercolor{section in foot}{bg=gray!30, fg=black!90!orange}
\setbeamercolor{subsection in foot}{bg=blue!30}
\setbeamercolor{date in foot}{bg=black}
\setbeamertemplate{footline}
{
  \leavevmode%
  \hbox{%
  \begin{beamercolorbox}[wd=.333333\paperwidth,ht=2.25ex,dp=1ex,center]{section in foot}%
    \usebeamerfont{section in foot} \insertsection
  \end{beamercolorbox}%
  \begin{beamercolorbox}[wd=.333333\paperwidth,ht=2.25ex,dp=1ex,center]{subsection in foot}%
    \usebeamerfont{subsection in foot}  \insertsubsection
  \end{beamercolorbox}%
  \begin{beamercolorbox}[wd=.333333\paperwidth,ht=2.25ex,dp=1ex,right]{date in head/foot}%
    \usebeamerfont{date in head/foot} \insertshortdate{} \hspace*{2em}
    \insertframenumber{} / \inserttotalframenumber \hspace*{2ex} 
  \end{beamercolorbox}}%
  \vskip0pt%
}
\makeatother

\makeatletter
\patchcmd{\beamer@sectionintoc}{\vskip1.5em}{\vskip0.8em}{}{}
\makeatother

%\newlength{\depthofsumsign}
%\setlength{\depthofsumsign}{\depthof{$\sum$}}
% \newcommand{\nsum}[1][1.4]{% only for \displaystyle
%     \mathop{%
%         \raisebox
%             {-#1\depthofsumsign+1\depthofsumsign}
%             {\scalebox
%                 {#1}
%                 {$\displaystyle\sum$}%
%             }
%     }
% }
\def\scaleint#1{\vcenter{\hbox{\scaleto[3ex]{\displaystyle\int}{#1}}}}
\def\scaleoint#1{\vcenter{\hbox{\scaleto[3ex]{\displaystyle\oint}{#1}}}}
\def\bs{\mkern-12mu}

\usefonttheme{serif}

\title{Tema 0 - Introducción a python 2}
\author{M. en C. Gustavo Contreras Mayén}
\date{16 de agosto de 2022}

\begin{document}

\maketitle

\section*{Contenido}
\frame[allowframebreaks]{\tableofcontents[currentsection, hideallsubsections]}

\section{Colecciones}
\frame{\tableofcontents[currentsection, hideothersubsections]}
\subsection{Definición}

\begin{frame}
\frametitle{¿Qué es una colección?}
Hemos revisado algunos tipos de datos básicos en \python: números, booleanos y cadenas de texto.
\\
\bigskip
\pause
Ahora presentaremos los tipos de colecciones que se manejan en este lenguaje:
\pause
\setbeamercolor{item projected}{bg=tangerine,fg=black}
\setbeamertemplate{enumerate items}{%
\usebeamercolor[bg]{item projected}%
\raisebox{1.5pt}{\colorbox{bg}{\color{fg}\footnotesize\insertenumlabel}}%
}
\begin{enumerate}[<+->]
\item Listas.
\item Tuplas.
\item Diccionarios.
\end{enumerate}
\end{frame}

\subsection{Listas}

\begin{frame}
\frametitle{Las listas en \python}
Las listas son el tipo de dato más versátil de los datos compuestos de \python.
\\
\bigskip
Una lista contiene elementos separados por comas y entre corchetes $[ \quad ]$.
\end{frame}
\begin{frame}
\frametitle{Las listas en \python}
En cierta medida, las listas son similares a los arreglos (arrays) o vectores.
\\
\bigskip
Un punto importante de las listas: \pause es que \textbf{todos los elementos pertenecientes a una lista pueden ser de tipo de datos diferente}.
\end{frame}
\begin{frame}[fragile]
\frametitle{Ejemplos con listas}
Para introducir una lista, abrimos corchetes y separamos mediante comas los elementos de la misma:
\pause
\begin{lstlisting}[caption=Definiendo dos listas]
lista1 = ['abcd', 786, 2.23, 'salmon', 70.2]

lista2 = [ 123, 'pizza']

print(lista1)
print(lista2)
\end{lstlisting}
\end{frame}

\begin{frame}
\frametitle{Las listas en \python}
Los valores almacenados en una lista se recuperan usando la misma técnica de \emph{slicing} que vimos con las cadenas: \pause $[ \: ]$ y $[:]$, \pause donde los índices van desde  $0$ hasta el   $n - 1$.
\end{frame}
\begin{frame}[fragile]
\frametitle{Manejando las listas}
\begin{lstlisting}[caption=Usando el slicing en una lista]
print(lista1[2:])
print()
print(lista1[-1])
print()
print(lista1[2:4])
\end{lstlisting}
\end{frame}
\begin{frame}[fragile]
\frametitle{Operaciones con las listas 2}
También se ocupan las operaciones de \emph{concatenación} ($+$) \pause y repetición en las listas ($*$):
\pause
\begin{lstlisting}[caption=Concatenación y repetición de listas]
print(lista1 + lista2)
print()
print(lista2*2)
\end{lstlisting}
\pause
Toma en cuenta que el resultado permanece en la memoria de la computadora hasta que se cierre el programa, no se está asignando a una variable.
\end{frame}
\begin{frame}[fragile]
\frametitle{Agregar elementos a una lista}
Las listas en \python{} son los únicos objetos en los que podemos agregar nuevos elementos (son \textbf{mutables}), para ello hay que utilizar la función \textoazul{append}:
\pause
\begin{lstlisting}[caption=Agregando un elemento a la lista]
lista1.append(654.321)
print(lista1)
\end{lstlisting}
\pause
El nuevo elemento ocupa el último lugar dentro de la lista.
\end{frame}
\begin{frame}[fragile]
\frametitle{Reemplazo de un elemento de la lista}
Con una lista en \python{} se puede reemplazar el contenido específico de un elemento, haciendo referencia al índice en particular:
\pause
\begin{lstlisting}[caption=Modificando el contenido de un elemento de la lista]
print(lista2)
lista2[1] = 'coordenada'
print()
print(lista2)
\end{lstlisting}
\end{frame}
\begin{frame}[fragile]
\frametitle{Característica de las listas}
Las listas son un tipo de objeto en \python{} que permite que dentro de la misma, pueda contener a su vez, otra lista:
\pause
\begin{lstlisting}[caption=Una lista dentro de una lista]
print(lista1)
lista1.append(lista2)
print()
print(lista1)
\end{lstlisting}
\end{frame}

\subsection{Tuplas}

\begin{frame}
\frametitle{Las tuplas en \python}
Una tupla es otro tipo de datos de secuencia que es similar a la lista.
\\
\bigskip
\pause
Una tupla consiste en un grupo de valores separados por comas, identificamos a una tupla por que ésta usa paréntesis $( \quad )$.
\end{frame}
\begin{frame}
\frametitle{Característica de las tuplas}
Las principales características de las tuplas son:
\pause
\setbeamercolor{item projected}{bg=airforceblue,fg=aliceblue}
\setbeamertemplate{enumerate items}{%
\usebeamercolor[bg]{item projected}%
\raisebox{1.5pt}{\colorbox{bg}{\color{fg}\footnotesize\insertenumlabel}}%
}
\begin{enumerate}[<+->]
\item Los elementos de las tuplas no pueden modificarse (son \textbf{inmutables}).
\item No es posible agregar nuevos elementos a un tupla.
\item No podemos modificar el contenido de los elementos de la tupla.
\item Identificamos lo que una tupla contiene mediante el manejo de índices: usando el \emph{slicing}.
\end{enumerate}
\pause
Las tuplas pueden ser consideradas como listas de sólo lectura.
\end{frame}
\begin{frame}[fragile]
\frametitle{Ejemplos con tuplas}
\begin{lstlisting}[caption=Definiendo tuplas]
tupla1 = ('abcd', 786, 2.23, 'arena', 70.2)
tupla2 = (3.14, 'playa')

print(tupla1)
print()
print(tupla2)
print()
print(type(tupla1))
\end{lstlisting}
\end{frame}
\begin{frame}[fragile]
\frametitle{Manejando las tuplas}
Podemos seleccionar los elementos de la tupla mediante el uso de índices de contenido:
\pause
\begin{lstlisting}[caption=Recuperando elementos de una tupla]
print(tupla1[0])
print()
print(tupla1[-1])
\end{lstlisting}
\end{frame}
\begin{frame}[fragile]
\frametitle{Manejando las tuplas}
Podemos seleccionar los elementos de la tupla mediante el uso de \emph{slicing}:
\pause
\begin{lstlisting}[caption=Recuperando elementos de una tupla]
print(tupla1[1:3])
print()
print(tupla1[2:5])
\end{lstlisting}
\pause
\textbf{¿Por qué no tenemos un error al indicar un índice que no corresponde a la tupla?}
\end{frame}
\begin{frame}[fragile]
\frametitle{Errores con el manejo de tuplas}
Si queremos modificar el contenido de una tupla, obtendremos un mensaje de error:
\pause
\begin{lstlisting}[caption=Intentando modificar una tupla]
tupla1[2] = 'hola'
\end{lstlisting}
\end{frame}
\begin{frame}[fragile]
\frametitle{Errores con el manejo de tuplas}    
\begingroup
\fontsize{10}{10}\selectfont
\begin{verbatim}
TypeError     Traceback (most recent call last)

<stdin> in <module>()
----> 1 tupla1[2] = 'hola'

TypeError: 'tuple' object does not support
 item assignment
\end{verbatim}
\endgroup
\end{frame}
\begin{frame}[fragile]
\frametitle{Errores con el manejo de tuplas}
Si queremos agregar un elemento a la tupla, obtendremos un mensaje de error:
\pause
\begin{lstlisting}[caption=Intentando agregar un nuevo elemento a la tupla]
tupla1.append(100.56)
\end{lstlisting}
\end{frame}
\begin{frame}[fragile]
\frametitle{Errores con el manejo de tuplas}    
\begingroup
\fontsize{10}{10}\selectfont
\begin{verbatim}
TypeError     Traceback (most recent call last)

<stdin> in <module>()
----> 1 tupla1.append(100.56)

AttributeError: 'tuple' object has no 
  attribute 'append'
\end{verbatim}
\endgroup
\end{frame}

\subsection{Diccionarios}

\begin{frame}
\frametitle{Diccionarios}
Los diccionarios de \python{} son de tipo tabla-hash.
\\
\bigskip
\pause
Funcionan como matrices asociativas y consisten en pares \emph{\textcolor{cadmiumred}{llave} - \textcolor{cadmiumgreen}{valor}}.
\end{frame}
\begin{frame}
\frametitle{Elementos del diccinario}
La \emph{\textcolor{cadmiumred}{llave}} del diccionario puede ser casi de cualquier tipo de dato, pero suelen ser comúnmente números o cadenas.
\\
\bigskip
\pause
El \textcolor{cadmiumgreen}{valor}, por otra parte, puede ser cualquier tipo de objeto arbitrario de \python.
\end{frame}
\begin{frame}[fragile]
\frametitle{Escribir un diccionario}
Para crear un diccionario se debe de iniciar con las llaves $\{ \quad \}$, \pause el siguiente valor corresponde a la llave seguida de dos puntos y a continuación, el valor:
\\
\bigskip
\pause
\begin{align*}
\mbox{mi\_dict } = \{ \mbox{'valor1'} : \mbox{'llave1'}, \mbox{'valor2'} : \mbox{'llave2'} , \ldots \} 
\end{align*}
\end{frame}
\begin{frame}[fragile]
\frametitle{Escribir un diccionario}
La siguiente instrucción se escribe en una sola línea:
\pause
\begin{lstlisting}[caption=Escribiendo un diccionario]
fisicos = {1 : "Eistein", 2 : "Bohr", 
3 : "Pauli", 4 : "Schrodinger", 
5 : "Hawking"}

print(fisicos)
\end{lstlisting}
\end{frame}
\begin{frame}[fragile]
\frametitle{Recuperando los elementos de un diccionario}
Hay un conjunto de funciones que nos permiten recuperar tanto las llaves como los valores de un diccionario:
\pause
\begin{lstlisting}[caption=Recuperando el contenido de un diccionario]
fisicos.keys()

fisicos.values()
\end{lstlisting}
\end{frame}
\begin{frame}[fragile]
\frametitle{Agregar un nuevo elemento al diccionario}
Es posible agregar un nuevo elemento al diccionario con la siguiente función:
\pause
\begin{lstlisting}[caption=Agregando un elemento al diccionario]
print(fisicos)

fisicos.update({6:'Dirac'})
print()
print(fisicos)
\end{lstlisting}
\pause
Recordemos que este cambio solo queda en memoria, ya que no se ha asignado a una variable.
\end{frame}

\section{Identificadores en python}
\frame{\tableofcontents[currentsection, hideothersubsections]}
\subsection{Reglas para los identificadores}

\begin{frame}
\frametitle{Reglas para los identificadores}
Los identificadores son nombres que hacen referencia a los objetos que componen un programa: \textbf{constantes}, \textbf{variables}, \textbf{funciones}, \textbf{módulos}, \textbf{clases} etc.
\end{frame}
\begin{frame}
\frametitle{Reglas para los identificadores}
Se recomienda seguir las reglas para construir identificadores:
\begin{itemize}[<+->]
\item[\ding{212}] El primer carácter debe ser una letra o el carácter de subrayado (guión bajo)
\item[\ding{212}] El primer carácter puede ir seguido de un número variable de dígitos numéricos, letras o carácteres de subrayado.
\end{itemize}
\end{frame}
\begin{frame}
\frametitle{Reglas para los identificadores}
\begin{itemize}[<+->]
\item[\ding{212}] No pueden utilizarse espacios en blanco, ni símbolos de puntuación.
\item[\ding{212}] En \python{} se distingue de las mayúsculas y minúsculas.
\end{itemize}
\end{frame}
\begin{frame}
\frametitle{Reglas para los identificadores}
Existe un estándar para la escritura del código en \python, revisa en la siguiente liga, el manejo de los nombres de los identificadores en: \href{shorturl.at/fOUV7}{\color{blue}{\underline{Referencia para nombres de objetos en \python{}}}}, del estándar PEP-8.
\end{frame}
\begin{frame}
\frametitle{Palabras reservadas}
No pueden utilizarse las palabras reservadas de \python{} para ningún tipo de identificador, \pause ya que son palabras reservadas para la ejecución de comandos, funciones, tareas, etc. propias de \python{} (de igual manera que son reservadas en otros lenguajes de programación o del mismo sistema operativo), entre las más comunes tenemos:
\end{frame}
\begin{frame}
\frametitle{Palabras reservadas}
\texttt{
\fontsize{12}{12}\selectfont
\begin{table}
\begin{tabular}{c c c c c }
del & for & is & raise & assert \\ \hline
elif & global & else & or & yield  \\ \hline
from & lamda & return & break & system \\ \hline
not & try & class & except & if \\ \hline
while & continue & exec & import & pass \\ \hline
def & finally & in & print & del \\ \hline
\end{tabular}
\end{table}
}
\end{frame}

\section{Estructuras de control}
\frame{\tableofcontents[currentsection, hideothersubsections]}
\subsection{¿Qué son las estructuras de control?}

\begin{frame}
\frametitle{Estructuras de control}
En cualquier lenguaje de programación se incluye una serie de estructuras de control para ampliar el control, la lógica y ejecución de un programa.
\\
\bigskip
En \python, manejaremos las más comunes, que son relativamente sencillas de usar, cuidado siempre la sintaxis respectiva.
\end{frame}
\begin{frame}
\frametitle{Las estructuras de control}
Revisaremos las siguientes estructuras:
\setbeamercolor{item projected}{bg=darkbrown,fg=cream}
\setbeamertemplate{enumerate items}{%
\usebeamercolor[bg]{item projected}%
\raisebox{1.5pt}{\colorbox{bg}{\color{fg}\footnotesize\insertenumlabel}}%
}
\begin{enumerate}[<+->]
\item Condicionales.
\item  Bucles (iterativas).
\end{enumerate}
\end{frame}

\section{Condicionales}

\begin{frame}
\frametitle{Los condicionales}
Una estructura condicional inicialmente evalúa si una o más condiciones cumplen con un valor \textoazul{\texttt{True}}.
\\
\bigskip
\pause
Si la condición (o condiciones) se cumplen, entonces pasa a un bloque de instrucciones que se van a ejecutar.
\end{frame}
\begin{frame}
\frametitle{Los condicionales}    
En caso de que el valor de la condición (o condiciones) no se cumpla, \pause es decir, tienen un valor \textcolor{red}{\texttt{False}}, \pause no se pasa al bloque con las instrucciones contenidas.
\\
\bigskip
\pause
Se ejecuta la siguiente línea de código fuera del condicional.
\end{frame}
\begin{frame}[fragile]
\frametitle{El condicional if}
El condicional \funcionazul{if} requiere de una expresión inicial que va a evaluar, como ya se mencionó, en caso de que no se cumpla el valor \textoazul{\texttt{True}}, no se ejecutan las instrucciones contenidas.
\begin{verbatim}
if expresion:
    instruccion1
    instruccion2

siguiente línea código
\end{verbatim}
\end{frame}
\begin{frame}[fragile]
\frametitle{Ejemplo de condicional if}
El siguiente ejemplo es un condicional \textoazul{if}:
\pause
\begin{lstlisting}[caption=La estructura condicional if]
distancia = 100
tiempo = 60

if distancia < 1000:
    print('La distancia es menor a un kilómetro')

print(distancia/tiempo)
\end{lstlisting}
\end{frame}
\begin{frame}
\frametitle{El bloque else:}
Cuando tenemos un condicional con \funcionazul{if} y la expresión que se evalúa tiene un valor \textcolor{red}{False}, sabemos que saldrá del bloque condicional.
\\
\bigskip
\pause
Pero si queremos que se ejecute un bloque de instrucciones a pesar de que la expresión evaluada sea \textcolor{red}{False}, \pause recurrimos a la instrucción \funcionazul{else:}, por lo que se ejecutan las instrucciones contenidas dentro de este bloque.
\end{frame}
\begin{frame}[fragile]
\frametitle{El bloque \texttt{else:}}
\begin{verbatim}
if expresion1:
    instruccion1
    instruccion2
else:
    instruccion-else-1
    instruccion-else-2

siguiente línea código
\end{verbatim}
\end{frame}
\begin{frame}[fragile]
\frametitle{Ejemplo de condicional if-else}
El siguiente ejemplo es un condicional \textoazul{if-else}:
\pause
\begin{lstlisting}[caption=La estructura condicional if-else]
distancia = 1500
tiempo = 60

if distancia > 1000:
    print('La distancia es menor a un kilómetro')
else:
    print('La distancia es mayor a un kilómetro')

print(distancia/tiempo)
\end{lstlisting}
\end{frame}
\begin{frame}[fragile]
\frametitle{El condicional \texttt{elif:}}
El condicional \funcionazul{if} evalúa solo una expresión (o expresiones), \pause en ocasiones se puede incluir la evaluación de una segunda expresión mediante:
\\
\bigskip
\pause
\begin{verbatim}
if expresion1:
    instruccion1
    instruccion2
elif expresion2:
    otra-instruccion1
    otra-instruccion2

siguiente línea código
\end{verbatim}
\end{frame}
\begin{frame}
\begin{frame}
\frametitle{El condicional \texttt{elif:}}
La \texttt{expresion1} no se cumple, por lo que se pasa a la siguiente línea.
\\
\bigskip
\pause
Si \texttt{expresion2} devuelve un valor \textoazul{\texttt{True}}, entonces se ejecutan las instrucciones contenidas en ese bloque.
\end{frame}
\begin{frame}[fragile]
\frametitle{Combinación en un bloque condicional}
Podemos ocupar en un bloque condicional la evaluación de dos expresiones, en caso de que ambas sean falsas, se ejecuta el código contenido en \funcionazul{else:}
\end{frame}
\begin{frame}[fragile]
\frametitle{Combinación en un bloque condicional}
\fontsize{13}{13}\selectfont
\begin{verbatim}
if expresion1:
    instruccion1
    instruccion2
elif expresion2:
    instruccion-elif-1
    instruccion-elif-2
else:
    instruccion-else-1
    instruccion-else-2

siguiente línea código
\end{verbatim}
\end{frame}
\begin{frame}[fragile]
\frametitle{Ejemplo de condicional}
\begin{lstlisting}
a = int(input('Introduce el valor de a'))
if a > 0:
    print ("a es positivo")
    a = a + 1
elif a == 0: 
    print ("a es 0")
else:
    print ("a es negativo")
\end{lstlisting}
\end{frame}

% \subsection{Bucles o Loops}

% \begin{frame}
% \frametitle{Bucles}
% Un bucle es una sentencia que repite un número determinado de veces, un conjunto de instrucciones.
% \\
% \bigskip
% Se evalúa inicialmente una condición, en caso de que se cumple (valor \funcionazul{True}) se ejecutan las instrucciones contenidas.
% \end{frame}
% \begin{frame}
% \frametitle{Bucles}
% Posteriormente, se revisa el valor de la condición, mientras sea verdadero, las instrucciones se ejecutan nuevamente, el bucle termina cuando que el valor de la condición sea un valor \textcolor{red}{\texttt{False}}.
% \end{frame}
% \begin{frame}
% \frametitle{¿Cuántas veces se va a repetir?}
% Hay que considerar que se puede conocer de antemano el número de veces que se va a repetir el ciclo.
% \\
% \bigskip
% \pause
% Cuando no se conoce el número de veces que se va a repetir, hay que ser cuidadosos y evitar los bucles infinitos.
% \end{frame}
% \subsection{Sentancia \texttt{for ... in}}
% \begin{frame}\frametitle{Sentencia for ... in}
% Es una forma genérica de iterar sobre una secuencia.
% \\
% \bigskip
% Podemos usar como secuencia: tanto listas como tuplas o generar una para ejecutar el bucle un número determinado de veces.
% \end{frame}
% \begin{frame}[fragile]
% \frametitle{Ejemplo}
% Usaremos un objeto lista y el ciclo \funcionazul{for}
% \pause
% \begin{lstlisting}
% secuencia = ["uno", "dos", "tres"]
% for elemento in secuencia:
%     print (elemento)
% \end{lstlisting}
% \pause
% En este ejemplo la instrucción \funcionazul{print} se ejecutará tantas veces como elementos haya en la lista, y en cada iteración la variable elemento tomará el valor de cada uno de los elementos de la lista secuencia.
% \end{frame}
% \begin{frame}
% \frametitle{Iteración sobre secuencia de números}
% ¿Cómo le hacemos para iterar sobre una serie de números naturales consecutivos?
% \\
% \bigskip
% Por ejemplo de 1 a 20.
% \\
% \bigskip
% \pause
%  Para ello usaremos la función \funcionazul{range(\ )}. Esta función genera una lista con una progresión aritmética de números naturales.
%  \end{frame}
% \begin{frame}
% \frametitle{Argumentos de la función \texttt{range}}
% \setbeamercolor{item projected}{bg=green!70!black,fg=black}
% \setbeamertemplate{enumerate items}[circle]
% \begin{enumerate}[<+->]
% \item Si le pasamos un único parámetro se generará una lista que va desde $0$ hasta $n-1$. 
% \item Si le damos dos argumentos: \funcionazul{(inicio,fin)}, generará una lista desde \textoazul{inicio} hasta \textoazul{fin - 1}.
% \item Si le damos tres argumentos: \funcionazul{(inicio,fin, paso)}, usará el \textoazul{paso} como incremento para generar los elementos de la lista.
% \end{enumerate}  
% \end{frame}
% \begin{frame}[fragile]
% \frametitle{Ejemplos de for ... in}
% Escribimos las instrucciones:
% \begin{lstlisting}
% print(list(range(10)))
% print(list(range(5, 10)))
% print(list(range(0, 10, 3)))
% \end{lstlisting}
% \pause
% La salida será:
% \begin{lstlisting}
% [0, 1, 2, 3, 4, 5, 6, 7, 8, 9]
% [5, 6, 7, 8, 9]
% [0, 3, 6, 9]
% \end{lstlisting}
% \end{frame}
% \begin{frame}[fragile]
% \frametitle{Ejemplos de for ... in}
% En el siguiente ejemplo el programa imprimirá los números de $0$ a $5$ cada uno en una línea.
% \pause
% \begin{lstlisting}
% for i in range(6):
%     print(i)
% \end{lstlisting}
% \end{frame}
% \begin{frame}[fragile]
% \frametitle{Ejemplos de for ... in}
% Podemos usar una cadena de caracteres como secuencia, de forma que cada iteración del bucle tomaremos una letra de la cadena.
% \pause
% \begin{lstlisting}
% for char in 'ABCD':
%     print(char)
% \end{lstlisting}
% \end{frame}
% \begin{frame}[fragile]
% \frametitle{Ejemplos de for ... in}
% En caso de que necesitemos iterar sobre una lista y a la vez tener el índice de cada posición de la lista usaremos la función \funcionazul{enumerate( )} que devuelve dos valores: la posición y el contenido de la lista.
% \pause
% \begin{lstlisting}
% secuencia = ["manzanas", "peras", "platanos"]
% for posicion, elemento in enumerate(secuencia):
%     print (posicion, elemento)
% \end{lstlisting}
% \end{frame}
% \subsection{El bucle while}
% \begin{frame}
% \frametitle{El bucle \texttt{while}}
% Ese bucle repite un conjunto de instrucciones mientras se cumpla una determinada condición que se evalúa al principio de cada ejecución.
% \end{frame}
% \begin{frame}
% \frametitle{El bucle \texttt{while}}
% Es evidente que las instrucciones del interior del bucle tendrán que hacer algo que pueda cambiar esa condición hasta que en algún momento deje de cumplirse.
% \\
% \bigskip
% En caso contrario tendremos un bucle infinito y el programa no terminará nunca su ejecución.
% \end{frame}
% \begin{frame}
% \frametitle{El bucle \texttt{while}}
% Una de las características del bucle \funcionazul{while} es que no está fijado previamente el número de veces que se ejecutan las instrucciones del bucle.
% \\
% \bigskip
% Se ejecutarán todas las que sean necesarias mientras se cumpla la condición.
% \end{frame}

% \begin{frame}
% \frametitle{Forzar la salida del bucle \texttt{while}}
% Como hemos mencionado, el ciclo \funcionazul{while} va a iterar mientras se cumpla una condición.
% \\
% \bigskip
% Pero vamos a encontrar que en ocasiones, necesitamos \enquote{salir} del bucle sin que tengamos que esperar a que la condición cambie.
% \end{frame}
% \begin{frame}[fragile]
% \frametitle{Ejemplos while}
% Hay dos palabras reservadas que se usan dentro de un bucle, se trata de \funcionazul{break} y \funcionazul{continue}.
% \\
% \bigskip
% \funcionazul{continue} hace que pasemos de nuevo al principio del bucle aunque no se haya terminado de ejecutar el ciclo anterior.
% \end{frame}
% \begin{frame}[fragile]
% \frametitle{Ejemplo con \texttt{while}}
% \begin{lstlisting}
% edad = 0
% while edad < 18:
%     edad = edad + 1
%     if edad % 2 == 0:
%         continue
%     print ("Felicidades, tienes " + str(edad))
% \end{lstlisting}
% \end{frame}
% \begin{frame}[fragile]
% \frametitle{Ejemplos while}
% Por su parte \funcionazul{break} hace que salgamos del bucle \funcionazul{while} directamente sin necesidad de volver a evaluar la condición y aunque siga siendo cierta.
% \end{frame}
% \begin{frame}[fragile]
% \frametitle{Ejemplos while}
% \begin{lstlisting}
% while True:
%     entrada = input("> ")
%     if entrada == "adios":
%         break
%     else:
%         print (entrada)
% \end{lstlisting}
% \end{frame}
% \section{Manejo de excepciones}
% \frame{\tableofcontents[currentsection, hideothersubsections]}
% \subsection{Control de errores}
% \begin{frame}[fragile]
% \frametitle{Manejo de excepciones}
% Cuando comenzamos a programar, nos podemos encontrar con mensajes de error al momento de ejecutar el programa, siendo las causas más comunes:
% \end{frame}
% \begin{frame}
% \frametitle{Manejo de excepciones}
% \begin{itemize}[<+->]
% \item Errores de dedo, escribiendo incorrectamente una instrucción, sentencia, variable o constante.
% \item Errores al momento de introducir los datos, por ejemplo, si el valor que se debe de ingresar es $123.45$, y si nosotros tecleamos $1234.5$, el resultado ya se considera un error.
% \end{itemize}
% \end{frame}
% \begin{frame}
% \frametitle{Manejo de excepciones}
% \begin{itemize}[<+->]
% \item Errores que se muestran en tiempo de ejecución, es decir, todo está bien escrito y los datos están bien introducidos, pero hay un error debido a la lógica del programa o del método utilizado, ejemplo: división entre cero.
% \end{itemize}
% \end{frame}
% \begin{frame}[fragile]
% \frametitle{Manejo de errores}
% En el siguiente ejemplo, obtendremos de antemano un error por intentar una operación matemática no permitida.
% \\
% \bigskip
% \verb|c = 12.0/0.0| \\
% \pause
% %\begin{exampleblock}{}
% \verb|Traceback (most recent call last):| \\
% \verb|File ''<pyshell#0>'', line 1, in ?| \\
% \verb|c = 12.0/0.0| \\
% \verb|ZeroDivisionError: float division|
% %\end{exampleblock}
% \end{frame}
% \begin{frame}[fragile]
% \frametitle{Manejo de excepciones}
% Veamos el siguiente ejemplo
% \begin{lstlisting}
% while True print('Hola mundo')

%   File "<stdin>", line 1
%     while True print('Hola mundo')
%                    ^
% SyntaxError: invalid syntax
% \end{lstlisting}
% \end{frame}
% \begin{frame}
% \frametitle{Información del error}
% \fontsize{13}{13}\selectfont
% El intérprete repite la línea culpable y muestra una pequeña \enquote{flecha} que apunta al primer lugar donde se detectó el error.
% \\
% \bigskip
% Este es causado por (o al menos detectado en) el símbolo que precede a la flecha: en el ejemplo, el error se detecta en la función \azulfuerte{print()}, ya que faltan dos puntos \azulfuerte{(:)} antes del mismo.
% \\
% \bigskip
% Se muestran el nombre del archivo y el número de línea para que sepas dónde mirar en caso de que la entrada venga de un programa.    
% \end{frame}
% \begin{frame}
% \frametitle{Tipos de error en \python}
% Es importante conocer los distintos tipos de error que pueden generarse en \python, en la documentación oficial, podremos encontrar una lista con el nombre del tipo de error y por qué se genera.
% \end{frame}
% \begin{frame}
% \frametitle{Errores aritméticos}
% \setbeamercolor{item projected}{bg=green!70!black,fg=white}
% \setbeamertemplate{enumerate items}[circle]
% \begin{enumerate}[<+->]
% \item \funcionazul{OverflowError}
% \item \funcionazul{ZeroDivisionError}
% \item \funcionazul{FloatingPointError}
% \end{enumerate}
% \end{frame}
% \begin{frame}
% \frametitle{Errores generales}
% \setbeamercolor{item projected}{bg=green!70!black,fg=white}
% \setbeamertemplate{enumerate items}[circle]
% \begin{enumerate}[<+->]
% \item \funcionazul{ImportError}
% \item \funcionazul{IndexError}
% \item \funcionazul{KeyboardInterrupt}
% \item \funcionazul{NameError}
% \item \funcionazul{SyntaxError}
% \item \funcionazul{TabError}
% \item \funcionazul{ValueError}
% \end{enumerate}
% \end{frame}
% \begin{frame}[fragile]
% \frametitle{Manejando excepciones}
% Es posible escribir programas que manejen determinadas excepciones. 
% \\
% \bigskip
% En el siguiente ejemplo, se le pide al usuario una entrada hasta que ingrese un entero válido, pero permite al usuario interrumpir el programa (usando \keys{\ctrl} + \keys{C}) o lo que sea que el sistema operativo soporte)
% \end{frame}
% \begin{frame}[fragile]
% \frametitle{Manejando excepciones}
% \begin{lstlisting}
% while True:
%     try:
%         x = int(input("Por favor ingrese un numero: "))
%         break
%     except ValueError:
%         print("Oops! No era valido. Intente nuevamente...")
% \end{lstlisting}
% \end{frame}
% \begin{frame}
% \frametitle{La declaración \texttt{try}}
% La declaración try funciona de la siguiente manera:
% \begin{itemize}[<+->]
% \item Primero, se ejecuta el bloque \azulfuerte{try} (el código entre las declaración \azulfuerte{try} y \azulfuerte{except}).
% \item  Si no ocurre ninguna excepción, el bloque \azulfuerte{except} se salta y termina la ejecución de la declaración \azulfuerte{try}.
% \end{itemize}
% \end{frame}
% \begin{frame}
% \frametitle{La declaración \texttt{try}}
% \begin{itemize}[<+->]
% \item  Si ocurre una excepción durante la ejecución del bloque \azulfuerte{try}, el resto del bloque se salta. Luego, si su tipo coincide con la excepción nombrada luego de la palabra reservada \azulfuerte{except}, se ejecuta el bloque \azulfuerte{except}, y la ejecución continúa luego de la declaración \azulfuerte{try}.
% \end{itemize}
% \end{frame}
% \begin{frame}
% \frametitle{La declaración \texttt{try}}
% \begin{itemize}[<+->]
% \item Si ocurre una excepción que no coincide con la excepción nombrada en el \azulfuerte{except}, esta se pasa a declaraciones \azulfuerte{try} de más afuera; si no se encuentra nada que la maneje, es una \emph{excepción no manejada}, y la ejecución se frena con un mensaje como los mostrados arriba.
% \end{itemize}
% \end{frame}
% \begin{frame}[fragile]
% \frametitle{Manejo de excepciones más elaborado}
% El siguiente código considera un manejo de excepciones por su tipo:
% \fontsize{12}{11}\selectfont
% \begin{verbatim}
% try:
%     <código suceptible de errores>
% except <ErrorTipo1>:
%     <bloque inscrito a ErrorTipo1>
% except< ErrorTipo2>:
%     <bloque inscrito a ErrorTipo2>
% except (<ErrorTipo3>, <ErrorTipo4>):
%     <bloque inscrito a ErrorTipo3 y ErrorTipo4>
% except:
%     <bloque inscrito a except general>
% \end{verbatim}
% \end{frame}
% \begin{frame}[fragile]
% \frametitle{Consideraciones sobre la gestión de excepciones}
% Con la inclusión de las excepciones hemos visto que el programa no se interrumpe cuando existe un error.
% \\
% \bigskip
% Es posible mostrar más información al usuario sobre el tipo de error, recordemos que habrá alguien que ocupe nuestro programa y al contar con más información, podrá resolver la situación.
% \end{frame}
% \begin{frame}[allowframebreaks, fragile]
% \frametitle{Ejemplo}
% Veamos el siguiente ejemplo (las dos diapositivas contienen el código:
% \fontsize{11}{10}\selectfont
% \begin{lstlisting}
% ocurre_error = False

% try:
%     numero = float(input('Introduce un numero: '))
%     print("La raiz cuadrada de numero %f es %f" % (numero, numero ** 0.5))

% except TypeError as descripcion:
%     ocurre_error = True
%     print("Ocurrio un error previsto:", descripcion)

% except:
%     ocurre_error = True
%     print("!No se que paso!")

% if ocurre_error:
%     print("Lastima.")
% else:
%     print("Buen dia.")
% \end{lstlisting}
% \end{frame}
% \begin{frame}[fragile]
% \frametitle{Ejecutando el código}
% Ahora introducimos algunos valores para revisar la operación del código:
% \\
% \bigskip
% \pause
% \verb|Introduce un numero: 12|
% \\
% \pause
% \begin{lstlisting}
% La raiz cuadrada del numero 12.000000 es 3.464102
% Buen dia.
% \end{lstlisting}
% \end{frame}
% \begin{frame}[fragile]
% \frametitle{Ejecutando el código}
% Ahora introducimos un valor que provocará un error:
% \\
% \bigskip
% \pause
% \verb|Introduce un numero: -6|
% \\
% \pause
% \begin{lstlisting}
% Ocurrio un error previsto: can't convert complex to float
% Lastima.
% \end{lstlisting}
% \end{frame}
% \begin{frame}[fragile]
% \frametitle{Ejecutando el código}
% Ejecutamos nuevamente el código e introducimos un valor que provocará un error:
% \\
% \bigskip
% \pause
% \verb|Introduce un numero: q|
% \\
% \pause
% \begin{verbatim}
% No se que paso
% Lastima.
% \end{verbatim}
% \end{frame}
% \begin{frame}
% \frametitle{Explicación del resultado}
% Cuando se introdujo un número negativo $(-1)$, sabíamos que se presentaría un \textcolor{red}{TypeError} ya que no se puede evaluar la raíz cuadrada de un número negativo.
% \\
% \bigskip
% \pause
% Este tipo de error coincide con el señalado en la sentencia \funcionazul{except}, por lo que se muestra la descripción del error a modo de mensaje.
% \end{frame}
% \begin{frame}
% \frametitle{Explicación del resultado}
% Mientras que al introducir un caracter (q), el tipo de error no corresponde con \textcolor{red}{TypeError}, por lo que no \enquote{entra} en la primera sentencia \funcionazul{except} y pasa a la siguiente sentencia \funcionazul{except} de tipo general.
% \end{frame}


\end{document}