\documentclass[hidelinks,12pt]{article}
\usepackage[left=0.25cm,top=1cm,right=0.25cm,bottom=1cm]{geometry}
%\usepackage[landscape]{geometry}
\textwidth = 20cm
\hoffset = -1cm
\usepackage[utf8]{inputenc}
\usepackage[spanish,es-tabla, es-lcroman]{babel}
\usepackage[autostyle,spanish=mexican]{csquotes}
\usepackage[tbtags]{amsmath}
\usepackage{nccmath}
\usepackage{amsthm}
\usepackage{amssymb}
\usepackage{mathrsfs}
\usepackage{graphicx}
\usepackage{subfig}
\usepackage{caption}
%\usepackage{subcaption}
\usepackage{standalone}
\usepackage[outdir=./Imagenes/]{epstopdf}
\usepackage{siunitx}
\usepackage{physics}
\usepackage{color}
\usepackage{float}
\usepackage{hyperref}
\usepackage{multicol}
\usepackage{multirow}
%\usepackage{milista}
\usepackage{anyfontsize}
\usepackage{anysize}
%\usepackage{enumerate}
\usepackage[shortlabels]{enumitem}
\usepackage{capt-of}
\usepackage{bm}
\usepackage{mdframed}
\usepackage{relsize}
\usepackage{placeins}
\usepackage{empheq}
\usepackage{cancel}
\usepackage{pdfpages}
\usepackage{wrapfig}
\usepackage[flushleft]{threeparttable}
\usepackage{makecell}
\usepackage{fancyhdr}
\usepackage{tikz}
\usetikzlibrary{patterns, arrows}
\usetikzlibrary{decorations.text}
\usetikzlibrary{decorations.markings}
\usetikzlibrary{matrix}
\usepackage{bigints}
\usepackage{menukeys}
\usepackage{tcolorbox}
\tcbuselibrary{breakable}
\usepackage{scalerel}
\usepackage{pgfplots}
\usepackage{pdflscape}
\pgfplotsset{compat=1.16}
\spanishdecimal{.}
\renewcommand{\baselinestretch}{1.5} 
\renewcommand\labelenumii{\theenumi.{\arabic{enumii}})}

\newcommand{\python}{\texttt{python}}
\newcommand{\textoazul}[1]{\textcolor{blue}{#1}}
\newcommand{\azulfuerte}[1]{\textcolor{blue}{\textbf{#1}}}
\newcommand{\funcionazul}[1]{\textcolor{blue}{\textbf{\texttt{#1}}}}

\newcommand{\pderivada}[1]{\ensuremath{{#1}^{\prime}}}
\newcommand{\sderivada}[1]{\ensuremath{{#1}^{\prime \prime}}}
\newcommand{\tderivada}[1]{\ensuremath{{#1}^{\prime \prime \prime}}}
\newcommand{\nderivada}[2]{\ensuremath{{#1}^{(#2)}}}


\newtheorem{defi}{{\it Definición}}[section]
\newtheorem{teo}{{\it Teorema}}[section]
\newtheorem{ejemplo}{{\it Ejemplo}}[section]
\newtheorem{propiedad}{{\it Propiedad}}[section]
\newtheorem{lema}{{\it Lema}}[section]
\newtheorem{cor}{Corolario}
\newtheorem{ejer}{Ejercicio}[section]

\newlist{milista}{enumerate}{2}
\setlist[milista,1]{label=\arabic*)}
\setlist[milista,2]{label=\arabic{milistai}.\arabic*)}
\newlength{\depthofsumsign}
\setlength{\depthofsumsign}{\depthof{$\sum$}}
\newcommand{\nsum}[1][1.4]{% only for \displaystyle
    \mathop{%
        \raisebox
            {-#1\depthofsumsign+1\depthofsumsign}
            {\scalebox
                {#1}
                {$\displaystyle\sum$}%
            }
    }
}
\def\scaleint#1{\vcenter{\hbox{\scaleto[3ex]{\displaystyle\int}{#1}}}}
\def\scaleoint#1{\vcenter{\hbox{\scaleto[3ex]{\displaystyle\oint}{#1}}}}
\def\scaleiiint#1{\vcenter{\hbox{\scaleto[3ex]{\displaystyle\iiint}{#1}}}}
\def\bs{\mkern-12mu}

\newcommand{\Cancel}[2][black]{{\color{#1}\cancel{\color{black}#2}}}

\title{Ejericicios del Examen Tema 3 \\ {\large Curso Física Computacional}}
\author{M. en C. Gustavo Contreras Mayén. \texttt{gux7avo@ciencias.unam.mx}}


\title{Ejericicios del Examen Tema 3 \\ {\large Curso Física Computacional}}
\author{M. en C. Gustavo Contreras Mayén. \texttt{gux7avo@ciencias.unam.mx}}

\date{ }
\begin{document}

\maketitle
\fontsize{14}{14}\selectfont

A continuación se presenta la lista de ejercicios a resolver que conforman la parte de la evaluación del Tema 3.
\par
Cada ejercicio tiene una calificación de un punto siempre y cuando esté debidamente resuelto.
\par
Revisa cuidadosamente cada ejercicio, ya que deberás de mencionar qué técnica de solución has elegido y justificar su uso, en caso de que el problema indique expresamente la técnica a utilizar, no será necesario que lo expliques.

\begin{enumerate}
% Ref. Kiusaalas (2005) 7.3 RK Methods
% Problema 11
\item Un péndulo está suspendido en un collar deslizante. El sistema está en reposo, posteriormente se aplica al collar un movimiento oscilante: $y (t) = Y \, \sin \omega t$, en $t = 0$. La ecuación diferencial que describe el movimiento del péndulo es:
\begin{align*}
\ddot{\theta} = - \dfrac{g}{L} \sin \theta + \dfrac{\omega^{2}}{L} Y \, \cos \theta \, \sin \omega t
\end{align*}
\begin{figure}[H]
\centering
\begin{tikzpicture}[font=\small]
	\draw (-0.1,-0.2) [pattern = north east lines] rectangle (0.1,0.4);
	\draw (4.9,-0.2) [pattern = north east lines] rectangle (5.1,0.4);
	\draw (0.1,0.0) rectangle (4.9,0.2);
	\draw (2.1,-0.1) [fill=white] rectangle (2.9,0.3);
	\draw [dashed] (2.5,-1.5) -- (2.5,0.7);
	\draw [-stealth, thick] (2.5,0.6) -- node [right=0.6cm] {$y (t)$} (3.5,0.6);
	\draw [thick] (2.5,0.1) -- node [near end, above=0.5cm]{$L$}(3.51,-2.);
	\draw (3.6,-2.2) circle (0.2) node [right=0.2cm] {$m$};
	\draw (2.5,-0.45) arc (270:290:7mm) node [below=0.15cm]{$\theta$};
\end{tikzpicture}
\end{figure}
Grafica $\theta$ contra $t$ en el intervalo de $t = 0$ a $t = 10$ segundos, así mismo, determina el desplazamiento mayor de $\theta$ durante éste período. Usa $g = \SI{9.80665}{\meter\per\square\second}$, $L = \SI{1.0}{\meter}$, $Y = \SI{0.25}{\meter}$ y $\omega = \SI{2.5}{\radian\per\second}$.

% Ref. Kiusaalas (2013) 7.3 RK Methods
% Problema 12
% \item Tenemos un sistema que consiste en un masa que se desliza sobre una barra guía que está en reposo, la masa se ubica en $r = \SI{0.75}{\meter}$. Al tiempo $t = 0$ se enciende un motor que proporiona un movimiento dado por la expresión $\theta (t) = (\pi/12) \cos \pi t$ sobre la barra. La EDO que describe el movimiento resultante de la masa deslizante es:
% \begin{align*}
% \ddot{r} = \left( \dfrac{\pi^{2}}{12}\right)^{2}  r \sin^{2} \pi t - g \sin \left( \dfrac{\pi}{12} \cos \pi t \right)
% \end{align*}
% \begin{figure}[H]
% \centering
% \begin{tikzpicture}[font=\small]
% 	\draw (0,0) [pattern = north east lines] rectangle (2,0.2);
% 	\draw (0.2,0.2) -- (0.55,1.2);
% 	\draw (1.8,0.2) -- (1.45,1.2);
% 	\draw (0.9,0.9) [rotate around={15:(1,1)}]rectangle (6,1.15);
% 	\draw (1,1) [fill=white] circle (0.5);
% 	\draw (1,1) circle (0.1);
% 	\draw (4,0.7) [fill=white, rotate around={15:(1,1)}]rectangle (4.5,1.4);
% 	\draw (2,1) -- (4,1);
% 	\draw (3.4,1) arc (0:25:12mm);
% 	\draw (3.8,1.3) node {$\theta(t)$};
% 	\draw (0.8,1.6) [rotate around={15:(0.8,1.6)}] -- (0.8,3);
% 	\draw (4,2.3) [rotate around={15:(4,2.3)}] -- (4,3);
% 	\draw (5.75,2.5) [rotate around={15:(5.75,2.5)}] -- (5.75,4.3);
% 	\draw [stealth-stealth, rotate around={15:(0.7,2)}] (0.7,2)  -- node [rotate=15,midway, fill=white]{$r$}(4,2);
% 	\draw [stealth-stealth, rotate around={15:(0.5,2.8)}] (0.5,2.8)  -- node [rotate=15,midway, fill=white]{$2 \mbox{ m}$}(5.5,2.8);
% 	\end{tikzpicture}
% \end{figure}
% Calcula el tiempo para el cual, la masa deslizante alcanza el extremo final de la barra guía (la punta de la barra). Usa el valor de $g = \SI{9.80665}{\meter\per\square\second}$.

% Problema 13
\item Una bala de masa $m = \SI{0.25}{\kilo\gram}$ se lanza con una velocidad $v_{0} = \SI{50}{\meter\per\second}$ en la dirección que se indica en la siguiente figura.
\begin{figure}[H]
\centering
\begin{tikzpicture}[font=\small]
	\draw (0,0) [pattern = north east lines, draw = none] rectangle (6,0.5);
	\draw (0,0.5) -- node [midway, above] {$R$} (6.3,0.5);
	\draw (1,0.5) -- node [near end, left] {$y$} (1,3);
	\draw [fill=white] (1,0.5) circle (0.1);
	\draw (0.6,0.8) node {$m$};
	\draw [-stealth, rotate around={30:(1,0.5)}, thick] (1,0.5) -- node [near end, above] {$v_{0}$} (3,0.5);
	\draw (2,0.5) arc (0:30:10mm);
	\draw (2.5,0.8) node {$30^{\circ}$};
	\draw (1,0.5) .. controls (3.8,2.2) and (4.6,2.4) .. (5.5,0.5);
%		  (4,2) .. controls (3.9,1.9).. and (4.1,2.1)..(5.5,0.5);
%		  (3.3,1.4) ..controls (4,1.5) and (4.2,1.3) .. (5.5,0.2);%.. controls (3.8,0.3) and (3.5,0.2) .. (5.5,0);
\end{tikzpicture}
\end{figure}
Si la fuerza aerodinámica de arrastre sobre la bala es $F_{D}= C_{D} \, v^{3/2}$, las ecuaciones diferenciales que describen el movimiento son:
\begin{align*}
\ddot{x} = - \dfrac{C_{D}}{m} \, \dot{x} \, v^{1/2} \hspace{2cm} \ddot{y}= - \dfrac{C_{D}}{m} \, \dot{y} \, v^{1/2} - g
\end{align*}
donde $v = \sqrt{\dot{x}^{2} + \dot{y}^{2}}$, $C_{D} = 0.03 \mbox{ kg/(ms)}^{1/2}$ y $g = \SI{9.80665}{\meter\per\square\second}$.

Calcula el tiempo de vuelo y el alcance $R$.

% Problema 19
% \item La solución al problema:
% \begin{align*}
% \sderivada{y} + \dfrac{1}{x} \, \pderivada{y} + y \hspace{1.5cm} y (0) = 1 \hspace{1cm} \pderivada{y}(0) = 0
% \end{align*}
% es la función de Bessel $J_{0} (x)$. Integra numéricamente para calcular $J_{0} (5)$ y compara el resultado con $-0.17760$, que es el valor que se obtiene de tablas matemáticas. Tip: para evitar la singularidad en $x = 0$, inicia la integración en $x = \num{d-12}$.

% Problema 19
\item Considera el siguiente problema de valores iniciales:
\begin{align*}
\sderivada{y} = 16.81 \, y \hspace{1.5cm} y (0) = 1.0 \hspace{1cm} \pderivada{y} (0) = -4.1
\end{align*}
\begin{enumerate}
\item Obtén la solución analítica.
\item ¿Anticipas dificultades en la solución numérica de este problema?
\item Realiza la integración numérica de $x = 0$ a $8$ para ver si tus preocupaciones estaban justificadas.
\end{enumerate}

% Ref. Kiusaalas 7.5 Adaptive RK method
% Problema 7
\item Calcula la solución numérica de la ED:
\begin{align*}
\sderivada{y} = 16.81 \, y
\end{align*}
de $x = 0$ a $2$ con el método RK Adaptativo y grafica los resultados. Usa las condiciones iniciales:
\begin{enumerate}
\item $y (0) = 1.0$, $\pderivada{y} (0) = -4.1$
\item $y (0) = 1.0$, $\pderivada{y} (0) = -4.11$
\end{enumerate} 
Explica las diferencias en las dos soluciones. Tip: Deriva y gráfica las soluciones analíticas.

% Ref. Kiusaalas 8.2 Shooting method
% Problem Set 8.1
% Problema 18
\item Resuelve el problema de CDF:
\begin{align*}
\nderivada{y}{4} = -2 \, y \, \sderivada{y}
\end{align*}
con $y (0) = \pderivada{y} (0) = 0 \hspace{1.5cm} y (4) = 0 \hspace{1cm} \pderivada{y} (4) = 1$.

% Ref. Kiusaalas 8.3 Finite Differences
% Problem Set 8.2
% Inciso 7
\item Resuelve con el método de diferencias finitas con $m = 20$:
\begin{align*}
\sderivada{y} + 2 \, \pderivada{y} + y = 0, \hspace{1.5cm} y (0) = 0 \hspace{1cm} y (1) = 1
\end{align*}
La solución exacta es $y = x \, e^{1-x}$

% Inciso 8
% \item Resuelve con el método de diferencias finitas con $m = 20$:
% \begin{align*}
% x^{2} \, \sderivada{y} + x \, \pderivada{y} + y = 0, \hspace{1.5cm} y (1) = 0 \hspace{1cm} y (2) = 0.638961
% \end{align*}
% La solución exacta es $y = \sin(\ln x)$.
\end{enumerate}

\end{document}