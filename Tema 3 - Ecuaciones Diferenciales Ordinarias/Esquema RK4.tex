\documentclass[11pt]{article}
%\usepackage[utf8]{inputenc}
\usepackage[latin1]{inputenc}
\usepackage[spanish]{babel}
\usepackage{anysize}
\usepackage{graphicx} 
\marginsize{1cm}{2cm}{0cm}{2cm}  
\title{Esquema de Runge-Kutta de cuarto orden \\ \begin{large}Curso de F�sica Computacional\end{large}}
\author{M. en C. Gustavo Contreras May�n}
\date{ }
\begin{document}
\maketitle
\section{Explicaci�n}
Se presenta un programa de Runge-Kutta de cuarto orden para resolver una ecuaci�n diferencial de primer orden. Antes de ejecutar el programa, el usuario debe de definir la EDO a resolver, en el subprograma \texttt{FUN}. Cuando se ejecuta el programa, se le preguntar� al usuario el n�mero de pasos I, en el intervalo de impresi�n \textit{t}, denotado \texttt{TD}. Entonces, el intervalo de tiempo se hace igual a $h = TD/I$. Tambi�n se le pregunta al usuario, el m�ximo \textit{t} en el que debe evaluarse la soluci�n.
\section{Variables}
H : intervalo de tiempo, \textit{h} \\
F: $f(y,t)$ \\
K1, K2, K3, K4: $k_{1}$, $k_{2}$, $k_{3}$, $k_{4}$, respectivamente \\
Y : y \\
YA : \textit{y} en el subprograma que define la ecuaci�n diferencial \\
X : \textit{t} \\
XA : \textit{t} de la ecuaci�n diferencial en el subprograma \\
XL : valor m�ximo de \textit{t} \\
TD : intervalo de impresi�n de \textit{t} (la soluci�n se imprime despu�s de cada incremento de \textit{t} por TD).
\section{C�digo}
A continuaci�n se indica el c�digo del programa.\\
\texttt{
REAL K1, K2, K3, K4 \\
PRINT * \\
PRINT *, 'Esquema de Runge-Kutta de cuarto orden' \\
PRINT * \\
PRINT *, 'Intervalo de impresi�n de T?' \\
READ *, XPR \\
PRINT *, 'Numero de pasos en un intervalo de impresion?' \\
READ *, I \\
PRINT *, 'Maximo?' \\
READ *, XL \\
\\
!AQUI SE FIJA EL VALOR INICIAL DE LA SOLUCION \\
Y=0 \\
!H ES EL INTERVALO DE TIEMPO \\
H=XPR/I \\
PRINT *, 'H= ', H \\
!SE INICIALIZA EL TIEMPO \\
XP=0 \\
HH=H/2 \\
PRINT * \\
PRINT *,'--------------------------------------------' \\
PRINT *,'	T  \hspace{3cm}			Y  ' \\
PRINT *,'--------------------------------------------' \\
PRINT 82, XP, Y \\
82 FORMAT (1X, F10.6, 7X, 1PE15.6) \\
!AVANZA I PASOS EN CADA INTERVALO DE IMPRESI�N \\
\begin{tabbing}
30 DO \= J= 1,I \\
\> XB=XP \\
\> XP=XP+H \\
\> YN=Y \\
\> XM=XB+HN \\
\> K1=H*FUN(XB,YN) \\
\> K2=H*FUN(YN+K1/2,XM) \\
\> K3=H*FUN(YN+K2/2,XM) \\
\> K4=H*FUN(YN+K3,XP) \\
\> Y=YN+(K1+K2*2+K3*2+K4)/6 \\
\end{tabbing}
END DO \\
PRINT 82, XP, Y \\
IF (XP .LE. XL) GOTO 30 \\
PRINT * \\
PRINT *,'Se ha excedido del limite de X' \\
PRINT * \\
200 PRINT * \\
PRINT *, 'Oprime 1 para continuar, 0 para terminar' \\
READ *, K \\
IF (K .EQ. 1) GOTO 1 \\
PRINT * \\
END PROGRAM rungek4 \\
!+++++++++++++++++++++++++++++++++++++ \\
FUNCTION FUN(X,Y) \\
FUN=X*Y+1 \\
RETURN \\
END 
}\\
\\
N�tese que no est� implementado en el c�digo, el almacenamiento de los datos en un archivo, por lo que habr� que agregarlo y posteriormente trabajar con ese archivo.
\end{document}