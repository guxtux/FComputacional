\documentclass[12pt]{beamer}
\graphicspath{{Imagenes/}{../Imagenes/}}
\usepackage[utf8]{inputenc}
\usepackage[spanish]{babel}
\usepackage{etex}
%\reserveinserts{28}
\usepackage{hyperref}
\usepackage{amsmath}
\usepackage{amsthm}
\usepackage{booktabs}
\usepackage{tabulary}
\usepackage{nccmath}
\usepackage{multicol}
\usepackage{multirow}
\usepackage{graphicx}
\usepackage{color}
\usepackage{tikz}
\usepackage[siunitx]{circuitikz}
\usetikzlibrary{arrows,patterns,shapes}
\usetikzlibrary{decorations.markings}

%\usepackage{theorem}
%\newtheorem{teorema}{Teorema}
%\usepackage{tikz}
%\usetikzlibrary{patterns}


%\usetikzlibrary{positioning}

% Introduce a new counter for counting the nodes needed for circling
\newcounter{nodecount}
% Command for making a new node and naming it according to the nodecount counter
\newcommand\tabnode[1]{\addtocounter{nodecount}{1} \tikz \node (\arabic{nodecount}) {#1};}

% Some options common to all the nodes and paths
%\tikzstyle{every picture}+=[remember picture,baseline]
%\tikzstyle{every node}+=[inner sep=0pt,anchor=base,minimum width=2.2cm,align=center,text depth=.15ex,outer sep=1.5pt]
%\tikzstyle{every path}+=[thick]

\renewcommand*{\multirowsetup}{\centering}
\DeclareGraphicsExtensions{.pdf,.png,.jpg}
\renewcommand {\arraystretch}{1.5}
\mode<presentation>
{
  \usetheme{Warsaw}
  \setbeamertemplate{headline}{}
  \useoutertheme{default}
  \setbeamercovered{invisible}
}
\AtBeginSection[]
{
\begin{frame}<beamer>{Contenido}
\tableofcontents[currentsection]
\end{frame}
}
\usefonttheme[onlymath]{serif}