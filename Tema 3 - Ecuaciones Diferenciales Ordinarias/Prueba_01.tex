\documentclass[letterpaper]{article}
\usepackage[utf8]{inputenc}
%\usepackage[latin1]{inputenc}
\usepackage[spanish]{babel}
\usepackage{geometry}
\usepackage{anysize}
\usepackage{graphicx} 
\usepackage{amsmath}
\usepackage{tikz}
% \usetikzlibrary{patterns}
% \usetikzlibrary{decorations.pathmorphing,patterns}
% \usetikzlibrary{decorations.markings}
% \usetikzlibrary{matrix}
\usepackage{xy}
\usepackage{siunitx}
\usepackage[american,cuteinductors,smartlabels]{circuitikz}
\usetikzlibrary{calc}
\usepackage{color}
%\numberwithin{equation}{list}
% \marginsize{1cm}{2cm}{0cm}{2cm}  

\title{Ejercicios del Examen Tema 3 \\ {\large Curso Física Computacional}}
\author{M. en C. Gustavo Contreras Mayén. \texttt{gux7avo@ciencias.unam.mx}}

\date{ }
\begin{document}

\maketitle
\fontsize{14}{14}\selectfont

A continuación se presenta la lista de ejercicios a resolver que conforman la parte de la evaluación del Tema 3.
\par
Cada ejercicio tiene una calificación de un punto siempre y cuando esté debidamente resuelto.
\par
Revisa cuidadosamente cada ejercicio, ya que deberás de mencionar qué técnica de solución has elegido y justificar su uso, en caso de que el problema indique expresamente la técnica a utilizar, no será necesario que lo expliques.


\end{document}