\documentclass[12pt]{beamer}
\usepackage[utf8]{inputenc}
%\usepackage[latin1]{inputenc}
\usepackage[spanish]{babel}
%\usetheme{Warsaw}
%\usepackage{euler}
\usepackage{hyperref}
\usepackage{amsmath}
\usepackage{amsthm}
\usepackage{multicol}
\usepackage{graphicx}
\usepackage{tikz}
\usepackage{color}
%\usepackage{epstopdf}
\DeclareGraphicsExtensions{.pdf,.png,.jpg}
\renewcommand {\arraystretch}{1.5}
\mode<presentation>
{
  \usetheme{Warsaw}
  \setbeamertemplate{headline}{}
  %\useoutertheme{infolines}
  \useoutertheme{default}
  \setbeamercovered{transparent}
  % or whatever (possibly just delete it)
}
\AtBeginSection[] 
{ 
\begin{frame}<beamer>{Contenido} 
\tableofcontents[currentsection] 
\end{frame} 
} 
\title{Curso de Física Computacional}
\author[]{M. en C. Gustavo Contreras Mayén \\ Fís. Ludmila Palomino}
\date{ }
%\email{curso.fisica.comp@gmail.com}
%\ptsize{10}
\begin{document}
\maketitle
\fontsize{14}{14}\selectfont
\spanishdecimal{.}
\begin{frame}
\frametitle{Contenido}
\tableofcontents[pausesections]
\end{frame}
\section{Objetivos}
\begin{frame}
\frametitle{Objetivos 1}
El propósito del curso es enseñar al estudiante las ideas de computabilidad usadas en distintas áreas de la  física para resolver un conjunto de problemas modelo. A partir de planteamientos analíticos se pretende obtener resultados numéricos reproducibles consistentes, y que predigan situaciones físicas asociadas al problema bajo estudio.
\end{frame}
\begin{frame}
\frametitle{Objetivos 2}
El alumno debe asimilar las ideas básicas del análisis numérico, como son las de estabilidad en el cálculo y la sensibilidad de las respuestas a las perturbaciones en la estructura del problema.El curso también le dará al estudiante capacidad de juicio sobre la calidad de los resultados numéricos obtenidos.
\end{frame}
\begin{frame}
En particular se hará énfasis en la confiabilidad de los resultados respecto a los errores tanto del algoritmo de solución como de las limitaciones numéricas de la computadora. Esta capacidad se adquirirá a lo largo del curso comparando resultados numéricos con otros tipos de análisis, en las regiones en las cuales se pueden llevar ambos a cabo. Por otra parte permitirá al estudiante explorar regiones de comportamiento físico sólo accesibles al cálculo numérico.
\end{frame}
\section{Lugar y horario}
\begin{frame} 
\textbf{Lugar: }Laboratorio de Enseñanza en Cómputo de Física, Edificio Tlahuizcalpan.
\\
\bigskip
\textbf{Horario: } Martes y Jueves de 18 a 21 horas.
\end{frame}
\section{Metodología de Enseñanza}
\begin{frame}
\frametitle{Metodología de Enseñanza}
\textbf{Antes de la clase.}
\\
\medskip
Para facilitar la discusión en el aula, el alumno revisará el material de trabajo que se le proporcionará oportunamente, de tal manera que ya llegará conociendo el tema a desarrollar a la clase. Damos por entendido de que el alumno realizará la lectura.
\end{frame}
\begin{frame}
\frametitle{Metodología de Enseñanza}
\textbf{Durante la clase}
\\
\medskip
Se dará un tiempo para la exposición con diálogo y discusión del material de trabajo con los temas a cubrir durante el semestre. Se busca que sea un curso totalmente práctico por lo que se va a trabajar con los equipos de cómputo del laboratorio.
\end{frame}
\begin{frame}
\frametitle{Metodología de Enseñanza}
\textbf{Después de la clase}
\\
\medskip
El curso requiere que le dediquen al menos el mismo n\'{u}mero de horas de trabajo en casa, es decir, les va a demandar seis horas como mínimo; si cuentan con una experiencia en programación, tienen un paso adelantado, pero si no han programado, se verán en la necesidad de dedicarle más tiempo.
\\
\medskip
Las técnicas de programación que vayan adquiriendo serán el reflejo de su trabajo fuera de clase. En caso de no trabajar o dedicarle el tiempo al curso, se complicará bastante, situación que esperamos no se presente.
\end{frame}
\section{¿Programación?}
\begin{frame}
\frametitle{¿Programación?}
El planteamiento de los algoritmos y la solución de un problema, se expondrá de manera general, propiamente \textcolor{blue}{NO es un curso de programación bajo alg\'{u}n lenguaje en particular}, es importante que cuenten con conocimientos de programación básicos en alg\'{u}n lenguaje com\'{u}n.
\end{frame}
\begin{frame}
\frametitle{Lo que usaremos en el curso}
En el curso utilizaremos Python para programar, se dará un breviario de programación con Python básico como Tema 0, que no será evaluado ni formará parte de la calificación final.
\end{frame}
\begin{frame}
Cuentan con la completa libertad de elegir el lenguaje o software para trabajar durante el curso:
\begin{multicols}{2}
\begin{itemize}
\item Fortran
\item Java
\item C++
\item C
\item Delphi
\item Mathematica
\item Maple
\item Matlab
\item Scilab
\item Octave
\end{itemize}
\end{multicols}
Deberán de entregar su código fuente y el archivo ejecutable.
\end{frame}
\begin{frame}
\frametitle{Software adicional}
Para la ejecución de los algoritmos con códigos de Python, será necesario usar un editor de texto, pueden elegir el de su gusto y preferencia, para aprovechar al máximo la integración de herramientas, usaremos en el curso: \textcolor{red}{Spyder2}.
\\
\bigskip
Además será necesario usar software para graficación de datos, para ello usaremos gnuPlot y la librería matplotlib integrada en Python, pero igual pueden usar el que ya conozcan y manejen bien.
\end{frame}
\begin{frame}
\frametitle{Opcionales}
Pueden traer una laptop para el trabajo en el curso, no es requisito, ya que tenemos equipos suficientes en el laboratorio.
\\
\medskip
Deberán de configurar en su equipo, ya sea en Linux o en Windows, la paquetería necesaria afortunadamente está disponible de manera libre, es decir, es software GNU.
\end{frame}
\section{Temario}
\begin{frame}
\frametitle{Temario del curso}
Llevaremos el temario oficial del curso, que está disponible en la página de la Facultad, haciendo un ajuste en el orden de los temas, siendo entonces:
\end{frame}
\begin{frame}
\frametitle{\textbf{Tema 1: Escalas, condición y estabilidad}}
\begin{enumerate}
\item Introducción.
\item Sistemas numéricos de punto flotante y lenguajes.
\item Dimensiones y escalas.
\item Errores numéricos y su amplificación.
\item Condición de un problema y estabilidad de un método.
\end{enumerate}
\end{frame}
\begin{frame}
\frametitle{\textbf{Tema 2: Operaciones matemáticas básicas}}
\begin{enumerate}
\item Interpolación y extrapolación.
\item Diferenciación numérica.
\item Integración numérica.
\item Evaluación numérica de soluciones.
\end{enumerate}
\end{frame}
\begin{frame}
\frametitle{\textbf{Tema 3: Ecuaciones diferenciales ordinarias}}
\begin{enumerate}
\item Métodos simples.
\item Métodos implícitos y de multipasos.
\item Métodos de Runge-Kutta.
\item Estabilidad de las soluciones.
\item Orden y caos en el movimiento de dos dimensiones.
\end{enumerate}
\end{frame}
\begin{frame}
\frametitle{\textbf{Tema 4: Análisis numérico de problemas matriciales}}
\begin{enumerate}
\item Inversión de matrices y n\'{u}mero de condición.
\item Valores propios de matrices tridiagonales.
\item Discretización de la ecuación de Laplace y métodos iterativos de solución.
\item Solución numérica de ecuaciones diferenciales elípticas en una y dos dimensiones.
\end{enumerate}
\end{frame}
\begin{frame}
\frametitle{\textbf{Tema 5: Problemas clásicos y cuánticos de valores propios}}
\begin{enumerate}
\item Algoritmo de Numerov.
\item Integración de problemas con valores en la frontera.
\item Formulación matricial para problemas de valores propios.
\item Formulaciones variacionales.
\end{enumerate}
\end{frame}
\begin{frame}
\frametitle{\textbf{Tema 6: Simulación computacional}}
\begin{enumerate}
\item Método de Monte Carlo.
\item Dinámica molecular.
\item Otros algoritmos de simulación.
\item Aplicación a problemas de física de interés actual.
\end{enumerate}
\end{frame}
\begin{frame}
\frametitle{\textbf{Tema 7: Ecuaciones de evolución}}
\begin{enumerate}
\item La ecuación de ondas y su discretización en diferencias finitas. Criterio de Courant.
\item La ecuación de Fourier para el calor y su discretización en diferencias finitas. Estabilidad del esquema.
\end{enumerate}
\end{frame}
\section{Bibliografía}
\begin{frame}
\frametitle{Bibliografía}
\fontsize{10}{10}\selectfont
\begin{itemize}
\item Kahaner, D., Moler, C., Nash, S., 1989, Numerical methods and software, Prentice Hall, USA.
\item Klein, A., Godunov, A. Introductory Computational Physics. Cambridge University Press. 2006.
\item Gould, H., Tobochnik, J., 1988, An introduction to computer simulation methods: Applications to physical systems, Addison Wesley Publishing Company, USA.
\item Vesely, F., 1994, Computational physics: An introduction, Plenum Press, USA.
\item Rojas, J.F., Morales, M.A., Rangel, A., Torres, I. Física computacional: una propuesta educativa. Revista Mexicana de Física E 55 (1) 97–111, Junio 2009.
\item Janert, P. K. Gnuplot in action. Understanding data with graphs. Manning Publications Co. 2010.
\item Mejía, C.E., Restrepo, T., Trefftz, C. LAPACK, una colección de rutinas para resolver problemas de álgebra lineal numérica. Universidad Eafit, julio-septiembre, número 123, Universidad Eafit, Medillín, Colombia, pp. 73-80. 2001.
\end{itemize}
\end{frame}
\section{Consideraciones importantes}
\begin{frame}
\frametitle{Consideraciones importantes 1}
\begin{itemize}
\item El cupo para el curso es de 25 alumnos.
\item Se le dará prioridad en la inscripción a los alumnos que están cursando regularmente la carrera, es decir, alumnos que están inscritos en el séptimo semestre.
\item Se les solicita que si consideran quedarse en el curso y se les firma la tira de materias, entendemos que completarán en el curso, si quieren revisar otras opciones de horarios o profesores, se les pide amablemente no requieran la firma, para darle oportunidad a quienes ya están seguros de llevar el curso.
\end{itemize}
\end{frame}
\begin{frame}
\frametitle{Consideraciones importantes 2}
\begin{itemize}
\item Para tener derecho a calificación, se requiere la asistencia mínima del 80\%.
\item Si alguien desea particpar como oyente sin inscripción, podrá hacerlo siempre y cuando haya espacio de trabajo o traiga laptop, pero NO se guardarán calificaciones.
\end{itemize}
\end{frame}
\section{Evaluación}
\begin{frame}
\frametitle{Evaluación}
Se distribuye de la siguiente manera:
\begin{itemize}
\item \textbf{Ejercicios en clase $\mathbf{20\%}$:} para tener derecho a este porcentaje se requiere que el alumno esté presente en la clase, es decir, el ejercicio se entregará en la clase o se dejará para su solución y presentación en la siguiente sesión, en caso de que no asistan y se enteren del ejercicio, se les revisará el trabajo que entreguen, pero no se les tomará en cuenta para el porcentaje, (moraleja: hay que asistir a clase) 
\end{itemize}
\end{frame}
\begin{frame}
%\fontsize{10}{10}\selectfont
\begin{itemize}
\item \textbf{Tareas $\mathbf{40\%}$} : Se entregará una tarea por tema, se les proporcionará de manera adelantada y con fecha de entrega definida, no se reciben tareas extemporáneas, ni por correo.
\item \textbf{Exámenes $\mathbf{40\%}$} : Uno por tema, de tipo teóricos-prácticos. 
\\
\bigskip
\underline{No habrá reposiciones de exámenes parciales.}
\end{itemize}
\end{frame}
\begin{frame}
\frametitle{Examen final}
\textbf{Para tener derecho al examen final:} se deberán de cumplir todos los siguientes puntos:
\begin{itemize}
\item Cumplir con el 80\% de asistencia.
\item Haber presentado todos los exámenes parciales.
\item Haber entregado todas las tareas del curso.
\item Que el promedio final sea menor a $6.0$.
\item No haber acreditado algún examen parcial, así tengan promedio mayor a $6.0$.
\end{itemize}
\end{frame}
\begin{frame}
\frametitle{Muy importante}
Habrá dos rondas de examen final, si en la primera de ellas no se acredita el examen, será posible presentarlo en una segunda y \'{u}ltima ronda, se aclara que para tener derecho al segundo examen, se debe de presentar el primero.
\\
\bigskip
\emph{En caso de haber presentado al menos un examen parcial o haber entregado al menos una tarea, y el promedio final sea menor a 6, la calificación final que se asentará en el acta, será 5. No hay renuncias a calificaciones.}
\end{frame}
\section{Fechas importantes}
\begin{frame}
\frametitle{Fechas importantes}
\begin{itemize}
\item 27 de enero. Inicio del semestre.
\item 14 al 18 de abril, Semana Santa.
\item 1 de mayo, feriado.
\item 15 de mayo, feriado.
\item 23 de mayo. Fin de Semestre.
\item 26 al 30 de mayo. Primera semana de exámenes finales.
\item 2 al 6 de junio. Segunda semana de exámenes finales.
\end{itemize}
\end{frame}
\end{document}