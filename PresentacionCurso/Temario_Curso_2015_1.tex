\documentclass[12pt]{article}
\usepackage[utf8]{inputenc}
\usepackage[spanish]{babel}
\usepackage{amsmath}
\usepackage{amsthm}
\usepackage{graphicx}
\usepackage{color}
\usepackage{float}
\usepackage{multicol}
\usepackage{enumerate}
\usepackage{anyfontsize}
\usepackage{anysize}
\renewcommand{\baselinestretch}{1.5}
\marginsize{1.5cm}{1.5cm}{-1cm}{2cm}
\author{M. en C. Gustavo Contreras Mayén. \texttt{curso.fisica.comp@gmail.com}\\
Fís. Abraham Lima Buendía. \texttt{abraham3081@ciencias.unam.mx}}
\title{Curso de Física Computacional\\{\large Semestre 2015-1 Grupo 8156}}
\date{ }
\begin{document}
%\renewcommand\theenumii{\arabic{theenumii.enumii}}
\renewcommand\labelenumii{\theenumi.{\arabic{enumii}}}
\maketitle
\fontsize{12}{12}\selectfont
\textbf{Objetivos:}
El propósito del curso es enseñar al estudiante las ideas de computabilidad usadas en distintas áreas de la  física para resolver un conjunto de problemas modelo. A partir de planteamientos analíticos se pretende obtener resultados numéricos reproducibles consistentes, y que predigan situaciones físicas asociadas al problema bajo estudio.
\\
\\
El alumno debe asimilar las ideas básicas del análisis numérico, como son las de estabilidad en el cálculo y la sensibilidad de las respuestas a las perturbaciones en la estructura del problema.El curso también le dará al estudiante capacidad de juicio sobre la calidad de los resultados numéricos obtenidos. En particular se hará énfasis en la confiabilidad de los resultados respecto a los errores tanto del algoritmo de solución como de las limitaciones numéricas de la computadora. Esta capacidad se adquirirá a lo largo del curso comparando resultados numéricos con otros tipos de análisis, en las regiones en las cuales se pueden llevar ambos a cabo. Por otra parte permitirá al estudiante explorar regiones de comportamiento físico sólo accesibles al cálculo numérico.
\\
\textbf{Lugar: }Laboratorio de Enseñanza en Cómputo de Física, Edificio Tlahuizcalpan.
\\
\textbf{Horario: } Martes y Jueves de 18 a 21 horas.
\\
\section{Metodología de Enseñanza}
\textbf{Antes de la clase.}
\\
Para facilitar la discusión en el aula, el alumno revisará el material de trabajo que se le proporcionará oportunamente, de tal manera que llegará conociendo el tema a desarrollar a la clase. Damos por entendido de que el alumno realizará la lectura y actividades establecidas.
\\
\textbf{Durante la clase.}
\\
Se dará un tiempo para la exposición con diálogo y discusión del material de trabajo con los temas a cubrir durante el semestre. Se busca que sea un curso totalmente práctico por lo que se va a trabajar con los equipos de cómputo del laboratorio.
\\
\textbf{Después de la clase.}
\\
El curso requiere que le dediquen al menos el mismo número de horas de trabajo en casa, es decir, les va a demandar seis horas como mínimo; si cuentan con una experiencia en programación, tienen un paso adelantado, pero si no han programado, se verán en la necesidad de dedicarle más tiempo.
\\
\\
Las técnicas de programación que vayan adquiriendo serán el reflejo de su trabajo fuera de clase. En caso de no trabajar o dedicarle el tiempo al curso, se complicará bastante, situación que esperamos no se presente.
\\
\\
Se han elaborado guías de apoyo complementarias para la consulta tanto de los conceptos principales de la física involucrada en el problema, así como de programación con \texttt{Python}, de manera que podrán tener una referencia inicial y ya por su cuenta, consultar otros materiales y con ello, lograr un entendimiento completo del problema y su solución.

\section{¿Programación?}
El planteamiento de los algoritmos y la solución de un problema, se expondrán de manera general, propiamente \textcolor{blue}{NO es un curso de programación bajo algún lenguaje en particular}, es importante que cuenten con conocimientos de programación básicos en algún lenguaje común.
\\
\\
En el curso utilizaremos \texttt{Python} para programar, se dará un breviario de programación con \texttt{Python} básico como Tema 0, que no será evaluado ni formará parte de la calificación final.
\\
\\
Cuentan con la completa libertad de elegir el lenguaje o software para trabajar durante el curso:
\begin{multicols}{2}
\begin{itemize}
\item Fortran
\item Java
\item C++
\item C
\item Delphi
\item Mathematica
\item Maple
\item Matlab
\item Scilab
\item Octave
\end{itemize}
\end{multicols}
\subsection{Software adicional}
Para la escritura, revisión y ejecución de los algoritmos con códigos de \texttt{Python}, será necesario usar un editor de texto, pueden elegir el de su gusto y preferencia, para aprovechar al máximo la integración de herramientas, usaremos en el curso con los equipos del laboratorio: Spyder2.
\\
Habrá un conjunto de librerías que complementarán el trabajo que hagamos con \texttt{Python}, en la parte de graficación, utilizaremos una librería con la que resolveremos desde una gráfica básica de dos variables, hasta gráficos más complicados, hay que recordar que nuestro curso se enfoca a la discusión de los resultados obtenidos, no a dejar una gráfica muy detallada y presentable. Existen otros programas para graficar, por lo que también se verá la manera de exportar datos a archivos de texto plano, y también recuperar datos obtenidos de un experimento, ya sea en formato \texttt{*.dat} o \texttt{*.txt} y ocuparlos con \texttt{Python} para analizarlos.
\\
\\
Para que puedan contar con el lenguaje de programación \texttt{Python} en sus equipos de casa o laptop, se proporcionará una guía para instalar \emph{Anaconda}, que es una plataforma que integra el lenguaje, así como Spyder y otras herramientas y librerías que utilizaremos durante el semestre, en la guía podrán revisar el tipo de instalación dependiendo del sistema operativo que usen (Windows, Linux y Mac)
\subsection{Opcionales}
Pueden traer una laptop para el trabajo en el curso, no es requisito, ya que tenemos equipos suficientes en el laboratorio. Deberán de configurar en su equipo, ya sea en Linux o en Windows, la paquetería necesaria afortunadamente está disponible de manera libre, es decir, es software GNU.
%\newpage
\\
\\
Se recomienda que traigan una memoria USB para el respaldo de los archivos que se estarán creando durante las clases, no hay garantía de que los equipos del laboratorio permanezcan en el mismo lugar o sean llevados a mantenimiento. De tal manera que al concluir la clase, tendrán que respaldar todos sus archivos para concluir los ejercicios. No todos los equipos del laboratorio cuentan con salida a internet para que puedan respaldar los archivos en la nube o enviarse por correo.
\section{Temario del curso}
Llevaremos el temario oficial del curso, que está disponible en la página de la Facultad, haciendo un ajuste en el orden de los temas, siendo entonces:
\\
\\  
\textbf{Tema 1: Escalas, condición y estabilidad}
\begin{enumerate}
\item Introducción.
\item Sistemas numéricos de punto flotante y lenguajes.
\item Dimensiones y escalas.
\item Errores numéricos y su amplificación.
\item Condición de un problema y estabilidad de un método.
\end{enumerate}
\textbf{Tema 2: Operaciones matemáticas básicas}
\begin{enumerate}
\item Interpolación y extrapolación.
\item Diferenciación numérica.
\item Integración numérica.
\item Evaluación numérica de soluciones.
\end{enumerate}
\textbf{Tema 3: Ecuaciones diferenciales ordinarias}
\begin{enumerate}
\item Métodos simples.
\item Métodos implícitos y de multipasos.
\item Métodos de Runge-Kutta.
\item Estabilidad de las soluciones.
\item Orden y caos en el movimiento de dos dimensiones.
\end{enumerate}
\textbf{Tema 4: Análisis numérico de problemas matriciales}
\begin{enumerate}
\item Inversión de matrices y número de condición.
\item Valores propios de matrices tridiagonales.
\item Discretización de la ecuación de Laplace y métodos iterativos de solución.
\item Solución numérica de ecuaciones diferenciales elípticas en una y dos dimensiones.
\end{enumerate}
\textbf{Tema 5: Problemas clásicos y cuánticos de valores propios}
\begin{enumerate}
\item Algoritmo de Numerov.
\item Integración de problemas con valores en la frontera.
\item Formulación matricial para problemas de valores propios.
\item Formulaciones variacionales.
\end{enumerate}
\textbf{Tema 6: Simulación computacional}
\begin{enumerate}
\item Método de Monte Carlo.
\item Dinámica molecular.
\item Otros algoritmos de simulación.
\item Aplicación a problemas de física de interés actual.
\end{enumerate}
\textbf{Tema 7: Ecuaciones de evolución}
\begin{enumerate}
\item La ecuación de ondas y su discretización en diferencias finitas. Criterio de Courant.
\item La ecuación de Fourier para el calor y su discretización en diferencias finitas. Estabilidad del esquema.
\end{enumerate}
\section{Bibliografía}
\begin{itemize}
\item Kahaner, D., Moler, C., Nash, S., 1989, Numerical methods and software, Prentice Hall, USA.
\item Klein, A., Godunov, A. Introductory Computational Physics. Cambridge University Press. 2006.
\item Gould, H., Tobochnik, J., 1988, An introduction to computer simulation methods: Applications to physical systems, Addison Wesley Publishing Company, USA.
\item Vesely, F., 1994, Computational physics: An introduction, Plenum Press, USA.
\item Rojas, J.F., Morales, M.A., Rangel, A., Torres, I. Física computacional: una propuesta educativa. Revista Mexicana de Física E 55 (1) 97–111, Junio 2009.
\item Janert, P. K. Gnuplot in action. Understanding data with graphs. Manning Publications Co. 2010.
\item Mejía, C.E., Restrepo, T., Trefftz, C. LAPACK, una colección de rutinas para resolver problemas de álgebra lineal numérica. Universidad Eafit, julio-septiembre, número 123, Universidad Eafit, Medillín, Colombia, pp. 73-80. 2001.
\end{itemize}
\section{Evaluación}
Para tener derecho a calificación, se requiere la asistencia mínima del 80\%.
\\
\\
Los elementos y el peso de la calificación del curso, se distribuyen de la siguiente manera:
\begin{itemize}
\item \textbf{Ejercicios en clase $\mathbf{20\%}$:} para tener derecho a este porcentaje se requiere estar presente en la clase, es decir, el ejercicio se entregará en la clase o se dejará para la siguiente, en caso de que no asistan y se enteren del ejercicio, se les revisará el trabajo que entreguen, pero no se les tomará en cuenta para el porcentaje, (moraleja: hay que asistir a clase) 
\item \textbf{Tareas $\mathbf{40\%}$} : Se entregará una tarea por tema, se les proporcionará de manera adelantada y con fecha de entrega definida, no se reciben tareas extemporáneas, ni por correo.
\item \textbf{Exámenes $\mathbf{40\%}$} : Uno por tema, de tipo teóricos-prácticos. 
\end{itemize}
\underline{No habrá reposiciones de exámenes parciales.}
\section{Examen final}
\textbf{Para tener derecho al examen final:} se deberán de haber presentado todos los exámenes parciales y haber entregado todas las tareas del curso. Habrá dos rondas de examen final, si en la primera de ellas no se acredita el examen, será posible presentarlo en una segunda y última ronda, se aclara que para tener derecho al segundo examen, se debe de presentar el primero.
\\
\\
En caso de haber presentado al menos un examen parcial o haber entregado al menos una tarea, y el promedio final sea menor a 6, la calificación final que se asentará en el acta, será 5. No hay renuncias a calificaciones.
\section{Fechas importantes}
\begin{itemize}
\item Lunes 4 de agosto. Inicio del semestre.
\item Martes 16 de septiembre, día feriado.
\item 22 de noviembre. Fin de Semestre.
\item Viernes 21 de noviembre, fin de semestre.
\item 24 al 28 de noviembre, primera semana de finales.
\item 1 al 5 de diciembre, segunda semana de finales.
\item 15 de diciembre, inicio de vacaciones de fin de año.
\end{itemize}
\end{document}